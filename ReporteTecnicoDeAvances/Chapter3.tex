% Chapter 3

\chapter{Análisis}
\label{Capitulo 3} 

\section{Problemáticas actuales}

\begin{itemize}
	\item Consecuencia 1. Gasto de recursos
	\begin{itemize}
		\item Alto consumo de hojas en comprobantes para alumnos y gesti\'on de caja con aproximadamente 100 a 200 hojas al d\'ia.
		\item Gasto de tinta para impresiones.
		\item Por reglamento se tiene establecido que los comprobantes generados deben ser almacenados por 5 años, ocupando 		espacio f\'isico.
	\end{itemize}
	
	\item Consecuencia 2. Adquisición de servicios
	\begin{itemize}
		\item En ocaciones el alumno pierde su comprobante de pago sin haber efectuado el servicio. 
		\item El alumno no esta supervisado a realizar el pago luego de haber recibido el servicio en área dental.
	\end{itemize}

	\item Consecuencia 3. Tiempos.
	\begin{itemize}
		\item Para efectuar un servicio en CELEX es necesario entregar personalmente un voucher de pago en caja con información personal al reverso.
		\item Para hacer valido cualquier servicio es indispensable presentar físicamente el comprobante de pago generado en caja.
	\end{itemize}

%	\item Consecuencia 4. Pagos en línea.
%	\begin{itemize}
%		\item Se tiene un modo de pago por transferencia Bancomer pero %		solo para algunos servicios.
%	\end{itemize}

\end{itemize}