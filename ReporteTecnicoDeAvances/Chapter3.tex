% Chapter 3

\chapter{Análisis}
\label{Capitulo 3} 

En este capítulo se describirá el análisis realizado para este proyecto, partiendo del planteamiento del problema y seguido del desglose de los problemas actuales estimando también sus consecuencias.\\ 

Así también, se plantean los objetivos que buscamos cumplir para lograr una solución, justificando también nuestro desarrollo y describiendo nuestra propuesta de trabajo.\\

\section{Planteamiento del problema}
A pesar de que en cada una de las áreas de servicios ya se cuenta con un equipo de cómputo y conexión a Internet, ninguna de ellas tiene un sistema que ayude a la gestión de sus pagos, obligando al personal a tener que realizar todos sus procesos de forma manual o ayudarse de las herramientas de ofimática que se les proporcionan.\\

Aunado a esto, se carece de un sistema que comunique directamente a las áreas con el servicio de caja para confirmar y aprobar un pago, sólo se  mantiene una comunicación entre ellas por medio de comprobantes impresos comprometiendo al usuario a asistir forzosamente a caja para poder realizar o recibir cualquier servicio.\\

Lo anterior, se convierte en un problema que puede ser visto desde rubros diferentes. Uno de ellos es el del usuario que necesita disponer de alguno de los servicios, pues en todos los casos, se ve obligado a presentarse directamente en la caja para comprobar el pago independientemente de la forma en cómo lo haya efectuado. Por tanto, si el usuario no le es posible acudir por alguna razón, el pago no podría llegar a ser aprobado desencadenando así más inconvenientes. Hablamos de que en algunos casos el usuario tendrá que realizar de nueva cuenta el pago, o bien, podría retrasarse más de lo esperado para poder recibir el servicio por el cual está pagando.\\

También recae en el usuario toda la responsabilidad de conservar el comprobante de pago emitido por caja hasta que le sea posible llevarlo al área donde lo hará válido. De la mano tenemos que una vez entregado ese comprobante, el usuario se queda sin ninguna garantía de que en algún momento efectúo su pago dejándolo sin ningún recurso para argumentar lo contrario. Esta situación en específico se presenta en las áreas de Fotocopiado, CELEX y Biblioteca.\\

Otro rubro que podemos considerar es el de las áreas que proporcionan los servicios, pues como mencionamos, carecen de un sistema que les apoye en el proceso de pagos de sus servicios. Esto los orilla a idear métodos que creen son los más convenientes, pero en realidad sólo rezagan sus procesos, y en ocasiones se salen de la normatividad impuesta por la ESCOM. Sumado a esto, tenemos que cada una de estas áreas debe de almacenar por cinco años las boletas de pago recibidas, lo que nos habla de carpetas llenas de comprobantes de pago que al final de ese periodo simplemente se van a desechar siendo un gasto de recurso material innecesario tanto en lo económico como en lo ambiental.\\

Un último rubro a considerar será el del departamento de Recursos Financieros, que si bien ya cuentan con un sistema que recaba la información de los pagos que se reciben diariamente, no cuentan con un archivo de pagos que le permita reducir el gasto de recursos materiales, espacios físicos y sobretodo tiempo para el usuario. Decimos esto, porque cada pago recibido implica la impresión de dos boletas de pago, el resguardo de la misma por un periodo de cinco años y además la presencia obligatoria del usuario para confirmar el pago de algún servicio.\\

Si bien el número de impresiones de estos comprobantes por alumno no es significativo, si lo es considerando que los alumnos, personal administrativo, docentes y externos al menos en una ocasión requieren de algún tipo de servicio.\\

Bajo esta perspectiva, hablamos de que en promedio se realizan durante un día 100 impresiones, sin considerar que en periodos de término de semestre o de exámenes a Título de Suficiencia se llegan a imprimir hasta 200 hojas. Considerando la cantidad promedio por día hablamos de que durante un semestre se imprimen aproximadamente 12,000 hojas, contemplando que el tiempo efectivo del semestre son 120 días (4 meses).\\

Si lo trasladamos a datos ambientales, nos daremos cuenta del impacto negativo al medio ambiente que esto tiene. Hablamos de que un árbol sirve para producir 8000 hojas de papel, lo que nos lleva a pensar que en un año se estaría acabando con tres árboles aproximadamente, sumando el uso de 8880 litros de agua para su fabricación teniendo en cuenta que por cada hoja de papel se ocupan 370 cc.\\

Por todo lo anterior, es importante el desarrollo de un sistema que nos permita controlar todos estos problemas, pensando en una solución que contemple el aspecto administrativo y ambiental.\\

\section{Problemáticas actuales}
A raíz del planteamiento del problema, pudimos detectar todas las problemáticas actuales existentes en el proceso de pagos, dándonos la oportunidad de señalar las causas y consecuencias para cada una.\\

\begin{itemize}
	\item {\bf Problema 1. Gasto de recursos materiales: papel y tinta.}\\
	Causas:
	\begin{itemize}
		\item Alto consumo de hojas de papel en la impresión de comprobantes de pago para alumnos y caja.
		\item Gasto de tinta para impresiones.
	\end{itemize}
	Consecuencias:
	\begin{itemize}
		\item Mayor asignación de recursos económicos para la compra de artículos de papelería.
		\item Papel agotado.
		\item Generación de tickets de pago que carecen de una validez oficial.
	\end{itemize}
	
	{\bf \item Problema 2. Espacios físicos insuficientes.}\\
	Causas:
	\begin{itemize}
		\item Todas las boletas de pago emitidas por el SIG@ se deben de almacenar por un periodo de cinco años.
	\end{itemize}
	Consecuencias:
	\begin{itemize}
		\item Estantes llenos de carpetas con los comprobantes de pago.
		\item Oficinas carentes de un orden.
		\item Mala imagen de las oficinas.
	\end{itemize}
	{\bf \item Problema 3. El responsable de la caja de la ESCOM no se encuentra en el lugar.}\\
	Causas:
	\begin{itemize}
		\item Tomó su hora de comida.
		\item Se presenta el cambio de turno.
		\item Ausencia.
		\item Acudió al sanitario.
		\item No cumple con la normatividad de la escuela.
	\end{itemize}
	Consecuencias:
	\begin{itemize}
		\item No se puede efectuar el pago por un servicio, específicamente de Fotocopiado, Biblioteca y Dentales.
		\item ESCOM pierde la oportunidad de recibir ingresos.
		\item Pérdida de tiempo para el usuario que necesita un servicio.
		\item Mala experiencia del usuario.		
	\end{itemize}
	{\bf \item Problema 4. Pérdida del comprobante de pago emitido por el SIG@.}\\
	Causas:
	\begin{itemize}
		\item Caja entrega el comprobante físicamente al usuario.
		\item El usuario no acude el mismo día de emisión a entregar el comprobante en el área.
	\end{itemize}
	Consecuencias:
	\begin{itemize}
		\item El usuario no cuenta con una garantía para reclamar su servicio.
		\item La realización del pago de servicio por segunda ocasión.
	\end{itemize}
	{\bf \item Problema 5. Evasión de pago para servicios dentales.}\\
	Causas:
	\begin{itemize}
		\item El dentista debe brindar el servicio dental al usuario antes de realizar el pago.
	\end{itemize}
	Consecuencias:
	\begin{itemize}
		\item El usuario no realiza su pago en caja.
		\item El usuario puede solicitar de nuevo un servicio sin haber pagado el anterior.
		\item Pérdida de ingresos para la ESCOM.
	\end{itemize}
	{\bf \item Problema 6. Tiempo en el proceso de pago.}\\
	Causas:
	\begin{itemize}
		\item Para hacer válido un servicio en CELEX es necesario entregar personalmente el voucher de pago en caja con información personal al reverso.
		\item Para hacer valido cualquier servicio es indispensable presentar físicamente el comprobante de pago generado en caja.
	\end{itemize}
	Consecuencias:
	\begin{itemize}
		\item El alumno se ve obligado a estar presente en cualquier pago de servicio.
		\item Para el usuario puede tomar más tiempo la conclusión de su pago.
		\item El usuario está limitado a la disposición de las áreas para recibir su pago.
	\end{itemize}
	{\bf \item Problema 7. Los administrativos no siguen las normas establecidas por la institución.}\\
	Causas:
	\begin{itemize}
		\item Abandonan sus puestos de trabajo.
		\item No cumplen con sus horarios de trabajo.
	\end{itemize}
	Consecuencias:
	\begin{itemize}
		\item Procesos inconclusos.
		\item Mala experiencia del usuario.
		\item Genera una imagen negativa de la actual administración.
	\end{itemize}
	{\bf \item Problema 8. El servicio dental carece de un sistema para la gestión de citas y control de servicios.}\\
	Causas:
	\begin{itemize}
		\item Desinterés por mejorar el servicio.
		\item Rechazo al cambio.
		\item Mantener una zona de confort.		
	\end{itemize}
	Consecuencias:
	\begin{itemize}
		\item Traslape de consultas.
		\item Pérdida de tiempo para los usuarios.
		\item Nuevo historial clínico cada vez que se acude a consulta.
		\item No se puede comprobar el pago del usuario.
	\end{itemize}
	{\bf \item Problema 9. Pago de servicios de forma presencial.}\\
	Causas:
	\begin{itemize}
		\item No existe un sistema de pagos electrónico.
		\item La boleta de pago emitida por el SIG@ se entrega físicamente en caja.
		\item Caja comprueba el pago del usuario solicitando físicamente el voucher de pago.
	\end{itemize}
	Consecuencias:
	\begin{itemize}
		\item Pérdida de tiempo del usuario.
		\item El pago de servicios se limita a que el usuario tenga la disponibilidad de acudir a caja.
	\end{itemize}
\end{itemize}

De estas problemáticas, se consideran las primeras ocho dentro del alcance del proyecto el cual se definirá más adelante. La última problemática que hace referencia al pago electrónico aún se sigue considerando como un posible entregable para la segunda parte de este Trabajo Terminal debido a que para la solución de ese punto necesitamos de un contrato con BBVA Bancomer para que se nos de acceso a su aplicación de pagos y podamos añadirla a nuestro sistema. Sin embargo, se nos pide estar declarados como personas físicas con actividad empresarial o como personas morales para poder firmar ese contrato por lo que se expuso esta propuesta al Subdirector Administrativo de la ESCOM y actualmente se encuentra en revisión.\\

\section{Objetivo general}
Desarrollar una aplicaci\'on m\'ovil sobre el sistema operativo Android en conjunto con una aplicaci\'on web para permitir el seguimiento de los pagos realizados en el departamento de recursos financieros de la ESCOM, limitándose al pago de multas de biblioteca, reposici\'on de credencial de biblioteca, servicio de impresiones, dentales y CELEX, con la finalidad de optimizar en tiempo, recursos materiales y espacios f\'isicos el proceso entre los departamentos involucrados y alumnos.\\

\section{Objetivos particulares}
\begin{itemize}
	\item Optimizar recurso material, espec\'ificamente papel durante el proceso de pago de alg\'un servicio, mediante la 				digitalizaci\'on de documentos.
	\item Optimizar el espacio f\'isico de los departamentos involucrados por medio de la reducci\'on de comprobantes de pago, notas de pago e historiales.
	\item Agilizar el proceso de seguimiento a pagos tanto para el alumno, como para las \'areas involucradas.
	\item Desarrollar una herramienta de reporteo derivada del historial de servicios pagados, que les permita a los distintos departamentos llevar a cabo toma de decisiones.
	\item Permitir el acceso a la aplicaci\'on web a aquellos alumnos carentes de tel\'efono inteligente Android para hacer uso de 			  las funciones b\'asicas de este sistema (consulta de servicios, historial y servicios por efectuar).
	\item Brindar una herramienta tecnol\'ogica escalable para futuros desarrollos.
\end{itemize}

\section{Justificaci\'on}
Los servicios proporcionados por la ESCOM siguen un proceso de pago para hacerlos válidos que actualmente involucra una gran demanda de recurso material, específicamente papel y espacios físicos. Además, del tiempo que le toma al usuario realizar este proceso. Con el análisis efectuado nos dimos cuenta que sería de gran ayuda el desarrollo de un sistema que mejore este proceso y que además formalice los actuales procedimientos que llevan a cabo cada una de las áreas involucradas.\\

Es por eso que desarrollaremos un sistema web y móvil que permitirá dar seguimiento a los pagos efectuados en la ESCOM. Esto se logrará a través del anexo del voucher de pago por parte del usuario para enviarlo directamente a caja para que ahí se realice su aprobación y al mismo tiempo se emita la boleta del SIG@, misma que será adjuntada en el sistema para su resguardo. Estos pagos también los podrán consultar las áreas con el objetivo de corroborar los pagos recibidos durante el día corriente. \\

Así, el sistema reducirá el número de impresiones por día, lo que se deduce en un ahorro de recurso material y espacios físicos.\\

También, se traduce en una reducción de tiempo para el usuario, pues no dependerá del hecho de acudir a caja para hacer válido su pago, pues lo podrá realizar desde cualquier lugar en el que tenga acceso a Internet, ya sea por medio de la aplicación web o móvil.\\ 

Con este Trabajo Terminal se pretende apoyar a las áreas de servicios, al departamento de Recursos Financieros, a los alumnos, docentes y externos en todo este proceso.\\

Cabe mencionar que el desarrollo de este sistema considera su escalabilidad a futuro, pensando en que sería un buen complemento para cualquier aplicación enfocada en la gestión de procesos.\\

\section{Descripci\'on de la propuesta}

La propuesta de trabajo se basa en un sistema web y móvil que permitirá dar seguimiento al proceso de pagos en la ESCOM. Se trata de un sistema que le permitirá al usuario seleccionar un concepto de pago del área que requiera y subir su voucher de pago al sistema con el fin de enviarlo directamente a caja para que sea el cajero quien visualice y confirme dicho pago. La confirmación de este pago se hará mediante el anexo al sistema del comprobante emitido por el SIG@. A su vez, se guardará el registro para su consulta posterior por parte del área a la que fue dirigido dicho depósito, o bien, para la revisión de pagos por parte de la contadora o el subdirector administrativo.\\

Tanto el cajero como el usuario podrán recibir notificaciones respecto a los pagos, pues el sistema actuará de forma interactiva mediante el envío de las mismas. Por un lado, el cajero recibirá una notificación en cuanto le sea enviado un pago, por otro lado el usuario recibirá una notificación en cuanto caja haya confirmado el pago o en el peor de los casos lo haya rechazado. De este modo, el cajero sabe de quién y para qué fue efectuado un pago y a su vez, el usuario sabe si su pago ya fue confirmado o presenta algún inconveniente.\\

También, el sistema incluirá un módulo para la gestión de citas de servicios dentales, buscando resolver la problemática planteada anteriormente. Con esto, se logra una mejor administración para el área, además de que se consigue cerrar esa brecha que existe en el pago de esos servicios.\\

\subsection{Alcance del proyecto}
El sistema de “Escomunidad-Servicios” descrito en esta propuesta cumplir\'a con los siguientes requerimientos.
\begin{itemize}
	\item Los administrativos en \'areas de servicios podr\'an visualizar y gestionar los pagos que reciban de caja para realizar un 		  servicio
	\item Los administrativos en \'areas de biblioteca y dentales podr\'an mandar una nota digital de pago a los usuarios.
	\item El contador y el encargado de recursos financieros podr\'an visualizar todos los conceptos de pago e imprimirlos en caso de ser necesarios
	\item El personal de caja podr\'a validar dos tipos de pago, en efectivo y por medio de un voucher de pago.
	\item El personal de caja podr\'a visualizar, aceptar o rechazar los voucher de pago.
	\item El alumno podr\'a visualizar los servicios disponibles y sus precios desde una pagina web o aplicaci\'on m\'ovil
	\item El alumno podr\'a seleccionar entre realiar un pago por transferencia o mandar una nota de pago a caja para realizar el 				  pago en efectivo.
	\item El alumno podr\'a agendar citas de servicio con el \'area de dentales.
	\item Los alumnos podr\'an escoger el metodo de pago que realizen, por transferencia o efectivo.
\end{itemize}

Por lo anterior, se ha propuesto el presente trabajo terminal el cual, después de un exhaustivo análisis por parte del equipo involucrado con los distintos encargados de las ya mencionadas áreas,  ha determinado que lo que se necesita es un sistema de pagos electrónicos el cual permita a un usuario interactuar a distancia con los servicios proporcionados en la ESCOM y que como resultado optimice los procesos explicados en capítulos anteriores así como los recursos que se desperdician actualmente. Principalmente se buscará ahorrar tiempo y esfuerzo por parte de los administrativos, alumnos y externos(comunidad ESCOM) evitando que estos se tengan que transportar hasta la institución para realizar un pago.

Así, este sistema está pensado para ser un desarrollo web y móvil que en conjunto brindarán una solución para la gestión del proceso de pagos en la ESCOM. Si bien, el cumplimiento de los objetivos recaen en gran parte en el desarrollo web, se planea realizar una aplicación móvil híbrida buscando reutilizar código y haciendo más accesible nuestra aplicación. Esto porque al final del desarrollo resultaría en una aplicación móvil que puede ser soportada en sistemas operativos tanto en Android como en iOS.

Las tecnologías contempladas para el desarrollo serán las siguientes:
\begin{itemize}
	\item Spring Boot
	\item Struts 2
	\item Hibernate ORM
	\item PostgreSQL
	\item Bootstrap
	\item JQuery - DataTables
	\item Velocity	
	\item Xamarin
\end{itemize}

En conjunto todas estas tecnologías nos permitirán realizar un proyecto cuya arquitectura está pensada en el Modelo Vista Controlador con el fin de tener una separación entre las capas del sistema.\\

Las bondades de distintos conceptos que nos brinda Spring Framework se notan en la inyección de dependencias que permitirán el desarrollo de una aplicación con código más limpio lo cual se traduce a código más mantenible y en una importante disminución de posible soporte. No sólo Spring Framework nos da la para nuestro desarrollo, también Struts 2 y Hibernate ORM, pues nos facilitan el acoplamiento entre la interfaz gráfica del usuario y la lógica de negocio. Así, el uso de Bootstrap nos permitirá la responsividad natural de la interfaz del usuario complementando el desarrollo pensado para la aplicación móvil puesto que con el uso de Xamarin nos permitirá generar una versión híbrida con llamadas a páginas web. 

\subsection{Interacción con el usuario}
En nuestra arquitectura de sistema es necesaria una comunicación entre distintos elementos que al trabajar en conjunto permitan un correcto funcionamiento. Entre estos elementos consideramos una base de datos para la persistencia de la información y un servidor web para el alojamiento de la pagina, además de todos los actores que se involucran en el proceso de pagos .\ 
\IUfig[1]{gui/a}{}{Arquitectura}
\newpage

\section{Metodología}
Para el desarrollo del sistema utilizaremos la metodologia incremental de Harlan Mills, la cual se basa en la idea de diseñar una implementación inicial, exponerla al comentario del usuario, y luego desarrollarla en sus diversas versiones hasta producir un sistema adecuado. %===========================Referenciar---------------
%Libro de ingeneria de software somerville 

Se tomo en cuenta esta metodología por los siguientes beneficios:
\begin{itemize}
	\item Permite descomponer el proyecto en varios incrementos aislados.
	\item En cada incremento se incorporan los requisitos básicos. 
	\item Es posible realizar un trabajo en paralelo por parte de los integrantes del equipo.
	\item Es sencillo obtener retroalimentación de los directores y coordinadores de área.
\end{itemize}

Así, se plantearon un total de 14 incrementos para el desarrollo completo del sistema. Todos estos se han trabajado de forma paralela hasta esta parte del Trabajo Terminal, siendo factor importante para los avances constantes en la realización de este proyecto.
