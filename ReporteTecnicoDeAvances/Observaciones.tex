En este capitulo se habla de todas las observaciones realizadas por parte de los sinodales y los cambios efectuados para la solución de las mismas.

\section{Observaciones por parte de los sinodales}

Luego de la primera presentación de Trabajo Terminal los sinodales hicieron mención de las siguientes observaciones:

\begin{itemize}
	\item Análisis de requerimientos.
	\item Normalización de la base de datos.
	\item Seguridad para autenticidad de pagos.
\end{itemize}

Para lo cual se realizo lo siguiente:

Un apartado de análisis de requerimientos con un listado de las reglas de negocio, requisitos funcionales y requisitos no funcionales identificados.

La base de datos fue normalizada hasta su tercera forma tomando en cuenta que: \begin{itemize}
	\item {\bf 1 Forma:} \begin{itemize}
		\item Todos los atributos, valores almacenados en las columnas, deben ser indivisibles.
		\item No deben existir grupos de valores repetidos.
	\end{itemize}
	\item {\bf 2 Forma:} Las tablas que están ajustadas a la primera forma normal, y además disponen de una clave primaria formada por una única columna con un valor indivisible, cumplen ya con la segunda forma normal.
	\item {\bf 3 Forma:}  No deben existir dependencias transitivas entre las columnas de una tabla, lo cual significa que las columnas que no forman parte de la clave primaria deben depender sólo de la clave, nunca de otra columna no clave. \cite{Normalizacion}
\end{itemize}

Para poder evitar la duplicidad de pagos se propuso usar como parámetros de seguridad la hora, fecha y número de cajero. Además de que si en algún momento el pago no se encuentra reflejado en el estado de cuenta de bancomer el usuario puede ser identificado rápidamente.

