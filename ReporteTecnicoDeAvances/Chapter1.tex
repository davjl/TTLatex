% Chapter 1 - Introducción
\label{Capitulo 1} % For referencing the chapter elsewhere, use \ref{Chapter1} 
\chapter{Introducci\'on} 

%El presente documento muestra un reporte técnico de avances del Trabajo Terminal 2017-B029, con título ESCOMunidad-Servicios Aplicación móvil de seguimiento a pagos y servicios para la comunidad estudiantil de la ESCOM. Lo mostrado en este documento abarca todo lo realizado para esta primera etapa del Trabajo Terminal.

El presente documento muestra un reporte técnico del Trabajo Terminal 2017-B029, con título ESCOMunidad-Servicios Aplicación móvil de seguimiento a pagos y servicios para la comunidad estudiantil de la ESCOM. Lo mostrado en este documento abarca todo lo realizado durante el periodo completo del Trabajo Terminal.
%----------------------------------------------------------------------------------------

% Define some commands to keep the formatting separated from the content 
\newcommand{\keyword}[1]{\textbf{#1}}
\newcommand{\tabhead}[1]{\textbf{#1}}
\newcommand{\code}[1]{\texttt{#1}}
\newcommand{\file}[1]{\texttt{\bfseries#1}}
\newcommand{\option}[1]{\texttt{\itshape#1}}

%----------------------------------------------------------------------------------------

\section{Presentaci\'on}
El documento muestra los Antecedentes, Análisis, Avances presentados en TT I, Resultados y conclusiones TT I además del Desarrollo elaborado en TT II para la elaboración de un sistema web y móvil pensado para la mejora al proceso de pagos en la ESCOM.\\

%----------------------------------------------------------------------------------------
\section{A quién va dirigido}
Este documento está dirigido principalmente a los directores y sinodales del presente Trabajo Terminal para fines de evaluación. Además de alumnos, profesores y personal administrativo de la Escuela Superior de Cómputo (ESCOM) del Instituto Politécnico Nacional (IPN) que se encuentre interesado en la gestión del proceso de pagos de la ESCOM o en las tecnologías que aquí se abordan, tales como, Java, Struts 2, Spring Boot, Hibernate, PostgreSQL y Android.

\section{Organización}
La sección de Antecedentes abordará una explicación del actual proceso de pagos que se lleva a cabo en las áreas de Cursos Extracurriculares de Lenguas Extranjeras (CELEX), Biblioteca, Fotocopiado y Dentales. Exponemos de forma breve la problemática a resolver y los sistemas y aplicaciones desarrollados hasta el momento que guardan una relación con nuestra propuesta de trabajo.\\

El apartado de Análisis detalla el problema a resolver mediante el planteamiento general del mismo y de todos los problemas específicos que de éste derivan. Mencionamos nuestro objetivo general y todos nuestros objetivos particulares, justificando el por qué de nuestro Trabajo Terminal y describiendo nuestra propuesta de solución.\\

En la sección de Avances presentados en TT I se da el detalle de todas las actividades hechas en la primera presentación derivadas de los antecedentes y el análisis. Mostramos mediante diagramas de caso de uso las funciones más relevantes de nuestro sistema y las relaciones que tienen con cada uno de los actores involucrados. Consideramos también, el modelo de datos que representa el acceso a la información.\\

%un primer modelo de datos para representar nuestro acceso a la información.\\

En el apartado de Resultados y Conclusiones TT I incluimos todo lo que logramos después de haber estudiado a detalle la problemática. Así también, hablamos de las experiencias obtenidas durante la primera fase  de TT.\\

Por último, en la sección de Avances presentados en TT II presentamos todos los entregables que darán fin a este Trabajo Terminal así como una descripción del impacto de nuestro sistema al ser implementado.	

\section{Uso y alcance}

Este documento está elaborado para dar a conocer al lector el actual proceso de pagos en la ESCOM y los problemas que de éste derivan. Además de presentar nuestra propuesta de trabajo como solución a esos problemas.\\

%Todo lo descrito en este documento forma parte de la primera etapa del presente Trabajo Terminal así como también correcciones posteriores a la primera entrega abarcando un análisis de las problemáticas desarrollo de nuestro sistema y pruebas de funcionalidad.\\ 

Por tanto, no debe olvidarse que el ámbito temporal de este documento está sujeto a un análisis realizado a partir de Noviembre del 2017 y que abarca hasta la entrega del Trabajo Terminal II.