% Chapter 1

\chapter{Introducci\'on} % Main chapter title

Con el fin de brindar un ambiente de trabajo pleno y agradable en la ESCOM se proveen diferentes servicios de apoyo a la comunidad como lo son Biblioteca, Sala de impresiones, servicios m\'edicos y/o dentales, papeler\'ia, laboratorios de c\'omputo, entre otros. Sin embargo, para hacer uso de alguno de estos servicios plenamente es necesario realizar un pago en el departamento de recursos financieros (caja). Estos pagos ayudan a la continuidad de los servicios otorgados, adem\'as que representan una parte importante de los ingresos auto generados por parte de la escuela. \newline
Entre los conceptos de pago que destacan debido a la demanda por parte de la comunidad son:

\begin{itemize}
	\item Pago de multas de biblioteca
	\item Reposici\'on de credencial de biblioteca
	\item Servicio de impresiones
	\item Servicios dentales
	\item CELEX
\end{itemize}

Cabe mencionar, que no todos los pagos mencionados anteriormente se realizan directamente en caja, tal es el caso del CELEX en donde el pago se efectúa en alguna instituci\'on financiera, posterior a eso se entrega en caja el comprobante de dicho pago y ah\'i, se otorga al alumno una boleta de pago, la cual tendr\'a que presentar en el departamento correspondiente para que ese servicio se haga v\'alido.\\
Adem\'as, en el caso espec\'ifico del servicio de impresiones es responsabilidad absoluta del alumno conservar el comprobante de pago, v\'alido hasta el momento de la elaboraci\'on de este documento por 15 impresiones. Estas 15 impresiones son v\'alidas para el servicio de copias, impresiones y ploteos. As\'i, esas 15 impresiones de inicio se podr\'ian ver reducidas dependiendo del n\'umero de copias, impresiones o ploteos que se deseen realizar en alg\'un momento. Estas equivalencias entre el n\'umero de impresiones disponibles y las necesarias para poder utilizar alguno de estos servicios son definidas semestre tras semestre por el departamento de recursos financieros. Hasta este momento durante el semestre vigente 2018-1 las tabulaciones están estipuladas de la siguiente manera:
\begin{itemize}
	\item 1 impresi\'on = \$0.58 = 1 impresi\'on en blanco y negro.
	\item 5 impresiones = \$2.90 = 1 impresi\'on a color.
	\item 15 impresiones = \$8.70 = 1 impresi\'on doble carta o 1 impresión 1/4  plotter.
	\item 30 impresiones = \$17.40 = 1 impresi\'on en 1/4 plotter.
	\item 60 impresiones = 1 impresi\'on en plotter completo
	\item O m\'as en m\'ultiplos de 15 impresiones.
\end{itemize}
En el caso del servicio dental no se cuenta con un cat\'alogo establecido en caja, los servicios disponibles para efectuar son confirmados personalmente por el odont\'ologo(a). Sin embargo, cualquiera que \'estos sean, tienen que ser pagados directamente en caja entregando una nota de pago realizada por la dentista luego de haber efectuado el servicio dental.\\
As\'i tambi\'en, el pago de multas de biblioteca y reposición de credencial se llevan a cabo de la misma manera, se acude a biblioteca para recibir una nota de pago donde se especifica el concepto, se realiza el pago en caja, se entrega una boleta de pago y \'esta es otorgada en biblioteca para confirmar y eliminar esa multa del sistema, o bien, hacer válida la reposici\'on de la credencial.\\
Como vemos, son procesos que se llevan a cabo al momento de realizar un pago de servicio, lo que implica tiempo y gasto en recursos materiales (papel) y espacio físico (almacenamiento de boletas de pago). Adem\'as, conlleva una responsabilidad para el alumno el preservar una copia de las boletas o tickets de pago que se le otorgan para poder contar con alguna garant\'ia.

\label{Capitulo 1} % For referencing the chapter elsewhere, use \ref{Chapter1} 

%----------------------------------------------------------------------------------------

% Define some commands to keep the formatting separated from the content 
\newcommand{\keyword}[1]{\textbf{#1}}
\newcommand{\tabhead}[1]{\textbf{#1}}
\newcommand{\code}[1]{\texttt{#1}}
\newcommand{\file}[1]{\texttt{\bfseries#1}}
\newcommand{\option}[1]{\texttt{\itshape#1}}

%----------------------------------------------------------------------------------------

\section{Presentaci\'on}
El siguiente documento presenta el desarrollo de un sistema de integraci\'on de diferentes servicios para la comunidad estudiantil ESCOM, Durante los siguientes cap\'itulos se encontrar\'an temas de an\'alisis, diseño, desarrollo e implementaci\'on que se realizaron durante el desarrollo de la aplicaci\'on web y m\'ovil.
%----------------------------------------------------------------------------------------

\section{Uso y alcance}

Este trabajo terminal esta planteado para la optimizaci\'on de servicios de pago dentro de la Escuela Superior de Computo las cuales comprenden las \'areas de Biblioteca, CELEX, Dentales e Impresiones. Estos servicios se ver\'an ampliamente beneficiados al tener un sistema web que interactu\'e con la gesti\'on de pagos generados en caja para dicho departamento y permita notificarles en tiempo real de quien y para que servicios fueron realizados.
Adem\'as de tener una unificaci\'on de los departamentos ya mencionados, tambi\'en se contara con una aplicaci\'on web y m\'ovil para el servicio del personal escolar o personal externo, en el cual se podrá consultar y pagar m\'ultiples servicios, as\'i mismo como verificar pagos anteriores.

\section {Marco teorico}