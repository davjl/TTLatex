%--------------------------------------------------
\section{Proceso ajustado}
En esta sección presentamos la funcionalidad de nuestro sistema aplicado al negocio.

\IUfig[.8]{gui/app}{}{Flujo con sistema implementado}

\begin{enumerate}
	\item El Usuario accede al sistema web o móvil "Escomunidad-Servicios".
	\item Consulta los servicios disponibles.
	\item En caso de que no sea necesario presentese en el área de servicio continuar en el punto 7.
	\item Acude al área de servicio para poder atender el servicio que requiera.
	\item El Coordinador de área atiende su solictud y genera una nota de pago en el sistema SPEE.
	\item El sistema ''Escomunidad-Servicios'' envía la nota a la cuenta del usuario para que pueda realizar el pago.
	\item En el sistema ''Escomunidad-Servicios'' el usuario selecciona un método de pago.
	\item El usuario tiene dos opciones de pago.
		\begin{enumerate}
			\item Pago en caja
			\begin{enumerate}[I]
				\item Acude a caja a realizar el pago.
			\end{enumerate}
			\item Envio de comprobante digítal por el sistema ''Escomunidad-Servicios''.
			\begin{enumerate}[I]
				\item Realiza el pago en una sucursal bancomer.
				\item Sube en el sistema ''Escomunidad-Servicios'' el voucher de pago.
				\item El sistema ''Escomunidad-Servicios'' envía el voucher a caja.
				\item El cajero Recibe el voucher.
				\item Comprueba que la información y datos sean correctos.
			\end{enumerate}
		\end{enumerate}
	\item El cajero registra el pago en el sistema SIG@.
	\item Adjunta el comprobante SIG@ en el sistema ''Escomunidad-Servicios''.
	\item El sistema ''Escomunidad-Servicios'' envía el comprobante SIG@ al usuario y al área donde fue requerido el pago.
	\item El usuario acude al área donde requirio el servicio (Si no es necesario que se presente saltar este punto).
	\item El coordinador de área consulta los servicios pagados en el día.
	\item Identifica el pago del usuario.
	\item Realiza el servicio.
	\item Confirma el servicio elaborado.
	
\end{enumerate}

%Describa el proceso como será con la introducción del sistema.

%--------------------------------------------------
\section{Procesos actuales}

% - - - - - - - - - - - - - - - - - - - - - - - - -
\subsection{Participantes}

En esta sección se describen los los actores y funciones que participan en los procesos de pagos actuales.

{\bf Encargado de caja:}\begin{itemize}
	\item Descripción: Persona que labora en un horario matutino o vespertino y es encargada de recibir y verificar los pagos para las los diferentes servicios que se brindan en la escuela.
	\item Área: Departamento de recursos financieros.
	\item Responsabilidades:
	\begin{itemize}
		\item Realizar corte de caja dentro de los horarios establecidos.
		\item Atender los pagos en efectivo que se realizen.
		\item Autorizar o rechazar los ticket de pago resividos día.
		\item Registrar el pago en el sistema SIG@.
		\item Hacer entrega del comprobante SIG@.
	\end{itemize}
\end{itemize}

{\bf Contador:}\begin{itemize}
	\item Persona encargada de administrar y dar seguimiento a los estados de cuenta en la ESCOM.
	\item Área: Departamento de recursos financieros.
	\item Responsabilidades:\begin{itemize}
		\item Verificar que las cuentas recibidas esten correctas y en orden.
		\item Atender cualquier anomalía en el registro de pagos.
		\item Llevar un correcto orden administrativo en cuanto a ingresos y egresos.
		\item Almacenar los voucher de pago y comprobante SIG@.
	\end{itemize}
\end{itemize}

{\bf Subdirector Administrativo:}\begin{itemize}
	\item Descripción: Persona encargada de administrar y dar seguimiento a todos los recursos financieros de la ESCOM.
	\item Área: Departamento de recursos financieros.
	\item Responsabilidades:\begin{itemize}
		\item Planificar y organizar los recursos financieros.
		\item Dirigir y controlar los recursos monetarios.
		\item Mantener una supervición de ingresos y egresos.
	\end{itemize}
\end{itemize}

{\bf Usuario:}\begin{itemize}
	\item Descripción: Se define como usuario a la persona que hace uso del sistema para la realización de pagos en las áreas de servicios escom. Existen tres tipos de perfil: alumno, empleado, externo.
	\item Responsabilidades:\begin{itemize}
		\item Realizar un pago por cualquier servicio que lo requiera.
		\item Hacer entrega de las notas de pago y comprobantes SIG@.
	\end{itemize}
\end{itemize}


{\bf Alumno:}\begin{itemize}
	\item Descripción: Persona que pertenece al sistema academico de la institución y es identificado por número de boleta.
\end{itemize}

{\bf Empleado:}\begin{itemize}
	\item Descripción: Persona que labora dentro de la institución y es identificado por su número de empleado.
\end{itemize}

{\bf Externo:}\begin{itemize}
	\item Descripción: Persona que no mantiene ninguna relación laboral o estudiantil dentro de ESCOM y es identificado por solo su CURP.
\end{itemize}

{\bf Coordinadores de área:}\begin{itemize}
	\item Descripción: Persona encargada de atender y coordinar los servicios dentro de su área correspondiente.
	\item Responsabilidades:\begin{itemize}
		\item Almacenar los comprobantes SIG@ que recibe.
		\item En caso de ser necesario otorgar una nota de pago al usuario.
		\item Brindar el servicio solicitado.
	\end{itemize}
\end{itemize}
%Describir, nombre, descripción y responsabilidades.

% - - - - - - - - - - - - - - - - - - - - - - - - -
\subsection{Proceso de pago en servicio dental}

\IUfig[.9]{gui/banco}{}{Flujo actual en servicios dentales}

\begin{enumerate}
	\item El usuario acude sin previa cita al área dental.
	\item Solicita una consulta.
	\item El dentista hace una consulta de revisión.
	\item Si es posible realizar la consulta continuar en 7.
	\item La dentista agenda una cita.
	\item El usuario se presenta el día acordado.
	\item La dentista realiza el servicio.
	\item Realiza una nota de pago al usuario.
	\item El usuario acude a caja a realizar el pago.
	\item El cajero reegistra el pago.
	\item Generá 2 comprobantes SIG@.
	\item Almacena un comprobante y el otro es entregado al usuario.
	\item El usuario entrega el comprobante SIG@ en el área dental.
	\item El dentista almacena el comprobante.
\end{enumerate}

%diagramas y explicación de los procesos: describir las actividades.

% - - - - - - - - - - - - - - - - - - - - - - - - -
\subsection{Proceso de pago para servicios sin nota de pago}
\IUfig[.9]{gui/efectivo}{}{Flujo actual de servicios sin nota de pago}

\begin{enumerate}
	\item El usuario acude al área del servicio que necesita.
	\item Consulta precios y servicios.
	\item El usuario tiene dos opciones de pago (Para el caso de CELEX solo se permite el pago en banco).
		\begin{enumerate}
			\item Pago en caja
			\begin{enumerate}[I]
				\item Acude a caja a realizar el pago.
			\end{enumerate}
			\item Pago en banco.
			\begin{enumerate}[I]
				\item Realiza el pago en una sucursal bancomer.
				\item Entrega el voucher recibido en caja.
				\item El cajero Recibe el voucher.
				\item Comprueba que la información y datos sean correctos.
			\end{enumerate}
		\end{enumerate}
	\item Registra el pago en el sistema SIG@.
	\item Imprime dos comprobantes SIG@.
	\item Almacena un comprobante y el otro es entregado al usuario.
	\item El usuario hace entrega del comprobante donde necesita el servicio.
	\item El coordinador del área realiza el servicio.
	\item Se almacena el comprobante. 
\end{enumerate}

%diagramas y explicación de los procesos: describir las actividades.
