El trabajo a futuro pensado para este sistema se basa en la escalabilidad del sistema pues fue desarrollado con el objetivo de hacer sencilla la integración de nuevos módulos, siempre y cuando se realice el análisis del negocio respectivo ya que a pesar de que el flujo crítico del sistema tiene un comportamiento genérico no es garantía de que se pueda implementar para todas las áreas o negocios que se desee.\\

Así, se plantean cuatro puntos importantes que se podrían implementar en un futuro para complementar el funcionamiento del sistema. Estos puntos son los siguientes:\\

\begin{itemize}
	\item Pago electrónico vía BBVA BANCOMER utilizando su API de desarrollo.
	\item Verificación de los pagos con la API de BBVA BANCOMER.
	\item Comunicación con el sistema SIG@ para la generación automática del comprobante de pago.
	\item Comunicación con el sistema de área central para la obtención automática de la información personal de los usuarios.
\end{itemize}

Estos puntos mencionados podrían mejorar el sistema notablemente pues modificaría el proceso de pagos radicalmente haciendo posible realizar un pago desde cualquier lugar en donde se encuentre el usuario. Valdría la pena pensar en este proyecto académico como una herramienta tecnológica viable para la implementación en los procesos de pago actuales de la ESCOM.