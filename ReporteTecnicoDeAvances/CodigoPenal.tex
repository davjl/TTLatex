\section{Iniciativa del proyecto con adiciones a la ley}

\subsection{Capítulo IV. Falsificación de Documentos en General}

{\bf Artículo 243.} El delito de falsificación se castigará, tratándose de documentos públicos, con prisión de cuatro a ocho años y de doscientos a trescientos sesenta días multa. En el caso de documentos privados, con prisión de seis meses a cinco años y de ciento ochenta a trescientos sesenta días multa.

{\bf Artículo 244.} El delito de falsificación de documentos se comete por alguno de los medios siguientes: 

\begin{itemize}
	\item III. Alterando el contexto de un documento verdadero, después de concluido y firmado, si esto cambiare su sentido sobre alguna circunstancia o punto substancial, ya se haga añadiendo, enmendando o borrando, en todo o en parte, una o más palabras o cláusulas, o ya variando la puntuación; 
	\item IV. Variando la fecha o cualquiera otra circunstancia relativa al tiempo de la ejecución del acto que se exprese en el documento;
	\item VIII. Expidiendo un testimonio supuesto de documentos que no existen; dándolo de otro existente que carece de los requisitos legales, suponiendo falsamente que los tiene; o de otro que no carece de ellos, pero agregando o suprimiendo en la copia algo que importe una variación substancial;
\end{itemize}

{\bf Artículo 245.} Para que el delito de falsificación de documentos sea sancionable como tal, se necesita que concurran los requisitos siguientes: 

\begin{itemize}
	\item I. Que el falsario se proponga sacar algún provecho para sí o para otro, o causar perjuicio a la sociedad, al Estado o a un tercero; 
	\item VIII.- El que a sabiendas hiciere uso de un documento falso o de copia, transcripción o testimonio del mismo, sea público o privado.
\end{itemize}