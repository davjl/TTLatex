
\section{Resultados}

Para esta entrega inicial se logro desarrollar el primer modelo funcional del sistema, en el cual pudimos realizar las primeras interacciones de servicio - usuario - caja.

Como primer resultado notable podemos observar una interacción mucho más fluida y segura para los comprobantes de pago generados en caja hacia las áreas de servicios, ya que al tener un medio de comunicación directo, el comprobante de pago es visualizado por el coordinador del área inmediatamente después de ser efectuado el pago.

Dicho sistema no solo facilita la comunicación entre áreas de servicios, si no además permite a las áreas una reducción de almacenamiento en cuanto a recibos físicos de pago, asimismo se reduce el agrupamiento en filas de usuarios al momento de tramitar un servicio. Con lo cual las áreas permanecerán como un lugar de trabajo mucho más despejadas.

Por parte de los usuarios, estos obtienen un medio de consulta por el cual podrán visualizar los servicios y precios disponibles en cada área de servicio, evitando así la asistencia directa con el coordinador de cada área.

En cuestiones de tiempo, el personal de caja logra evitar la impresión del comprobante de pago, además de eludir todos los problemas que lleguen a presentar al imprimir. Así mismo y más notoriamente los usuarios ya no están obligados a tener que presentar físicamente el comprobante de pago en cada área de servicios, conllevando la posible inasistencia por parte del usuario.

%Como parte del desarrollo del presente Trabajo Terminal, se acordaron repetidas reuniones de trabajo con las distintos respresantes de las áreas con el fin de comprender el negocio correspondiente y la problemática actual del proceso de gestión de pagos autogenerados de la Escuela Superior de Cómputo.

%Derivado de este análisis en esta primer parte de Trabajo Terminal encontramos varias debilidades en las distintas áreas involucradas para nuestro proyecto, siendo así intermediarios entre las mismas con el fin de lograr con ello un convencimiento parcial entre los distintos actores.

%Después de efectuar un detallado análisis, se logró producir un compendio de 55 interfaces de usuario con la aprobación de los involucrados en el área administrativa y en el equipo de desarrollo, desglosando de esto, 45 casos de uso alucivos a las mismas, de los cuales se presentan los más relevantes.

%También, se logró producir un modelo de datos capaz de realizar una gestión de pagos simple con fines de presentación, dándonos la pauta para robustecer el sistema en la siguiente entrega de Trabajo Terminal.

\section{Conclusiones}
Al inicio del Trabajo Terminal no se tenia un enfoque preciso sobre el modo de operación en caja y áreas de servicios, solo se contaba con el entendimiento propio de cada integrante al realizar la solicitud de un servicio, por lo cual el entender todo proceso que conlleva este sistema fue de lo más importante.

La realización de estas entrevistas fue de lo más importante para la comprención, así como también fue el proceso con más dificultad de desarrollo, ya que en algunas ocaciones se presentaron problematicas para lograr contactar con los coordinadores de las áreas, o poder acordar tiempos en los cuales se logrará realizar la entrevista.

Gracias a estas entrevistas fue que nos dimos cuenta del nuevo alcance y objetivo que estaba tomando nuestro Trabajo Terminal, ya que las áreas de servicios más que necesitar una aplicación móvil, necesitaban de un sistema web que pudiera ayudarlos a administrar los comprobantes de pago, así como de otorgar funcionalidades especificas de cada área.
De esta manera comprendimos la magnitud del proyecto y como es que a medida de cada entrevista las funcionalidades y caracteristicas aumentabán en gran cantidad. Estas circunstancias nos obligaron a realizar un reemplantamiento de alcances, los cuales luego de hablarlo con nuestros directores, se acordo por realizar solo las funcionalidades de mayor impacto, así como solo desarrollar una apicación hibrida de consulta para consumo de los actores que lo requieran, esto con fin de no salir de los acuerdos iniciales sobre la aplicación movil.

Durante el desarrollo se opto por hacer uso de un lenguaje de programación en el que todos tuviéramos al menos el conocimiento minimo de desarrollo, con una estructura de Modelo-Vista-Controlador (MVC) en conjunto del uso framewoks para una mayor organización.

Para conluir nos gustaría agregar que durante el desarrollo de este sistema, pudimmos identificar un alcance de mayores proporciones en el cual no solo se tome este sistema para uso especifico de pagos, si no además el poder agregar un uso mucho más extenso que conlleve la integración de departamentos académicos, servicios de educación continúa, servecios de papeleria, una gestión de listas de copias otorgadas por profesores, proyectos de Trabajo Terminal que tengan como finalidad otortgar un medio de apoyo a la comunidad, entre otras cosas.