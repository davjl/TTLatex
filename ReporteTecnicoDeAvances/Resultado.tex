
\section{Resultados}
Derivado del análisis pudimos obtener un total de 55 interfaces gráficas de usuario aprobadas por cada una de las áreas y un total de 45 casos de uso con las funciones de nuestro sistema.\\

Se obtuvo un primer modelo de base de datos para poder manejar nuestro acceso a la información durante el desarrollo de la primera parte del sistema.\\

En paralelo a lo anterior, pudimos elaborar la infraestructura de nuestro sistema utilizando las herramientas tecnológicas que se mencionaron.\\

Logramos programar para nuestro sistema 19 casos de uso y un total de 25 interfaces gráficas. Esto representa aproximadamente un 30\% del total de nuestro sistema.\\

Con lo anterior, pudimos observar una interacción mucho más fluida y segura con los comprobantes de pago generados en caja hacia las áreas de servicios, ya que al tener un medio de comunicación directo, el comprobante de pago es visualizado por el coordinador del área inmediatamente después de ser efectuado el pago.\\

\section{Conclusiones}
Al inicio del Trabajo Terminal no se tenía un enfoque preciso sobre el modo de operación en caja y áreas de servicios, sólo se contaba con el entendimiento propio de cada integrante al realizar un proceso de pago. Esto se volvió un punto crucial para el inicio de este proyecto.

La realización de las mesas de trabajo fue de lo más importante para la comprensión del proceso de pagos. Esto nos llevó a tratar con diversos tipos de personas y entender las peticiones que cada uno realizaba logrando tener una solución en común para todas de la manera más asertiva posible.\\

Gracias a estas reuniones fue que nos dimos cuenta del nuevo alcance y objetivo que estaba tomando nuestro Trabajo Terminal, ya que las áreas de servicios más que necesitar una aplicación móvil, necesitaban de un sistema web que pudiera ayudarlos a administrar los comprobantes de pago, así como de otorgar funcionalidades especificas de cada área.\\

De esta manera comprendimos la magnitud del proyecto y cómo es que a medida de cada entrevista las funcionalidades y características aumentaban en gran cantidad. Estas circunstancias nos obligaron a realizar un nuevo planteamiento de alcances, los cuales al llegar a un acuerdo con nuestros directores, se acordó realizar sólo las funcionalidades de mayor impacto, así como desarrollar una aplicación móvil híbrida buscando la reutilización de 
código.\\

Lo anterior, nos llevó a tomar algunas decisiones importantes dentro del equipo de trabajo, tales como la definición del lenguaje de programación, la forma de trabajo y la delegación de actividades para cada uno de los integrantes.\\

Por último, es importante mencionar que esta primera etapa de trabajo nos abrió el panorama para estar convencidos de que no sólo podría funcionar nuestro desarrollo para la mejora al proceso de pagos sino también para la mejora de los procesos generales de la ESCOM o de alguna otra institución que tenga debilidades en la gestión de sus procesos.