Como parte del desarrollo del presente Trabajo Terminal, se acordaron repetidas reuniones de trabajo con las distintos respresantes de las áreas con el fin de comprender el negocio correspondiente y la problemática actual del proceso de gestión de pagos autogenerados de la Escuela Superior de Cómputo.

Derivado de este análisis en esta primer parte de Trabajo Terminal encontramos varias debilidades en las distintas áreas involucradas para nuestro proyecto, siendo así intermediarios entre las mismas con el fin de lograr con ello un convencimiento parcial entre los distintos actores.

Después de efectuar un detallado análisis, se logró producir un compendio de 55 interfaces de usuario con la aprobación de los involucrados en el área administrativa y en el equipo de desarrollo, desglosando de esto, 45 casos de uso alucivos a las mismas, de los cuales se presentan los más relevantes.

También, se logró producir un modelo de datos capaz de realizar una gestión de pagos simple con fines de presentación, dándonos la pauta para robustecer el sistema en la siguiente entrega de Trabajo Terminal.







