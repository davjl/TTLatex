\section{Estimación por puntos de casos de uso (Karner)}
En esta sección se estudia por medio del método de estimación con puntos de casos de uso el posible costo de y esfuerzo realizado en este trabajo, este método nos permite realizar estimaciones a partir de un modelo orientado a objetos, en específico el modelo de casos de uso. En la primera parte se establecerán las métricas necesarias para llevar a cabo este procedimiento y una breve descripción del significado de algunos términos, en la segunda parte se aplicarán estas métricas a este trabajo y finalmente se mostrarán algunas conclusiones.

\begin{table}[htbp]
	\begin{center}
		\begin{tabular}{|p{30mm}|p{75mm}|p{25mm}|}
			\hline
			\multicolumn{3}{|c|}{Peso de los actores} \\
			\hline
			{\bf Tipo de actor} & {\bf Descripción} & {\bf Factor}\\
			\hline
			Simple & Otro sistema que interactúa con el sistema a desarrollar mediante una interfaz de programación (API). & 1 \\
			\hline
			Medio & Otro sistema interactuando a través de un protocolo (ej. TCP/IP) o una persona interactuando a través de una interfaz en modo texto. & 2 \\
			\hline
			Complejo & Una persona que interactúa con el sistema mediante una interfaz gráfica (GUI).& 3 \\
			\hline
		\end{tabular}
		\caption{Tipo y peso de actores.}
		\label{tabla:pesoActores}
	\end{center}
\end{table}

\begin{table}[htbp]
	\begin{center}
		\begin{tabular}{|p{40mm}|p{50mm}|p{40mm}|}
			\hline
			\multicolumn{3}{|c|}{Peso de los casos de uso} \\
			\hline
			{\bf Tipo de caso de uso} & {\bf Descripción} & {\bf Factor}\\
			\hline
			Simple & 3 transacciones o menos. & 5 \\
			\hline
			Medio & 4 a 7 transacciones. & 10 \\
			\hline
			Complejo & Más de 7 transacciones. & 15 \\
			\hline
		\end{tabular}
		\caption{Tipo y peso de los casos de uso.}
		\label{tabla:pesoCU}
	\end{center}
\end{table}

\begin{table}[htbp]
	\begin{center}
		\begin{tabular}{|p{40mm}|p{50mm}|p{40mm}|}
			\hline
			\multicolumn{3}{|c|}{Escalas de estimación de factores técnicos} \\
			\hline
			{\bf Tipo de caso de uso} & {\bf Descripción} & {\bf Factor}\\
			\hline
			Irrelevante & 3 transacciones o menos. & 5 \\
			\hline
			Medio & 4 a 7 transacciones. & 10 \\
			\hline
			Esencial & Más de 7 transacciones. & 15 \\
			\hline
		\end{tabular}
		\caption{Factores técnicos.}
		\label{tabla:pesoFactoresTecnicos}
	\end{center}
\end{table}


\begin{table}[htbp]
	\begin{center}
		\begin{tabular}{|p{40mm}|p{50mm}|p{40mm}|}
			\hline
			\multicolumn{3}{|c|}{Escalas de estimación de factores técnicos}\\
			\hline
			{\bf Tipo de caso de uso} & {\bf Descripción} & {\bf Factor}\\
			\hline
			Irrelevante & 3 transacciones o menos. & 5 \\
			\hline
			Medio & 4 a 7 transacciones. & 10 \\
			\hline
			Esencial & Más de 7 transacciones. & 15 \\
			\hline
		\end{tabular}
		\caption{Factores técnicos.}
		\label{tabla:pesoFactoresTecnicos}
	\end{center}
\end{table}