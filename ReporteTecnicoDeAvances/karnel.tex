\subsection{Estimación por puntos de casos de uso (Karner)}
En esta sección se estudia por medio del método de estimación con puntos de casos de uso el posible costo de y esfuerzo realizado en este trabajo, este método nos permite realizar estimaciones a partir de un modelo orientado a objetos, en específico el modelo de casos de uso. En la primera parte se establecerán las métricas necesarias para llevar a cabo este procedimiento y una breve descripción del significado de algunos términos, en la segunda parte se aplicarán estas métricas a este trabajo y finalmente se mostrarán algunas conclusiones.

\begin{table}[htbp]
	\begin{center}
		\begin{tabular}{|p{30mm}|p{75mm}|p{25mm}|}
			\hline
			\multicolumn{3}{|c|}{Peso de los actores} \\
			\hline
			{\bf Tipo de actor} & {\bf Descripción} & {\bf Factor}\\
			\hline
			Simple & Otro sistema que interactúa con el sistema a desarrollar mediante una interfaz de programación (API). & 1 \\
			\hline
			Medio & Otro sistema interactuando a través de un protocolo (ej. TCP/IP) o una persona interactuando a través de una interfaz en modo texto. & 2 \\
			\hline
			Complejo & Una persona que interactúa con el sistema mediante una interfaz gráfica (GUI).& 3 \\
			\hline
		\end{tabular}
		\caption{Tipo y peso de actores.}
		\label{tabla:pesoActores}
	\end{center}
\end{table}

\begin{table}[htbp]
	\begin{center}
		\begin{tabular}{|p{40mm}|p{50mm}|p{40mm}|}
			\hline
			\multicolumn{3}{|c|}{Peso de los casos de uso} \\
			\hline
			{\bf Tipo de caso de uso} & {\bf Descripción} & {\bf Factor}\\
			\hline
			Simple & 3 transacciones o menos. & 5 \\
			\hline
			Medio & 4 a 7 transacciones. & 10 \\
			\hline
			Complejo & Más de 7 transacciones. & 15 \\
			\hline
		\end{tabular}
		\caption{Tipo y peso de los casos de uso.}
		\label{tabla:pesoCU}
	\end{center}
\end{table}

\begin{table}[htbp]
	\begin{center}
		\begin{tabular}{|p{40mm}|p{50mm}|p{40mm}|}
			\hline
			\multicolumn{3}{|c|}{Escalas de estimación de factores técnicos}\\
			\hline
			{\bf Tipo de caso de uso} & {\bf Descripción} & {\bf Factor}\\
			\hline
			Irrelevante & 3 transacciones o menos. & 5 \\
			\hline
			Medio & 4 a 7 transacciones. & 10 \\
			\hline
			Esencial & Más de 7 transacciones. & 15 \\
			\hline
		\end{tabular}
		\caption{Factores técnicos.}
		\label{tabla:pesoFactoresTecnicos}
	\end{center}
\end{table}

{\bf UUCP} = {\bf UAW} + {\bf UUCW} \\

Donde: \begin{itemize}
	\item {\bf UUCP:} Puntos de Casos de Uso sin ajustar. 
	\item {\bf UAW:} Factor de Peso de los Actores sin ajustar.
	\item {\bf UUCW:} Factor de Peso de los Casos de Uso sin ajustar.
\end{itemize} 


\begin{table}[htbp]
	\begin{center}
		\begin{tabular}{|p{50mm}|p{50mm}|}
			\hline
			{\bf Actores} & {\bf Factor}\\
			\hline
			Alumno & 3.\\
			\hline
			Profesor & 3.\\
			\hline
			Externo & 3. \\
			\hline
			Cajero & 3 \\
			\hline
			Subdirector administrativo & 3 \\
			\hline
			Contador & 3 \\
			\hline
			Coordinador dental & 3 \\
			\hline
			Coordinador CELEX & 3 \\
			\hline
			Coordinador fotocopiado & 3 \\
			\hline
			Coordinador biblioteca & 3 \\
			\hline
		\end{tabular}
		\caption{Peso actores.}
		\label{tabla:pesoActoress}
	\end{center}
\end{table}

\newpage
El peso otorgado a todos los casos de uso fue de cinco ya que el desarrollo de cada uno de ellos implicaba tres transacciones o menos a excepción del caso de uso cargar pagos ya que cuenta con más de cuatro transacciones obteniendo así:

\begin{itemize}
	\item {\bf UUCW} = 215.
	\item {\bf UUCP} = {\bf UAW} + {\bf UUCW} = 30 + 215 = 245.
	\item {\bf UCP} = {\bf UUCP} x {\bf TCF} x {\bf EF}.
\end{itemize}

Donde : \begin{itemize}
	\item {\bf UCP:} Puntos de Casos de Uso ajustados.
	\item {\bf UUCP:} Puntos de Casos de Uso sin ajustar.
	\item {\bf TCF:} Factor de complejidad técnica.
	\item {\bf EF:} Factor de ambiente.
\end{itemize}

{\bf TCF} = 0.6 + 0.01 * $\sum$($Peso_{i}$ * $ValorAsignado_{i}$)\\

{\bf TCF} = 0.6 + 0.01 * (3*2 + 4*1 + 4*1 + 3*1 + 4*1 + 2*0.5 +  5*0.5 + 3*2 + 3*1 + 3*1 + 4*1 + 3*1 + 1*1) = 0.6 + 0.01 (44.5) = 1.045\\
\newpage
\begin{table}[htbp]
	\begin{center}
		\begin{tabular}{|p{20mm}|p{30mm}|p{20mm}|p{30mm}|}
			\hline
			{\bf Factor} & {\bf Descripción} & {\bf Peso} & {\bf Comentario}\\
			\hline
			T1 & Sistema distribuido & 3 & La aplicación utilizará varios O.Web\\
			\hline
			T2 & Objetivos de Comportamiento o tiempo de respuesta & 4 & \\
			\hline
			T3 & Eficacia del Usuario Final & 4 & Debe asegurarse que la aplicación proporcione una alta eficacia para los usuarios \\
			\hline
			T4 & Procedimiento Interno Complejo & 3 & No muchos cálculos complejos \\
			\hline
			T5 & El código debe ser reutilizable & 4 & \\
			\hline
			T6 & Facilidad de instalación & 2 & \\
			\hline
			T7 & Facilidad de Uso & 5 & Es muy necesario crear interfaz que sean fáciles de aprender, memorizar, que sean agradables y comprensibles. \\
			\hline
			T8 & Portabilidad & 3 & Es necesario que sea compatible con los distintos navegadores Web existentes. \\
			\hline
			T9 & Facilidad de cambio & 3 & \\
			\hline
			T10 & Concurrecia & 3 & \\
			\hline
			T11 & Incluye objetivos especiales de seguridad & 4 & Seguridad alta. \\
			\hline
			T12 & Provee Acceso directo a terceras partes & 3 & Los Usuarios Web tienen acceso \\ 
			\hline
			T13 & Se requiere facilidades especiales de entrenamiento a usuarios & 1 & \\
			\hline
		\end{tabular}
		\caption{Calculo de factores técnicos.}
		\label{tabla:factores}
	\end{center}
\end{table}

{\bf EF} = 1.4 – 0.03 * $\sum$($Peso_{i}$ * $Valor Asignado_{i}$)\\

{\bf EF} = 1.4 – 0.03 * $\sum$(1.5 * 3 + 0.5*3 + 1*4 + 0.5*3 + 1*5 + 2 -1 -4) = 1.4 – 0.03 (12.5) = 1.025
\newpage
\begin{table}[htbp]
	\begin{center}
		\begin{tabular}{|p{20mm}|p{30mm}|p{20mm}|p{20mm}|p{30mm}|}
			\hline
			{\bf Factor} & {\bf Descripción} & {\bf Peso} & {\bf Valor asignado} & {\bf Comentario}\\
			\hline
			E1 &Familiaridad con el modelo de proyecto utilizado. & 1.5 & 3 & El equipo está medianamente familizado con el modelo \\
			\hline
			E2 & Experiencia en la aplicación. & 0.5 & 3 & El equipo no ha trabajado mucho con esta aplicación. \\
			\hline
			E3 & Experiencia en orientación a objetos. & 1 & 4 & La mayoría del equipo programa Orientado a Objetos.\\
			\hline
			E4 & Capacidad del analista líder. & 0.5 & 3 & No se tiene especialista \\
			\hline
			E5 & Motivación & 1 & 5 & El equipo esta muy motivado \\
			\hline
			E6 & Estabilidad de los requerimientos & 2 & 1 & Los requerimientos son muy volatiles. \\
			\hline
			E7 & Personal part-time & -1 & 1 & \\
			\hline
			E8 & Dificultad del lenguaje de programación & -1 & 4 & Se usa el lenguaje de programacíon Java \\
			\hline
		\end{tabular}
		\caption{Factores de entorno}
		\label{tabla:entorno}
	\end{center}
\end{table}

{\bf Cálculo de los Puntos de Caso de Uso ajustados}\\
{\bf UCP} = 245 * 1.045 * 1.025 = 262.4256\\
{\bf E} = {\bf UCP} * {\bf CF} \\

Donde:\begin{itemize}
	\item {\bf E} es el Esfuerzo.
	\item {\bf CF} es Factor de Conversión.
	\item {\bf UCP} es Casos de Uso Ajustados.
\end{itemize}

El Factor de Conversión será 20 horas-hombre 

\begin{table}[htbp]
	\begin{center}
		\begin{tabular}{|p{50mm}|p{50mm}|}
			\hline
			{\bf Actividad} & {\bf Porcentaje} \\
			\hline
			Análisis & 10 \\
			\hline
			Diseño & 20 \\
			\hline
			Programación & 40 \\
			\hline
			Prueba & 15 \\
			\hline
			Sobrecarga (Otras actividades) & 15 \\
			\hline
		\end{tabular}
		\caption{Porcentaje de actividades}
		\label{tabla:entornoef}
	\end{center}
\end{table}
\newpage
Nota: Estos porcentajes pueden variase considerando la experiencia que se tenga en el desarrollo y se debe justificar la variación. 

{\bf E} = 262.4256 * 20 = 5248.51 Horas-Hombre

\begin{table}[htbp]
	\begin{center}
		\begin{tabular}{|p{50mm}|p{50mm}|}
			\hline
			{\bf Actividad} & {\bf Porcentaje} \\
			\hline
			Análisis & 524.851 \\
			\hline
			Diseño & 1049.702 \\
			\hline
			Programación & 2099.404 \\
			\hline
			Prueba & 787.2765 \\
			\hline
			Sobrecarga (Otras actividades) & 787.2765 \\
			\hline
			{\bf TOTAL} & {\bf 3608.01}\\
			\hline
		\end{tabular}
		\caption{Porcentaje de actividades}
		\label{tabla:entorns}
	\end{center}
\end{table}