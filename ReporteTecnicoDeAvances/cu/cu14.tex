% \IUref{IUAdmPS}{Administrar Planta de Selección}
% \IUref{IUModPS}{Modificar Planta de Selección}
% \IUref{IUEliPS}{Eliminar Planta de Selección}

% 


% Copie este bloque por cada caso de uso:
%-------------------------------------- COMIENZA descripción del caso de uso.

\begin{UseCase}{CU14}{Agendar cita}{
		Permite al usuario elaborar una cita de consulta en área dental.
	}
		\UCitem{Versión}{0.1}
		\UCitem{Autor}{David De Jesús López.}
		\UCitem{Estatus}{Edición.}
		\UCitem{Fecha de último estatus}{7 de septiembre de 2018.}
		\UCitem{Actor}{Usuario.}
		\UCitem{Propósito}{Mantener un medio de organizado en consultas dentales.}
		\UCitem{Entradas}{Ninguna.}
		\UCitem{Salidas}{Cita agendada}
		\UCitem{Precondiciones}{Haber una hora disponible de cita.}
		\UCitem{Postcondiciones}{Ninguna.}	
		\UCitem{Reglas de Negocio}{Ninguna.}
		\UCitem{Errores}{Ninguno.}
		\UCitem{Tipo}{Primario.}
	\end{UseCase}

	\begin{UCtrayectoria}{Principal}
		\UCpaso[\UCactor] Da clic en el botón \IUbutton {Agendar cita dental} del menú principal de la pantalla \IUref{IU5.1}{Bienvenida usuarios} 
		\UCpaso Muestra la pantalla \IUref{IU21}{Agendar cita}.
		\UCpaso [\UCactor] Selecciona una hora disponible de cita. \Trayref{A}
		\UCpaso Agenda el día y hora.
	\end{UCtrayectoria}
	
	\begin{UCtrayectoriaA}{A}{El usuario selecciona una hora no valida.}
		\UCpaso[\UCactor] Selecciona una hora no disponible de cita
		\UCpaso Muestra el mensaje \MSGref{MSG58}{Operación no completada}.
	\end{UCtrayectoriaA}
%-------------------------------------- TERMINA descripción del caso de uso.