% \IUref{IUAdmPS}{Administrar Planta de Selección}
% \IUref{IUModPS}{Modificar Planta de Selección}
% \IUref{IUEliPS}{Eliminar Planta de Selección}

% 


% Copie este bloque por cada caso de uso:
%-------------------------------------- COMIENZA descripción del caso de uso.
\begin{UseCase}{CU10}{Pagar nota de pago en caja}{
		Permite al usuario realizar el pago de un servicio en caja. 
}
		\UCitem{Versión}{0.1}
		\UCitem{Autor}{Roberto Mendoza Saavedra.}
		\UCitem{Estatus}{Edición.}
		\UCitem{Fecha de último estatus}{8 de septiembre de 2018.}
		\UCitem{Actor}{Usuario.}
		\UCitem{Propósito}{Que el usuario siga contando con la opción de pago en caja.}
		\UCitem{Entradas}{Nota de pago.}
		\UCitem{Salidas}{\MSGref{MSG7}{Realiza el pago en caja}.}
		\UCitem{Precondiciones}{
			\begin{itemize}
				\item Tener una nota de pago.
			\end{itemize}}
		\UCitem{Postcondiciones}{Ninguno.}
		\UCitem{Reglas de Negocio}{Ninguno.}
		\UCitem{Errores}{\MSGref{MSG64}{No fue posible enviar nota}.}
		\UCitem{Tipo}{Primario}
	\end{UseCase}		

	\begin{UCtrayectoria}{Principal}
		\UCpaso[\UCactor] Da clic en el icono \raisebox{-\mydepth}{\fbox{\includegraphics[height=\myheight]{icons/pagoCaja}}} en la pantalla \IUref{IU19}{Gestionar notas de pago usuarios}
		\UCpaso Dirige a la pantalla \IUref{IU17}{Pagar nota de pago en caja}
		\UCpaso[\UCactor] Selecciona la nota de pago.
		\UCpaso Se manda la nota de pago a caja.
		\UCpaso[\UCactor] El usuario puede pasar a caja a realizar el pago.
	\end{UCtrayectoria}
		
		
%-------------------------------------- TERMINA descripción del caso de uso.
