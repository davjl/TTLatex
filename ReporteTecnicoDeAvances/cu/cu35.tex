% \IUref{IUAdmPS}{Administrar Planta de Selección}
% \IUref{IUModPS}{Modificar Planta de Selección}
% \IUref{IUEliPS}{Eliminar Planta de Selección}

% 


% Copie este bloque por cada caso de uso:
%-------------------------------------- COMIENZA descripción del caso de uso.

\begin{UseCase}{CU35}{Reducir impresiones}{
		Permite al coordinador de fotocopiado reducir el número de impresiones al usuario.
	}
		\UCitem{Versión}{0.1}
		\UCitem{Autor}{David De Jesús López.}
		\UCitem{Estatus}{Edición.}
		\UCitem{Fecha de último estatus}{9 de abril de 2018.}
		\UCitem{Actor}{Coordinador fotocopiado.}
		\UCitem{Propósito}{Permitir la funcionalidad de reducir impresiones al usuario.}
		\UCitem{Entradas}{Ninguna.}
		\UCitem{Salidas}{Impresiones reducidas a la cuenta del usuario}
		\UCitem{Precondiciones}{Tener impresiones disponibles.}
		\UCitem{Postcondiciones}{Ninguna.}	
		\UCitem{Reglas de Negocio}{Ninguna.}
		\UCitem{Errores}{\MSGref{MSG65}{Impresiones insuficientes}.}
		\UCitem{Tipo}{Secundario.}
	\end{UseCase}		

	\begin{UCtrayectoria}{Principal}
		\UCpaso[\UCactor] Da clic en la opción reducir impresiones en la pantalla \IUref{IU}{Control impresiones}.
		\UCpaso Redirecciona a la pantalla \IUref{IU36}{Reducir impresiones}.
		\UCpaso[\UCactor] Selecciona el tipo y número de impresiones y oprime el botón \IUbutton {Aceptar}.
		\UCpaso Reduce el número indicado a la cuenta del usuario. \Trayref{A}
	\end{UCtrayectoria}

	\begin{UCtrayectoriaA}{A}{El usuario no cuenta con impresiones suficientes.}
		\UCpaso Muestra el mensaje \MSGref{MSG65}{Impresiones insuficientes}
		\UCpaso[\UCactor] Continua en paso 3 del caso de uso \UCref{35}
		
	\end{UCtrayectoriaA}


%-------------------------------------- TERMINA descripción del caso de uso.
