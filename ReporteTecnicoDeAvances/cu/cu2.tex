% \IUref{IUAdmPS}{Administrar Planta de Selección}
% \IUref{IUModPS}{Modificar Planta de Selección}
% \IUref{IUEliPS}{Eliminar Planta de Selección}

% 


% Copie este bloque por cada caso de uso:
%-------------------------------------- COMIENZA descripción del caso de uso.
\begin{UseCase}{CU2}{Modificar información personal}{
		Permite al Usuario modificar su información personal que registró al momento de darse de alta en el sistema. Podrá editar su nombre, primer apellido, segundo apellido, CURP, número de boleta o número de empleado según sea el caso, correo electrónico y teléfono celular. La modificación de sus datos la podrá realizar en el momento que lo requiera el Usuario.
	}
		\UCitem{Versión}{0.1}
		\UCitem{Autor}{Roberto Mendoza Saavedra.}
		\UCitem{Estatus}{Edición.}
		\UCitem{Fecha de último estatus}{9 de abril de 2018.}
		\UCitem{Actor}{Usuario: Alumno, Empleado, Externo.}
		\UCitem{Propósito}{Otorgar al Usuario un medio para que modifique sus datos personales.}
		\UCitem{Entradas}{Datos del usuario: 
		\begin{itemize}
		    \item Nombre(s)
		    \item Primer Apellido
		    \item Segundo Apellido
		    \item CURP
		    \item Número de boleta
		    \item Número de empleado
		    \item Correo electrónico
		    \item Teléfono celular
		\end{itemize}}
		\UCitem{Salidas}{\MSGref{MSG1}{Operación exitosa}.}
		\UCitem{Precondiciones}{
			\begin{itemize}
				\item Todos los campos obligatorios del formulario deben de ser completados.
				\item Debe existir un registro previo del usuario.
			\end{itemize}}
		\UCitem{Postcondiciones}{Se realiza una actualización a los datos del usuario.}
		\UCitem{Reglas de Negocio}{\BRref{RN01}{Unicidad del usuario}.}
		\UCitem{Errores}{\MSGref{MSG50}{Campo obligatorio},\MSGref{MSG51}{Campo no válido}, \MSGref{MSG52}{Longitud no válida}, \MSGref{MSG53}{Formato no válido}, \MSGref{MSG54}{Usuario duplicado}.}
		\UCitem{Tipo}{Secundario.}
	\end{UseCase}		

	\begin{UCtrayectoria}{Principal}
		\UCpaso[\UCactor] Selecciona la opción {\bf Modificar datos} del menú \IUref{IU5.1.2}{Menú de sesión}.
		\UCpaso Muestra la pantalla \IUref{IU02}{Modificar Información Personal}.
		\UCpaso Obtiene la información del usuario y la muestra en cada campo del formulario de la pantalla \IUref{IU2}{Modificar Información Personal}.
		\UCpaso[\UCactor] Edita los campos del formulario de la pantalla \IUref{IU2}{Modificar Información Personal} que desea actualizar y presiona el botón \IUbutton{Modificar}.\Trayref{A}
		\UCpaso Verifica que el usuario haya ingresado todos los campos obligatorios.\Trayref{B}
		\UCpaso Verifica que los campos ingresados sean válidos. \Trayref{C}
		\UCpaso Verifica que la longitud del CURP, número de boleta o número de empleado tengan la longitud correcta. \Trayref{D}
		\UCpaso Verifica que el formato del correo electrónico sea el correcto. \Trayref{E}
		\UCpaso Basado en la regla de negocio \BRref{RN01}{Unicidad del usuario} verifica que no exista un registro duplicado.
		\UCpaso Actualiza la información personal del Usuario.
		\UCpaso Direcciona al usuario a la pantalla \IUref{IU05}{Login} y muestra el mensaje \MSGref{MSG1}{Operación exitosa}.
	\end{UCtrayectoria}
		
		\begin{UCtrayectoriaA}{A}{El usuario solicita cancelar la operación.}
			\UCpaso[\UCactor] Presiona el botón \IUbutton{Cancelar} de la pantalla \IUref{IU2}{Modificar Información Personal}.
			\UCpaso Muestra la pantalla \IUref{IU5.1}{Bienvenida}
			\end{UCtrayectoriaA}
		
		\begin{UCtrayectoriaA}{B}{El usuario no ingresó todos los campos obligatorios.}
			\UCpaso  Muestra el Mensaje \MSGref{MSG50}{Campo obligatorio} en la pantalla \IUref{IU02}{Modificar Información Personal}.
			\UCpaso[\UCactor] Continua en el paso 4 del \UCref{CU2}.
		\end{UCtrayectoriaA}
	
		\begin{UCtrayectoriaA}{C}{El usuario ingresó campos no válidos.}
		\UCpaso  Muestra el Mensaje \MSGref{MSG51}{Campo no válido} en la pantalla \IUref{IU02}{Modificar Información Personal}.
		\UCpaso[\UCactor] Continua en el paso 4 del \UCref{CU2}.
	\end{UCtrayectoriaA}
	
		\begin{UCtrayectoriaA}{D}{El usuario ingresó el CURP, Número de boleta o Número de empleado	 con longitud incorrecta.}
		\UCpaso  Muestra el Mensaje \MSGref{MSG52}{Logitud no válida} en la pantalla \IUref{IU02}{Modificar Información Personal}.
		\UCpaso[\UCactor] Continua en el paso 4 del \UCref{CU2}.
	\end{UCtrayectoriaA}
	
		\begin{UCtrayectoriaA}{E}{El usuario ingresó un correo electrónico no válido.}
		\UCpaso  Muestra el Mensaje \MSGref{MSG53}{Formato no válido} en la pantalla \IUref{IU02}{Modificar Información Personal}.
		\UCpaso[\UCactor] Continua en el paso 4 del \UCref{CU2}.
	\end{UCtrayectoriaA}
	
%-------------------------------------- TERMINA descripción del caso de uso.
