\subsection{IU30 Visualizar historial clínico}

\subsubsection{Objetivo}{
	Mostrar el historial clínico del pasiente.
}
	
\subsubsection{Diseño}
	Se muestra la información básica del pasiente, una caja de texto que sirve como buscador, una tabla con los campos: \begin{itemize}
		\item Estatus.
		\item Fecha.
		\item Acciones.
	\end{itemize}
	dentro de acciones se muestran los siguientes iconos 
	\raisebox{-\mydepth}{\fbox{\includegraphics[height=\myheight]{icons/cuestionario}}},
	\raisebox{-\mydepth}{\fbox{\includegraphics[height=\myheight]{icons/verConsulta}}},
	\raisebox{-\mydepth}{\fbox{\includegraphics[height=\myheight]{icons/verPago}}}. También se muestra información hacerca del significado de colores en el estatus, además de un botón \IUbutton{Regresar}.

\IUfig[.6]{gui/Dentales/VisualizarHistorialClinico}{IU30}{Visualizar historial clínico}

%\subsubsection{Salidas}

%	Comentarios ingresados debajo de la información del componente electrónico.

%\subsubsection{Entradas}


\subsubsection{Comandos}
\begin{itemize}
	\item 
	\raisebox{-\mydepth}{\fbox{\includegraphics[height=\myheight]{icons/cuestionario}}}: Redirecciona a la pantalla \IUref{IU26}{Visualizar cuestionario}.
	\item 
	\raisebox{-\mydepth}{\fbox{\includegraphics[height=\myheight]{icons/verConsulta}}}: Redirecciona a la pantalla \IUref{IU31}{Visualizar consulta}.
	\item 
	\raisebox{-\mydepth}{\fbox{\includegraphics[height=\myheight]{icons/verPago}}}: Redirecciona a la pantalla \IUref{IU33}{Ver nota de pago}.
\end{itemize}

%\subsubsection{Mensajes}
%	\begin{Citemize}
%		\item {\bf MSG50} Campo obligatorio.
%		\item {\bf MSG55} Usuario y/o contraseña incorrectos.
%	\end{Citemize}