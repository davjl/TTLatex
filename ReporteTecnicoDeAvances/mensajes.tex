\begin{Message}{MSG1}{Operación exitosa}
	\MSGitem [Tipo: ] Informativo.
	\MSGitem [Proposito: ] Notificar al actor que la operación se realizo correctamente.
	\MSGitem [Redacción: ] La operación que realizo se llevo a cabo correctamente.
\end{Message}

\begin{Message}{MSG2}{Correo enviado}
	\MSGitem [Tipo: ] Informativo.
	\MSGitem [Proposito: ] Otorgar información al usuario de su nueva contraseña.
	\MSGitem [Redacción: ] Tu nueva contraseña a sido enviada a tu correo electrónico.
\end{Message}

\begin{Message}{MSG3}{Archivo enviado correctamente}
	\MSGitem [Tipo: ] Informativo.
	\MSGitem [Proposito: ] Informar al usuario del correcto almacenamiento del comprobante de pago.
	\MSGitem [Redacción: ] El voucher de pago a sido enviado correctamente, espera la notificación de caja.
\end{Message}

\begin{Message}{MSG4}{Archivos cargados correctamente}
	\MSGitem [Tipo: ] Informativo.
	\MSGitem [Proposito: ] Informar al cajero del correcto almacenamiento de los archivos de pago en el sistema.
	\MSGitem [Redacción: ] Los pagos realizados en el turno fueron correctamente almacenados.
\end{Message}

\begin{Message}{MSG5}{Archivo guardado correctamente}
	\MSGitem [Tipo: ] Informativo.
	\MSGitem [Proposito: ] Informar al cajero del correcto almacenamiento del comprobante SIG@ en el sistema.
	\MSGitem [Redacción: ] El comprobante SIG@ a sido almacenado correctamente.
\end{Message}

\begin{Message}{MSG6}{Comprobante aceptado}
	\MSGitem [Tipo: ] Informativo.
	\MSGitem [Proposito: ] Informar al cajero del correcto almacenamiento del voucher de pago.
	\MSGitem [Redacción: ] El voucher de pago a sido almacenado correctamente.
\end{Message}

\begin{Message}{MSG7}{Realiza el pago en caja}
	\MSGitem [Tipo: ] Informativo.
	\MSGitem [Proposito: ] Al usuario de la comprobación para el pago en caja.
	\MSGitem [Redacción: ] Tu nota de pago a sido enviada a caja, ahora puedes pasar a realizar tu pago.
\end{Message}

\begin{Message}{MSG50}{Campo obligatorio}
	\MSGitem [Tipo: ] Error.
	\MSGitem [Proposito: ] Informar al usuario de algún dato por ingresar.
	\MSGitem [Redacción: ] Los campos marcados con * son obligatorios.
\end{Message}

\begin{Message}{MSG51}{Campo no válido}
	\MSGitem [Tipo: ] Error.
	\MSGitem [Proposito: ] Informar al usuario del ingreso de un caracter no valido.
	\MSGitem [Redacción: ] ¡Error! El campo marcado contiene caracteres no validos.
\end{Message}

\begin{Message}{MSG52}{Longitud no válida}
	\MSGitem [Tipo: ] Error.
	\MSGitem [Proposito: ] Informar al usuario del ingreso de caracteres excesivo o faltante en un campo con longitud definida.
	\MSGitem [Redacción: ] ¡Error! El número de caracteres ingresados no es valido.
\end{Message}

\begin{Message}{MSG53}{Formato no válido}
	\MSGitem [Tipo: ] Error.
	\MSGitem [Proposito: ] Informar al usuario del ingreso de un dato mal redactado.
	\MSGitem [Redacción: ] ¡Error! El formato del dato marcado es incorrecto.
\end{Message}

\begin{Message}{MSG54}{Usuario existe}
	\MSGitem [Tipo: ] Error.
	\MSGitem [Proposito: ] Informar al usuario de que la cuenta que intenta crear ya existe.
	\MSGitem [Redacción: ] .
\end{Message}

\begin{Message}{MSG55}{Contraseñas diferentes}
	\MSGitem [Tipo: ] Error.
	\MSGitem [Proposito: ] Informar al usuario de un error de contraseña.
	\MSGitem [Redacción: ] ¡Error! La confirmación de contraseña es diferente.
\end{Message}

\begin{Message}{MSG56}{Usuario y/o contraseña incorrectos}
	\MSGitem [Tipo: ] Error.
	\MSGitem [Proposito: ] Informar a todos los actores de que el usario y/o contraseña no concuerdan con ninguno en el sistema.
	\MSGitem [Redacción: ] ¡Error! El usuario y/o contraseña son incorrectos.
\end{Message}

\begin{Message}{MSG57}{Formato de archivo no válido}
	\MSGitem [Tipo: ] Error.
	\MSGitem [Proposito: ] Informar al usuario de que el formato del archivo es incorrecto.
	\MSGitem [Redacción: ] ¡Error! El archivo que seleccionaste no cumple con el formato.
\end{Message}

\begin{Message}{MSG58}{Operación no completada}
	\MSGitem [Tipo: ] Error.
	\MSGitem [Proposito: ] Informar al actor de un problema en la operación que realizo.
	\MSGitem [Redacción: ] ¡Error! La operación no se pudo realizar correctamente, intenta de nuevo.
\end{Message}

\begin{Message}{MSG59}{No se pudo cargar el pago}
	\MSGitem [Tipo: ] Error.
	\MSGitem [Proposito: ] Informar al cajero de un error al intentar guardar los pagos aceptados.
	\MSGitem [Redacción: ] Error al intentar guardar los pagos aprobados. El sistema no se encuentra disponible.
\end{Message}

\begin{Message}{MSG60}{Error al guardar comprobante SIG@}
	\MSGitem [Tipo: ]  Error.
	\MSGitem [Proposito: ] Informar al cajero de un error al intentar anexar el comprobante SIG@ en el sistema.
	\MSGitem [Redacción: ] Error al intentar guardar el comprobante SIG@. El sistema no se encuentra disponible.
\end{Message}

\begin{Message}{MSG61}{No se pudo cambiar el estado del comprobante}
	\MSGitem [Tipo: ] Error.
	\MSGitem [Proposito: ] Informar al cajero de un error al intentar cambiar el estado del pago.
	\MSGitem [Redacción: ] Error al intentar cambiar el estado del pago. El sistema no se encuentra disponible.
\end{Message}

\begin{Message}{MSG62}{La hora de corte de caja a terminado}
	\MSGitem [Tipo: ] Error.
	\MSGitem [Proposito: ] Informar al cajero de que el tiempo de corte de caja a terminado.
	\MSGitem [Redacción: ] ¡Error! El tiempo del corte de caja a terminado.
\end{Message}

\begin{Message}{MSG63}{El usuario o correo no existen}
	\MSGitem [Tipo: ] Error.
	\MSGitem [Proposito: ] Informar al usuario de la inexistencia del número de usuario o correo electrónico.
	\MSGitem [Redacción: ] ¡Error! el número de usuario o correo electrónico no existen.
\end{Message}

\begin{Message}{MSG64}{No fue posible enviar nota}
	\MSGitem [Tipo: ] Error.
	\MSGitem [Proposito: ] Informar al usuario de un error al enviar la nota de pago.
	\MSGitem [Redacción: ] ¡Error! No fue posible mandar la nota de pago, intenta de nuevo.
\end{Message}

\begin{Message}{MSG70}{Verifique su información}
	\MSGitem[Tipo: ]Error.
	\MSGitem[Proposito: ] Informar a la persona de un error al ingresar los datos.
	\MSGitem[Redacción: ] Verifique su información
\end{Message}
