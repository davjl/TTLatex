\section{Encuestas}
Dentro de los métodos cualitativos aplicados a este trabajo se ha realizado una breve encuesta dirigida a alumnos y profesores de la ESCOM la cual consta de doce preguntas enfocadas en entender los gastos en tiempo, dinero y esfuerzo que realiza la comunidad para atender el proceso de pagos actual explicado en capítulos anteriores. Esta encuesta fue resuelta por un total de 102 usuarios. \\

La información que se ha recabado de forma anónima ha permitido hacer comparaciones del sistema propuesto con el proceso actual de gestión de pagos de la ESCOM, las preguntas se muestran a continuación:

\newpage
\IUfig[1]{encuesta/tiempoTransporte}{}{Tiempo de transporte}
\IUfig[1]{encuesta/gastoTransporte}{}{Gastos de transporte}
\newpage
\IUfig[1]{encuesta/gastoAlimentos}{}{Gastos de alimentación}
\IUfig[1]{encuesta/tiempoBancario}{}{Tiempo de pago en banco}
\newpage
\IUfig[1]{encuesta/tiempoProceso}{}{Consideración de servicio de pago}
\IUfig[1]{encuesta/mejoraProcesos}{}{Consideración en mejora de servicio}
\newpage
\IUfig[1]{encuesta/sistemaIncluyente}{}{Sistema incluyente con la comunidad}
\IUfig[1]{encuesta/mejoraProcesos}{}{Plataforma web/móvil en procesos de pago}
\newpage
\IUfig[1]{encuesta/consideraciones}{}{Consideraciones a mejorar}
\IUfig[1]{encuesta/perdidaRecibos}{}{Perdida de recibos}
\newpage
\IUfig[1]{encuesta/recuperarRecibo}{}{Recuperar recibo}
\IUfig[1]{encuesta/reponer}{}{Modo de reponer recibo}
\newpage

Con base a las respuestas obtenidas en cada una de las preguntas, podemos interpretar lo siguiente:

\begin{itemize}
	\item 54\% de los encuestados usan entre 1 hora y 2 horas para llegar a la escuela en transporte.
	\item 33\% de los encuestados gastan entre \$200 y \$500 en transporte pesos mensuales para ir a la escuela.
	\item 57\% de los encuestados gastan entre \$20 y \$60 diarios en alimentos cuando van a la escuela.
	\item 50\% de los encuestados indican que gastan entre 20 y 40 minutos en ir al banco y regresar a caja a entregar el recibo de pago, también, muestran inconformidad por perder clases durante el proceso.
	\item 57\% de los encuestados consideran el proceso en la caja es aceptablemente rápido sin embargo 39\% consideran que es muy malo y lento. 
	\item El 97\% de los encuestados opinan que la escuela puede mejorar el proceso actual.
	\item El 81\% de los encuestados desconocen la existencia de sistemas de gestión de pago como el SIG@ y consideran que los sistemas de la escuela no son incluyentes con la comunidad.
	\item El 88\% de la comunidad encuestada piensa que crear una plataforma WEB/móvil puede ayudar a mejorar el proceso actual.
	\item Del 88\% que piensa que una plataforma puede mejorar el proceso actual la mayoría opina que el principal beneficio es el ahorro de tiempo, la reducción de espera en las filas y la reducción de gastos.
	\item El 39\% de los encuestados dice perder sus comprobantes de pago (Copias, recibos de banco, etc.) y tener que volver a pagar el servicio.
	\item De los encuestados sólo una persona dice conocer un método de recuperación de sus recibos de pago.
\end{itemize}