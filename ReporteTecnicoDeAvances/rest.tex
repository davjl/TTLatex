\section{REST}
Se hace una descripción del desarrollo de una API que sirve para mantener la comunicación entre múltiples programas, la cual es esencial para poder garantizar la integridad de información que este interactuando entre la aplicación web y móvil, además de los beneficios en flexibilidad de conexión de una gran variedad de clientes para el caso móvil.
A continuación se presenta una descripción de los principales métodos  y funcionalidades integrados en la API:

\begin{itemize}
	\item {\bf Pagos.}
		\begin{table}[htbp]
			\begin{center}
				\begin{tabular}{|p{60mm}|p{70mm}|}
					\hline
					Nombre & getPaymentsByUserId. \\
					\hline
					Objetivo & Obtener los pagos emitidos por usuario basado en su identificador.\\
					\hline
					URL & /pagos/gestionar-pagos!getPaymentsByUserId?idUser={id}. \\
					\hline
					Método & GET \\
					\hline
					Parámetros & \begin{itemize}
						\item Obligatorio
						\item Id=[Integer]
						\item Ejemplo: id=14
					\end{itemize}\\
					\hline
					Respuesta Exitosa & \begin{itemize}
						\item Código: 200
						\item Contenido: \{ id : Integer, idUsuario : Integer, fecha : date, idCatalogo : Integer,  estado : Integer, idCuenta:Integer\}
					\end{itemize}\\ 
					\hline
					Respuesta de Error & \begin{itemize}
						\item Código: 401
						\item Contenido: {error: “no autorizado”}
						\item Descripción: Un usuario trata de acceder sin credenciales al servicio.
					\end{itemize}\\
					\hline
				\end{tabular}
				\caption{Método de obtención de pagos por id de usuario.}
				\label{tabla:getPaymentsByUserId}
			\end{center}
		\end{table}
		
		\begin{table}[htbp]
			\begin{center}
				\begin{tabular}{|p{60mm}|p{70mm}|}
					\hline
					Nombre & getAllServices. \\
					\hline
					Objetivo & Obtener el catalogo de servicios ofrecidos en la ESCOM. \\
					\hline
					URL &  /pagos/gestionar-servicios!getAllServices. \\
					\hline
					Método & GET\\
					\hline
					Parámetros & Sin parametros. \\
					\hline
					Respuesta Exitosa & \begin{itemize}
						\item Código: 200
						\item Contenido: { id : Integer, descripcion : “…”, costo : “…”, activo : boolean}
					\end{itemize}\\ 
					\hline
					Respuesta de Error & \begin{itemize}
						\item Código: 401
						\item Contenido: \{error: “no autorizado”\}
						\item Descripción: Un usuario trata de acceder sin credenciales al servicio.
					\end{itemize}\\
					\hline
				\end{tabular}
				\caption{Método de obtención de servicios}
				\label{tabla:getAllServices}
			\end{center}
		\end{table}
		
		\begin{table}[htbp]
			\begin{center}
				\begin{tabular}{|p{60mm}|p{70mm}|}
					\hline
					Nombre &  postPayment\\
					\hline
					Objetivo & Dar la capacidad de insertar pagos por medio de REST.\\
					\hline
					URL &  /pagos/cargar-pago/new \\
					\hline
					Método & POST\\
					\hline
					Respuesta Exitosa & \begin{itemize}
						\item Código: 200
						\item Contenido: \{ id : Integer, descripcion : “…”, costo : “…”, activo : boolean\}
					\end{itemize}\\ 
					\hline
					Respuesta de Error & \begin{itemize}
						\item Código: 401
						\item Contenido: \{error: “no autorizado”\}
						\item Descripción: Un usuario trata de acceder sin credenciales al servicio.
					\end{itemize}\\
					\hline
				\end{tabular}
				\caption{Método para agregar captura de pago}
				\label{tabla:postPayment}
			\end{center}
		\end{table}
		
		\item {\bf Notas.}
			\begin{table}[htbp]
				\begin{center}
					\begin{tabular}{|p{60mm}|p{70mm}|}
						\hline
						Nombre &  getNotesByUserId\\
						\hline
						Objetivo & Obtener las notas de pago por usuario. \\
						\hline
						URL &  /pagos/gestionar-notas-pago!getNotesByUserId?idUser={id} \\
						\hline
						Método & GET\\
						\hline
						Parámetros & \begin{itemize}
							\item Obligatorio
							\item Id=[Integer]
							\item Ejemplo: id=14
						\end{itemize} \\
						\hline
						Respuesta Exitosa & \begin{itemize}
							\item Código: 200
							\item Contenido: \{ id : Integer,  fecha : date , idServicio: Integer\}
						\end{itemize}\\ 
						\hline
						Respuesta de Error & \begin{itemize}
							\item Código: 401
							\item Contenido: \{error: “no autorizado”\}
							\item Descripción: Un usuario trata de acceder sin credenciales al servicio.
						\end{itemize}\\
						\hline
					\end{tabular}
					\caption{Método de obtención de notas de pago}
					\label{tabla:getNotesByUserId}
				\end{center}
			\end{table}
			
		\item {\bf Notificaciones}
			\begin{table}[htbp]
				\begin{center}
					\begin{tabular}{|p{60mm}|p{70mm}|}
						\hline
						Nombre & getNotificationsByUserId \\
						\hline
						Objetivo & Obtener notificaciones por usuario. \\
						\hline
						URL & /pagos/gestionar-notas-pago!getNotesByUserId?idUser={id} \\
						\hline
						Método & GET \\
						\hline
						Parámetros & \begin{itemize}
							\item Obligatorio
							\item Id=[Integer]
							\item Ejemplo: id=14
						\end{itemize} \\
						\hline
						Respuesta Exitosa & \begin{itemize}
							\item Código: 200
							\item Contenido: \{ id : Integer,  fecha : date , motivo : “…”, fecha : date\}
						\end{itemize}\\ 
						\hline
						Respuesta de Error & \begin{itemize}
							\item Código: 401
							\item Contenido: \{error: “no autorizado”\}
							\item Descripción: Un usuario trata de acceder sin credenciales al servicio.
						\end{itemize}\\
						\hline
					\end{tabular}
					\caption{Método notificaciones}
					\label{tabla:getNotificationsByUserId}
				\end{center}
			\end{table}
			
		\item{\bf Citas.}
			\begin{table}[htbp]
				\begin{center}
					\begin{tabular}{|p{60mm}|p{70mm}|}
						\hline
						Nombre & getAppointmentByUserId \\
						\hline
						Objetivo & Obtener citas agendadas por usuario. \\
						\hline
						URL &  /citas/gestionar-citas-dentales!getAppointmentByUserId?idUser={id} \\
						\hline
						Método & GET\\
						\hline
						Parámetros & \begin{itemize}
							\item Obligatorio
							\item Id=[Integer]
							\item Ejemplo: id=14
						\end{itemize} \\
						\hline
						Respuesta Exitosa & \begin{itemize}
							\item Código: 200
							\item Contenido: \{ id : Integer,  estado : boolean , descripcion : “…”, fecha : date\}
						\end{itemize}\\ 
						\hline
						Respuesta de Error & \begin{itemize}
							\item Código: 401
							\item Contenido: {error: “no autorizado”}
							\item Descripción: Un usuario trata de acceder sin credenciales al servicio.
						\end{itemize}\\
						\hline
					\end{tabular}
					\caption{Método de citas}
					\label{tabla:getAppointmentByUserId}
				\end{center}
			\end{table}
			
		\item{\bf Control acceso.}
			\begin{table}[htbp]
				\begin{center}
					\begin{tabular}{|p{60mm}|p{70mm}|}
						\hline
						Nombre & Login \\
						\hline
						Objetivo & Establecer un método de acceso al sistema por medio de REST. \\
						\hline
						URL &  /control-acceso/login?login=\{correo\}\&password=\{contraseña\} \\
						\hline
						Método & POST\\
						\hline
						Parámetros & \begin{itemize}
							\item Obligatorio
							\item Id=[String]
							\item Ejemplo: login=correo@gmail.com
							\item Obligatorio
							\item Id=[String]
							\item Ejemplo: password=contraseña
						\end{itemize} \\
						\hline
						Respuesta Exitosa & \begin{itemize}
							\item Código: 200
							\item Contenido: \{ id : Integer, perfil : Integer, activo : boolean\}
						\end{itemize}\\ 
						\hline
						Respuesta de Error & \begin{itemize}
							\item Código: 401
							\item Contenido: \{error: “no autorizado”\}
							\item Descripción: Un usuario trata de acceder sin credenciales al servicio.
						\end{itemize}\\
						\hline
					\end{tabular}
					\caption{Método de acceso}
					\label{tabla:Login}
				\end{center}
			\end{table}
\end{itemize}