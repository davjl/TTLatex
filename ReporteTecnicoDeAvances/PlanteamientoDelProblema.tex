\section{Planteamiento del problema}
A pesar de que en cada una de las áreas de servicios ya se cuenta con un equipo de cómputo y conexión a Internet, ninguna de ellas tiene un sistema que ayude a la gestión de sus pagos, obligando al personal a tener que realizar todos sus procesos de forma manual o ayudarse de las herramientas de ofimática que se les proporcionan.\\

Aunado a esto, se carece de un sistema que comunique directamente a las áreas con el servicio de caja para confirmar y aprobar un pago, sólo se  mantiene una comunicación entre ellas por medio de comprobantes impresos comprometiendo al usuario a asistir forzosamente a caja para poder realizar o recibir cualquier servicio.\\

Lo anterior, se convierte en un problema que puede ser visto desde rubros diferentes. Uno de ellos es el del usuario que necesita disponer de alguno de los servicios, pues en todos los casos, se ve obligado a presentarse directamente en la caja para comprobar el pago independientemente de la forma en cómo lo haya efectuado. Por tanto, si el usuario no le es posible acudir por alguna razón, el pago no podría llegar a ser aprobado desencadenando así más inconvenientes. Hablamos de que en algunos casos el usuario tendrá que realizar de nueva cuenta el pago, o bien, podría retrasarse más de lo esperado para poder recibir el servicio por el cual está pagando.\\

%identificar operacion
%Además de esto es muy importante marcar una posible debilidad en el sistema SIG@ ya que al verificar el pago de un comprobante y ser registrado en el sistema, es posible que después de unos días el usuario pueda mostrar el mismo comprobante y se registre de igual manera el pago. Si bien en caja se puede hacer 2 cortes de pago al día esto no garantiza que se pueda identificar al momento el pago duplicado y solo se pueda observar esta situación luego de 5 días al realizar (el estado de cuenta). Cabe recalcar que hasta el momento no se ha dado esta situación conforme lo platicado con el contador(a).

También recae en el usuario toda la responsabilidad de conservar el comprobante SIG@ emitido por caja hasta que le sea posible llevarlo al área donde lo hará válido. De la mano tenemos que una vez entregado ese comprobante, el usuario se queda sin ninguna garantía de que en algún momento efectúo su pago dejándolo sin ningún recurso para argumentar lo contrario. Esta situación en específico se presenta en las áreas de Fotocopiado, CELEX y Biblioteca.\\

Otro rubro que podemos considerar es el de las áreas que proporcionan los servicios, pues como mencionamos, carecen de un sistema que les apoye en el proceso de pagos de sus servicios. Esto los orilla a idear métodos que creen son los más convenientes, pero en realidad sólo rezagan sus procesos, y en ocasiones se salen de la normatividad impuesta por la ESCOM. Sumado a esto, tenemos que cada una de estas áreas debe de almacenar por cinco años los comprobantes SIG@ recibidos, lo que nos habla de carpetas llenas de comprobantes que al final de ese periodo simplemente se van a desechar siendo un gasto de recurso material innecesario tanto en lo económico como en lo ambiental.\\

Un último rubro a considerar será el del departamento de Recursos Financieros, que si bien ya cuentan con un sistema que recaba la información de los pagos que se reciben diariamente, no cuentan con un archivo de pagos que le permita reducir el gasto de recursos materiales, espacios físicos y sobretodo tiempo para el usuario. Decimos esto, porque cada pago recibido implica la impresión de dos comprobantes SIG@, el resguardo de la misma por un periodo de cinco años y además la presencia obligatoria del usuario para confirmar el pago de algún servicio.\\

Si bien el número de impresiones de estos comprobantes por alumno no es significativo, si lo es considerando que los alumnos, personal administrativo, docentes y externos al menos en una ocasión requieren de algún tipo de servicio.\\

Bajo esta perspectiva, hablamos de que en promedio se realizan durante un día 100 impresiones, sin considerar que en periodos de término de semestre o de exámenes a Título de Suficiencia se llegan a imprimir hasta 200 hojas. Considerando la cantidad promedio por día hablamos de que durante un semestre se imprimen aproximadamente 12,000 hojas, contemplando que el tiempo efectivo del semestre son 120 días (4 meses).\\

Si lo trasladamos a datos ambientales, nos daremos cuenta del impacto negativo al medio ambiente que esto tiene. Hablamos de que un árbol sirve para producir 8000 hojas de papel, lo que nos lleva a pensar que en un año se estaría acabando con tres árboles aproximadamente, sumando el uso de 8880 litros de agua para su fabricación teniendo en cuenta que por cada hoja de papel se ocupan 370 cc.\\

Por todo lo anterior, es importante el desarrollo de un sistema que nos permita controlar todos estos problemas, pensando en una solución que contemple el aspecto administrativo y ambiental.\\