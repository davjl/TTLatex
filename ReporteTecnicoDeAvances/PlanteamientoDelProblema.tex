\section{Planteamiento del problema}

En este capítulo tras haber analizado el proceso de pagos actual, se ha determinado listar algunos de los problemas más relevantes encontrados hasta el momento, los cuales están relacionados con relacionados con la eficacia y eficacia del proceso.

\subsection{Eficacia}

{\bf Duplicidad de pago:}\\

Si la cantidad de pagos emitidos en un día incrementa de manera considerable, los encargados de contabilizar la cantidad de pagos y la cantidad total recibida pueden tardar más de un día en hacer llegar los comprobantes a el área de caja, sin embargo, en el proceso actual la contadora y el subdirector están obligados a reportar  la cantidad total recibida durante el corte de caja del turno matutino y vespertino, respectivamente.

Considerando este factor, para cuando los encargados de contabilizar los pagos informan sus resultados a la contadora y el subdirector pueden hallar pérdidas por duplicidad de pago, es decir, un emisor pudo haber hecho válido el servicio solicitado con un comprobante de pago replicado. Esto es posible debido a que actualmente el área sólo comprueba la cantidad emitida por concepto de algún servicio del catálogo vía telefónica si esta es ingente.

{\bf Evasión del pago dental:}\\

Como se ha explicado anteriormente cuando un usuario solicita un servicio dental tiene que agendar una cita para que el médico en turno haga una inspección, en este punto si el médico tiene el material necesario realiza el servicio y una vez terminado le escribe una nota que debe llevar a caja para pagar el material utilizado más el costo del servicio realizado. Es en este punto es donde el usuario puede evadir el pago del servicio puesto que en el  proceso actual debido a las características de naturaleza médica se indica revisar y atender al paciente sin previo pago, por lo que al concluir el servicio el usuario puede optar por retirarse de las instalaciones sin pagar.

\subsection{Eficiencia}

{\bf Lentitud de comunicación entre caja y servicios:}\\

Actualmente la comunicación entre áreas se da por vía telefónica, cuando la demanda de servicios incrementa, especialmente en temporada de inscripciones y exámenes, la cantidad de pagos emitidos en un día puede volver lento el proceso de gestión de pagos derivando en problemas graves como pérdida de dinero, por otro lado, los administrativos de las áreas contempladas en el presente trabajo indican que al finalizar el proceso de contabilización de pagos deben imprimir dos comprobantes de pago y llevar una copia de cada comprobante al área de caja,.

Si bien no todas las áreas tienen la misma demanda, algunas áreas contempladas como la del servicio de fotocopiado pueden emitir un total de 80 pagos al día, esto sin tomar en cuenta que todas las áreas que hacen uso de los servicios de caja deben realizar el mismo proceso saturando el área de caja haciendo aún más complicada la comunicación entre áreas.

{\bf Tiempo y costo de traslado: }\\

Se ha observado que cuando incrementa la demanda de servicios en la unidad académica los alumnos y otros interesados en adquirir servicios pueden saturar los servicios de caja y bancarios, como parte del estudio realizado con el objetivo de conocer a los usuarios del sistema se ha encontrado que los alumnos tardan en promedio 45 minutos en ir desde la unidad académica hasta el banco más cercano, un caso particular son las incripciones del CELEX o los pagos de exámenes a título de suficiencia, a pesar de que se establece un periodo para realizar el pago de estos servicios.

Se ha encontrado que la mayoría de los alumnos tiende a realizar su pago generalmente el último día de dicho periodo, lo cual además forza los alumnos a trasladarse a la escuela para entregar su voucher de pago con los datos requeridos por la escuela al área de caja misma que genera dos comprobantes SIG@ uno para almacenar y el segundo para que el interesado haga válido su servicio en el área correspondiente.

{\bf Pérdida de comprobantes: }\\

Los usuarios de los servicios de la unidad académica no cuentan con un mecanismo de recuperación de comprobantes de pago, un ejemplo de ello es el área de impresiones, como se ha visto en el capítulo anterior el proceso empieza cuando una vez realizado el pago y entregado el comprobante al encargado del área de impresiones este le emite al usuario un comprobante con el número de impresiones disponible mismo que va restando manualmente cuando se hace uso del servicio, es en este caso que si el usuario pierde el comprobante emitido por el encargado de caja tendría que volver a pagar y realizar todo el proceso puesto que no cuenta con un mecanismo de recuperación de su comprobante.



A pesar de que en cada una de las áreas de servicios ya se cuenta con un equipo de cómputo y conexión a Internet, ninguna de ellas tiene un sistema que ayude a la gestión de sus pagos, obligando al personal a tener que realizar todos sus procesos de forma manual o ayudarse de las herramientas de ofimática que se les proporcionan.\\

Aunado a esto, se carece de un sistema que comunique directamente a las áreas con el servicio de caja para confirmar y aprobar un pago, sólo se  mantiene una comunicación entre ellas por medio de comprobantes impresos comprometiendo al usuario a asistir forzosamente a caja para poder realizar o recibir cualquier servicio.\\

Lo anterior, se convierte en un problema que puede ser visto desde rubros diferentes. Uno de ellos es el del usuario que necesita disponer de alguno de los servicios, pues en todos los casos, se ve obligado a presentarse directamente en la caja para comprobar el pago independientemente de la forma en cómo lo haya efectuado. Por tanto, si el usuario no le es posible acudir por alguna razón, el pago no podría llegar a ser aprobado desencadenando así más inconvenientes. Hablamos de que en algunos casos el usuario tendrá que realizar de nueva cuenta el pago, o bien, podría retrasarse más de lo esperado para poder recibir el servicio por el cual está pagando.\\

%identificar operacion
%Además de esto es muy importante marcar una posible debilidad en el sistema SIG@ ya que al verificar el pago de un comprobante y ser registrado en el sistema, es posible que después de unos días el usuario pueda mostrar el mismo comprobante y se registre de igual manera el pago. Si bien en caja se puede hacer 2 cortes de pago al día esto no garantiza que se pueda identificar al momento el pago duplicado y solo se pueda observar esta situación luego de 5 días al realizar (el estado de cuenta). Cabe recalcar que hasta el momento no se ha dado esta situación conforme lo platicado con el contador(a).

También recae en el usuario toda la responsabilidad de conservar el comprobante SIG@ emitido por caja hasta que le sea posible llevarlo al área donde lo hará válido. De la mano tenemos que una vez entregado ese comprobante, el usuario se queda sin ninguna garantía de que en algún momento efectúo su pago dejándolo sin ningún recurso para argumentar lo contrario. Esta situación en específico se presenta en las áreas de Fotocopiado, CELEX y Biblioteca.\\

Otro rubro que podemos considerar es el de las áreas que proporcionan los servicios, pues como mencionamos, carecen de un sistema que les apoye en el proceso de pagos de sus servicios. Esto los orilla a idear métodos que creen son los más convenientes, pero en realidad sólo rezagan sus procesos, y en ocasiones se salen de la normatividad impuesta por la ESCOM. Sumado a esto, tenemos que cada una de estas áreas debe de almacenar por cinco años los comprobantes SIG@ recibidos, lo que nos habla de carpetas llenas de comprobantes que al final de ese periodo simplemente se van a desechar siendo un gasto de recurso material innecesario tanto en lo económico como en lo ambiental.\\

Un último rubro a considerar será el del departamento de Recursos Financieros, que si bien ya cuentan con un sistema que recaba la información de los pagos que se reciben diariamente, no cuentan con un archivo de pagos que le permita reducir el gasto de recursos materiales, espacios físicos y sobretodo tiempo para el usuario. Decimos esto, porque cada pago recibido implica la impresión de dos comprobantes SIG@, el resguardo de la misma por un periodo de cinco años y además la presencia obligatoria del usuario para confirmar el pago de algún servicio.\\

Si bien el número de impresiones de estos comprobantes por alumno no es significativo, si lo es considerando que los alumnos, personal administrativo, docentes y externos al menos en una ocasión requieren de algún tipo de servicio.\\

Bajo esta perspectiva, hablamos de que en promedio se realizan durante un día 100 impresiones, sin considerar que en periodos de término de semestre o de exámenes a Título de Suficiencia se llegan a imprimir hasta 200 hojas. Considerando la cantidad promedio por día hablamos de que durante un semestre se imprimen aproximadamente 12,000 hojas, contemplando que el tiempo efectivo del semestre son 120 días (4 meses).\\

Si lo trasladamos a datos ambientales, nos daremos cuenta del impacto negativo al medio ambiente que esto tiene. Hablamos de que un árbol sirve para producir 8000 hojas de papel, lo que nos lleva a pensar que en un año se estaría acabando con tres árboles aproximadamente, sumando el uso de 8880 litros de agua para su fabricación teniendo en cuenta que por cada hoja de papel se ocupan 370 cc.\\

Por todo lo anterior, es importante el desarrollo de un sistema que nos permita controlar todos estos problemas, pensando en una solución que contemple el aspecto administrativo y ambiental.\\