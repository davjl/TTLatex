En el presenta documento se icorporan parte de los casos de uso que han derivado del análisis realizado a las areas involucradas, los participantes han tenido que ser mediadores entre los distintos departamentos para llegar a un acuerdo en el que el sistema les sea útil y práctico. En la primera entrega se presentará el documento técnico, la maquetación de las pantallas y la primera parte del modelo de datos que servirá para hacer una gestión simple del problema y por lo tanto se escribirá el código correspondiente sólo a esta gestión.\\ 

Se ha llegado a la conclusíon de que la aplicación no puede ser en su totalidad móvil debido a que las áreas involucradas informan que no sería práctico para sus correspondientes gestiones. Por tal motivo se ha decidido realizar una aplicación hibrida que cumplirá de cualquier forma con las funciones de una aplicación móvil al ser responsiva,  agregando la funcionalidad de notificaciones a dispositivos móviles para los actores del sistema quienes declararon les sería de utilidad.

\section{Base de datos}
El siguiente diagrama entidad-relación será mapeado en un modelo ORM debido a que las tecnologías empleadas en este trabajo incluyen herramientas que transforman entidades en objetos, lo cual reduce el tiempo de desarrollo.\\
Esta primera entrega consta de 13 entidades que modelan el flujo de una gestión de pago.

\IUfig[1]{gui/SPEE}{}{Modelo de base de datos}

\section{Casos de uso}
Se presentan los diagramas de casos de uso del sistema en sus respectivos paquetes:

\cfinput{CUS}

%\subsection{Redacción de casos de uso}
%El análisis de la primer parte del trabajo terminal nos arrojó 45 casos de uso de los cuales a continuación se presentan:\\
