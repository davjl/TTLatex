En el presente capítulo se habla del trabajo realizado para esta primera etapa del presente Trabajo Terminal. Se incluyen los casos de uso principales para el desarrollo de este proyecto derivados del análisis realizado durante las entrevistas con las áreas involucradas.\\

Hemos tenido que actuar como mediadores entre los distintos departamentos para llegar a una solución en común. En la primera entrega se considera este reporte técnico de avances, el maquetado de las pantallas y la primera parte del modelo de datos que servirá para hacer una gestión simple del problema y por lo tanto se escribirá el código correspondiente sólo a esta gestión.\\

\section{Análisis realizado}
El primer paso de todo el análisis se inicio con una serie de entrevistas elaboradas a las áreas de servicios de Caja, Contaduría, CELEX y Dentales, en las cuales nos mencionaron el proceso en que operan y como es que realizan los trámites de los servicios que imparte cada una. Durante estas primeras entrevistas se buscó identificar las mayores debilidades y necesidades que presentaban en su actual modelo de procesos.\\

Luego de tener identificados los puntos más cruciales del sistema, se desarrollaron una serie de interfaces para los usuarios (Mockups) logrando una propuesta de desarrollo inicial permitiendo tener una mayor comunicación, ya que al tener un modelo de vista, nos pudieron explicar a  detalle qué funcionalidades son las que necesitaban y cuáles no tenían ninguna importancia para ellos. Con los servicios de caja y contaduría se tuvo el mayor número de visitas para poder tener bien definidos que funcionalidades se encontraban dentro de nuestras posibilidades sin tener que entrar en conflicto con el sistema SIG@. Para el área de CELEX se presentó una comunicación de lo más sencilla, pudiendo captar de forma inmediata los requerimientos para esta área. No fue así en el caso de área dental, en la cual fue muy difícil llegar a un entendimiento mutuo con ambas dentistas, ya que al inicio de las entrevistas nos comentaron muchas deficiencias de procesos de pago, además de hacer mención de funcionalidades que ellas podrían usar personalmente en el sistema, y que después de presentar el modelado de vistas, ya no tomaban como funcionales. Posteriormente de estas entrevistas y no llegar a un acuerdo mutuo, se estableció junto con nuestros directores un modelo que cumpliera con las necesidades iniciales que presentaron las dentistas, con el fin de evitar una mayor pérdida de tiempo en el análisis y la implementación.\\

El siguiente proceso fue la definición y modelado de nuestros casos de uso, los cuales en una primera instancia resultaron en un total de 75, mismos que disminuyeron posteriormente al realizar una reingeniería. Así, se logró reducirlos hasta la cantidad de 45 casos de uso.\\

\section{Base de datos}
El siguiente diagrama entidad-relación será mapeado en un modelo ORM debido a que las tecnologías empleadas en este trabajo incluyen herramientas que transforman entidades en objetos, lo cual reduce el tiempo de desarrollo.\\
Esta primera entrega consta de 13 entidades que modelan el flujo de una gestión de pago.

\IUfig[1]{gui/SPEE}{}{Modelo de base de datos}

\newpage
\section{Casos de uso}
Se presentan los diagramas de los casos de uso mas relevantes del sistema en sus respectivos paquetes. La descripción de los  casos de uso principales para este sistema también se presentan, el compendio completo de los casos de uso mencionados con anterioridad se puede consultar en la siguiente dirección https://github.com/davjl/TTLatex\\

\cfinput{cu/CU1}
\cfinput{cu/CU8}
\cfinput{cu/CU14}
\cfinput{cu/CU34}
\cfinput{cu/CU35}
\cfinput{cu/CU36}
\newpage
\cfinput{CUS}


%\subsection{Redacción de casos de uso}
%El análisis de la primer parte del trabajo terminal nos arrojó 45 casos de uso de los cuales a continuación se presentan:\\
