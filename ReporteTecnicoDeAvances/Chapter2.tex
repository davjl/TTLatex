% Chapter 2

\chapter{Antecedentes}
\label{Capitulo 2} 

En este capítulo se da a conocer un contexto de la problemática que tratamos de resolver mediante el desarrollo del presente Trabajo Terminal. Por tal motivo, es que hablamos sobre el actual proceso de pagos en la ESCOM y todo lo que en eso se involucra.\\

Se habla de sistemas y aplicaciones que tienen una funcionalidad parecida a nuestro proyecto, detallando sus características, ventajas y desventajas con base en el contexto sobre el cual estamos trabajando para que así podamos tener algún modelo de referencia para nuestro desarrollo.\\

\section{Situación Actual}

Para tener una idea clara del problema que se ataca, es necesario dar un contexto sobre todo lo que se ve involucrado en el proceso que se lleva a cabo actualmente en la ESCOM para realizar un pago, con el fin de obtener un servicio o producto.\\

Estos servicios o productos de los que hablamos desde el principio del documento forman parte de una estrategia por parte de la ESCOM para poder adquirir recursos económicos independientes del presupuesto federal asignado, con la finalidad de dar manutención a las instalaciones, tales como, laboratorios de cómputo, aulas de clase, sanitarios, zonas verdes, entre otros.\\

Lo anterior, no sólo lo realiza la ESCOM, también las demás escuelas que conforman a todo el IPN, pero cada una de estas definen qué productos o servicios pondrán a disposición de sus estudiantes, empleados o externos. Todos esos productos o servicios se encuentran en un catálogo de cuotas de productos y cuotas de aprovechamiento de Cobro Regular emitido por la Secretaría de Hacienda y Crédito Público (SHCP) a través del sistema DEPAMIN: "Módulo de Estimación de Ingresos por Concepto de Derechos, Productos y Aprovechamientos" [1].\\

Dicho catálogo se renueva año con año presentando una relación aproximadamente de 1073 cuotas de productos y 19 cuotas de aprovechamientos de Cobro Regular [2]. Cada una de esas cuotas es incorporada en el SIG@, poniendo a disposición una boleta de pago, relacionada con cada cuota.\\

ESCOM, en lo particular toma sólo algunos de esos conceptos descritos en el catálogo y los ofrece a toda su comunidad, incluyendo también a externos. Suele tener una clasificación para cada servicio o producto que oferta, estas clasificaciones son las siguientes:\\

\begin{itemize}
	\item Cursos de educación continua
	\item Posgrado
	\item CELEX
	\item Odontológicos
	\item Productos 
\end{itemize}

En total son 36 conceptos de pago los que ESCOM considera para la oferta de servicios o productos repartidos en cada una de las clasificaciones anteriores. Los fondos recabados mediante el pago en caja, pago en sucursal bancaria o transferencia electrónica son depositados en una cuenta de débito de BBVA Bancomer, misma que el IPN monitorea para la declaración de ingresos, por lo tanto, es necesario que el departamento de Recursos Financieros de la ESCOM tenga siempre disponibles todas las boletas de pago emitidas por el SIG@ durante cada periodo escolar.\\

Tan es así, que se exige conservarlas por un periodo de cinco años a partir del año en que se expide. Esta regla no sólo aplica para el departamento de Recursos Financieros, también  para cada una de las áreas de donde derivan la prestación de los productos o servicios. Para ser más específicos, mencionaremos las áreas involucradas en cada una de las clasificaciones que se comentaron anteriormente.\\

\begin{itemize}
	\item {\bf Cursos de educación continua: }Coordinación de Educación Continua (CEC) [3].
	\item {\bf Posgrado: }Sección de Estudios de Posgrado e Investigación (SEPI) [4].
	\item {\bf CELEX: }Coordinación de Cursos Extracurriculares de Lenguas Extranjeras [5].
	\item {\bf Odontológicos: }Servicio Dental.
	\item {\bf Productos: }Fotocopiado y Biblioteca [6].
\end{itemize}

Si bien comentamos que esas boletas de pago se deben de almacenar por un periodo de cinco años, hasta el momento, ninguna de las áreas tiene claro qué es lo que sucede con toda esa documentación una vez finalizado dicho periodo. Actualmente, sólo las almacenan en carpetas y se conservan en los estantes.\\

Todos los ingresos obtenidos durante el día deberán de comprobarse, puesto que se genera un corte de caja de manera diaria que permite saber el total generado, que a su vez, es notificado al Subdirector Administrativo quien será el último en dar el visto bueno para los ingresos obtenidos del día corriente.

\subsection{Pagos en ESCOM}
El proceso de pago en ESCOM se lleva prácticamente de la misma manera en todos los servicios o productos que se ofertan, partiendo de la realización del pago en caja, sucursal bancaria o transferencia electrónica y concluyendo en la entrega al interesado por el producto o servicio de una boleta de pago emitida por el SIG@ la cual se da posteriormente al área involucrada como garantía de que se efectúo correctamente la cuota. Pueden existir algunas variantes en este proceso, pero no afectan para nada el flujo principal comentado anteriormente.\\

Enfocados en el alcance del proyecto daremos el detalle del proceso de pago para cada una de las áreas que estamos considerando, además de mostrar su catálogo de servicios disponibles y sus costos para este año en curso (2018). Esto se ha escrito derivado de múltiples entrevistas con las distintas áreas. Cabe mencionar que para efectos de este proyecto se entiende por usuario a todo aquel que se interese por algún producto o servicio.\\

\subsubsection{Servicio Dental}
Comencemos con el área de servicios dentales, sus costos se muestran en la tabla \ref{tabla:Dentales}.\\

\begin{table}[htbp]
	\begin{center}
		\begin{tabular}{|p{110mm}|p{25mm}|}
			\hline
			{\bf Producto/Servicio} &  {\bf Monto} \\
			\hline
			Servicio Médico de Aplicación de Amalgama. Estudiantes del IPN & \$ 74.00\\ \hline
			Servicio Médico de Aplicación de Amalgama. Público en General & \$ 142.00\\ \hline
			Servicio Médico de Aplicación de Amalgama. Trabajadores del IPN & \$ 101.00\\ \hline
			Servicio Médico de Cementación Incrustación. Estudiantes del IPN y Comunidad del IPN & \$ 68.00\\ \hline
			Servicio Médico de Consulta de Odontología. Comunidad del IPN & \$ 27.00\\ \hline
			Servicio Médico de Curación. Estudiantes del IPN & \$ 27.00\\ \hline
			Servicio Médico de Curación. Comunidad del IPN & \$ 74.00\\ \hline
			Servicio Médico de Curación. Trabajadores & \$ 47.00\\ \hline
			Servicio Médico de Curación. Trabajadores & \$ 59.00\\ \hline
			Servicio Médico de Extracción de piezas dentales. Docentes y empleados & \$ 66.00\\ \hline
			Servicio Médico de Extracción de piezas dentales. Docentes y empleados & \$ 180.00\\ \hline
			Servicio Médico de Impresión Parcial (hilo retractor, yeso velmix, hule, rectificador, alginato, yeso piedra y cera) & \$ 81.00\\ \hline
			Servicio Médico de Profilaxis. Estudiantes del IPN & \$ 108.00\\ \hline
			Servicio Médico de Profilaxis. Comunidad Politécnica & \$163.00\\ \hline
			Servicio Médico de Profilaxis. Público en General & \$ 196.00\\ \hline
			Servicio Médico de Resina autopolimerizable. Estudiantes del IPN & \$ 155.00\\ \hline
			Servicio Médico de Resina autopolimerizable. Docentes y empleados & \$ 175.00\\ \hline
			Servicio Médico de Resina chica & \$ 41.00\\ \hline
			Servicio Médico de Resina. Comunidad Politécnica & \$ 172.00\\ \hline		
		\end{tabular}
		\caption{Catálogo de servicios dentales.}
		\label{tabla:Dentales}
	\end{center}
\end{table}

En este caso, cuando un usuario requiere de algún servicio odontológico se presenta en el consultorio sin cita previa para la atención con algún dentista en turno, después, éste hace un análisis del problema del paciente, sin previo pago realiza el servicio diagnosticado o solicitado para finalmente realizar una nota de pago que le entrega al usuario la cual debe de presentar a caja para efectuar su pago en efectivo. Una vez realizado, el cajero genera e imprime en dos ocasiones el comprobante del SIG@ para después entregarle uno de ellos al usuario y otro para su resguardo. Este comprobante es el que el usuario entrega al dentista para confirmar su pago, mismo que almacena por políticas de la institución. Si el paciente necesita más servicios odontológicos el proceso se repite.\\

\subsubsection{CELEX}
Ahora bien, hablemos del área de CELEX, mencionando primero sus costos. Estos se muestran en la tabla \ref{tabla:Celex}.\\
\newpage
\begin{table}[htbp]
	\begin{center}
		\begin{tabular}{|p{110mm}|p{25mm}|}
			\hline
			{\bf Producto/Servicio} &  {\bf Monto} \\
			\hline
			Curso de Idiomas semanal o sabatino, 40 horas. Comunidad IPN & \$ 544.00\\ \hline
			Curso de Idiomas semanal o sabatino, 40 horas. Público en General & \$ 1,053.00\\ \hline			
		\end{tabular}
	\caption{Catálogo de servicios CELEX.}
\label{tabla:Celex}
\end{center}
\end{table}

Para esta área, el proceso de pago es muy similar, el usuario tiene que efectuar el pago directamente en cualquier sucursal bancaria BBVA Bancomer a la cuenta que previamente se le otorgó acudiendo a las oficinas del CELEX. Es importante decir, que también se puede efectuar el pago a través de una transferencia electrónica hacia el mismo banco. Aquí, no son permitidos los pagos directamente en la caja de la ESCOM.\\

Una vez efectuado el pago el usuario debe de acudir a la caja de la ESCOM para presentar el voucher emitido por el banco, o bien, la captura de pantalla o impresión que compruebe la transferencia electrónica. Hecho eso, el cajero emite la boleta de pago del SIG@ y la imprime en dos ocasiones, una la entrega al usuario como garantía de su pago y otra la deja para su resguardo. El usuario acude a las oficinas del CELEX con esa boleta y la entrega a la coordinadora del área para que ella lo registre en sus listas de inscripciones y posteriormente le asigne un grupo y un horario. 

Cabe destacar que el CELEX actualmente no cuenta con ningún sistema independiente de la ofimática que le apoye para la administración de sus inscripciones y grupos. Su gestión se basa en el uso de hojas de cálculo en Excel.

\subsubsection{Área de Fotocopiado}
Esta área se basa en un solo concepto de pago del que se derivan algunas consideraciones para la oferta de todos sus servicios. El concepto de pago y su costo se muestran en la tabla \ref{tabla:Fotocopiado}.\\

\begin{table}[htbp]
	\begin{center}
		\begin{tabular}{|p{110mm}|p{25mm}|}
			\hline
			{\bf Producto/Servicio} &  {\bf Monto} \\
			\hline
			Impresiones láser t/carta (equivalente a 15 impresiones) & \$ 9.28\\ \hline			
		\end{tabular}
		\caption{Catálogo de servicios de fotocopiado.}
		\label{tabla:Fotocopiado}
	\end{center}
\end{table}

Los servicios proporcionados por esta área se concentran en las impresiones a blanco y negro, a color, copias y ploteos. El costo para cada uno de ellos como bien lo apreciamos en la tabla parte de un solo concepto de pago equivalente a 15 impresiones.  Estas equivalencias entre el n\'umero de impresiones disponibles y las necesarias para poder utilizar alguno de estos servicios son definidas semestre tras semestre por el departamento de Recursos Financieros. Hasta este momento durante el semestre vigente 2018-2 las tabulaciones están estipuladas de la siguiente manera:
\begin{itemize}
	\item 1 impresi\'on = \$0.62 = 1 impresi\'on o copia en blanco y negro.
	\item 5 impresiones = \$3.083 = 1 impresi\'on o copia a color.
	\item 15 impresiones = \$9.25 = 1 impresi\'on o copia doble carta o 1 impresión 1/4  plotter.
	\item 30 impresiones = \$18.50 = 1 impresi\'on en 1/4 plotter.
	\item 60 impresiones = \$37.00 = 1 impresi\'on en plotter completo
\item O m\'as en m\'ultiplos de 15 impresiones.
\end{itemize} 

El proceso de pago comienza cuando el usuario acude a la caja de la ESCOM para realizar su pago en efectivo equivalente al número de impresiones deseadas. El cajero emite la boleta de pago del SIG@ y realiza la impresión de la misma en dos ocasiones, una de ellas se la otorga al usuario para que haga válido su pago en el área y otra la deja a su resguardo. 
El usuario con esa boleta se presenta en el área de fotocopiado y solicita cualquiera de los servicios que ya mencionamos.\\

Una vez efectuado el servicio, el encargado del área solicita la boleta de pago al usuario y la intercambia por un pequeño papel que entre otros datos contiene el número de impresiones por el que es válido, cada vez que un usuario hace una impresión el encargado representa con un símbolo en este papel el número de impresiones usadas hasta que estas se terminan y si el usuario necesita más tiene que empezar todo el proceso nuevamente.

Este papel del que hablamos, será válido únicamente por el semestre en curso y es responsabilidad totalmente del usuario la conservación del mismo. Si éste se extravía se tendrá que realizar de nuevo el pago a pesar de que se pudieron haber tenido todavía impresiones disponibles.

\subsubsection{Área de Biblioteca}
Los conceptos de pago que en esta área se consideran son únicamente dos y se muestran en la tabla \ref{tabla:Biblioteca}.\\

\begin{table}[htbp]
	\begin{center}
		\begin{tabular}{|p{110mm}|p{25mm}|}
			\hline
			{\bf Producto/Servicio} &  {\bf Monto} \\
			\hline
			Multa de biblioteca & \$ 6.50\\ \hline	
			Reposición de credencial de biblioteca & \$ 26.00\\ \hline		
		\end{tabular}
		\caption{Catálogo de servicios de fotocopiado.}
		\label{tabla:Biblioteca}
	\end{center}
\end{table}

El proceso de pago comienza con la asistencia del usuario a la Biblioteca de la ESCOM, ahí, el encargado del área le dará una nota de pago sin ningún formato con el concepto y la cantidad a pagar. El usuario acude a la caja de la ESCOM con dicha nota y efectúa el pago.\\

El cajero por su parte, emite el comprobante del SIG@ y realiza la impresión del mismo en dos ocasiones, uno se lo entrega al usuario y el otro lo resguarda. Hecho lo anterior, el usuario regresa a la biblioteca y presenta el comprobante que le fue dado en caja para que sea liberada la multa o bien, la credencial de la biblioteca.\\

Si el usuario en otra ocasión presenta una multa en su historial o pierde de nuevo la credencial tendrá que realizar el procedimiento nuevamente.\\

Es importante recordar que todos los comprobantes de pago emitidos por el SIG@ deberán de ser guardados por un periodo de cinco años tanto en el departamento de Recursos Financieros como en cada una de las áreas. Esto por reglas de negocio del mismo IPN.

\section{Estado del arte}

Durante nuestra investigaci\'on de mercado, nos encontramos con sistemas que tienen una fuerte relaci\'on con el proyecto a trabajar, pero implementados en situaciones distintas, adem\'as de tener funcionalidades variadas. Si bien, actualmente existen muchas aplicaciones o sistemas enfocados en la gestión de pagos, la gran mayoría funcionan como desarrollos independientes de un contexto dejando de lado la integración a futuro de más módulos o incluso sistemas. Es por ello, que nosotros buscamos desarrollar un sistema que permita la escalabilidad a favor de la gestión de procesos tomando como punto de partida justamente la gestión de pagos.\\

Los sistemas con mayor similitud a nuestro desarrollo los podemos encontrar en la tabla \ref{tabla:Productos}
\begin{table}[htbp]
	\begin{center}
		\begin{tabular}{|p{30mm}|p{75mm}|p{25mm}|}
			\hline
			{\bf Software} & {\bf Caracter\'isticas} & {\bf Costo} \\
			\hline 
			"SISTEMA DE MONEDERO VIRTUAL PARA PAGOS ESCOLARES" & El Sistema de Monedero Virtual para Pagos Escolares es un sistema de prepago, que permite a los alumnos realizar pagos dentro y fuera de la Unidad Profesional Interdisciplinaria en Ingenier\'ia y Tecnolog\'ias Avanzadas (UPIITA) [7] & Sin costo \\ \hline
			
			TTR-12-1-029 Prototipo para el manejo de “Cero Papel”. & Es un sistema que permite el manejo, intercambio y control de la informaci\'on dentro de una organizaci\'on para optimizar los procedimientos y tareas, disminuyendo el uso de papel mediante la implementaci\'on de un sistema que permita administrar los usuarios y los documentos [8] & Sin costo\\ \hline
			
			“Ventanilla Virtual UdeG” & Ventanilla Virtual tiene tecnolog\'ias desarrolladas por universitarios y busca brindar una plataforma para que, en una primera etapa, los estudiantes puedan hacer tr\'amites, dar seguimiento, recuperarlos y hacer pagos respectivos [9] & Sin costo \\ \hline
			
			“Campus Pay” & Desarrollo que ofrece a estudiantes la posibilidad de realizar todos los pagos relacionados con sus estudios desde dispositivos m\'oviles con cualquier tarjeta de cr\'edito o d\'ebito [10][11] & Sin costo \\ \hline
			
			“School control” & Aplicaci\'on m\'ovil que realiza reportes, asignaci\'on de pagos, consulta de estad\'isticas en tiempo real, monitoreo/edici\'on de la informaci\'on escolar, entrega de calificaciones, control de asistencia y comportamiento de los alumnos entre otras soluciones dise\~nadas espec\'ificamente para colegios que les deja tiempo valioso para dedicarlo a lo que verdaderamente saben hacer, que es enseñar [12][13]& desde \$149 por alumno al a\~no, costo absorbido totalmente por el colegio. \\ \hline
			
			“Aplicaci\'on escolar” & Es una aplicaci\'on m\'ovil para dispositivos Android e IOS en la cual las escuelas pueden enviar informaci\'on como mensajes de pagos, tareas, circulares, así como seguimientos acad\'emicos y calificaciones graficadas directo al celular de los padres o alumnos con notificación tipo WhatsApp. Los pap\'as descargan la aplicaci\'on con el nombre de su colegio desde Play Store o App Store ya que es personalizada a cada escuela [14][15]& Pago inicial de \$31,000, posterior a la prueba piloto se cobran \$6,000 mensuales por cada 400 alumnos. El costo es absorbido por el colegio. \\ \hline		
		\end{tabular}
		\caption{Sistemas o aplicaciones relacionadas}
		\label{tabla:Productos}
	\end{center}
\end{table}

\newpage
De estos sistemas encontramos las siguientes ventajas y desventajas considerando el contexto bajo el cual nosotros estaremos trabajando:
\begin{itemize}
	\item {\bf SISTEMA DE MONEDERO VIRTUAL PARA PAGOS ESCOLARES}
	\begin{itemize}
		\item Ventajas:
		\begin{itemize}
			\item Permite la realización de pagos de forma electrónica.
		\end{itemize}
		\item Desventajas:
		\begin{itemize}
			\item Carece de un diseño responsivo.
			\item Para realizar el pago es necesario imprimir un comprobante.
			\item Se necesita efectuar un abono previo.
			\item Sólo se planteó como un prototipo.
		\end{itemize}
	\end{itemize}
	\item {\bf TTR-12-1-029 Prototipo para el manejo de Cero Papel}
	\begin{itemize}
		\item Ventajas:
		\begin{itemize}
			\item Disminución del uso de papel.
			\item Control y seguridad de documentos.
		\end{itemize}
		\item Desventajas:
		\begin{itemize}
			\item Solo es un sistema que busca la optimización de recurso material (papel).
		\end{itemize}
	\end{itemize}
	\item {\bf Ventanilla Virtual UdeG}
	\begin{itemize}
		\item Ventajas:
		\begin{itemize}
			\item Presenta distintos medios de acceso como kioscos interactivos, sitio web y aplicación móvil.
			\item Pueden realizar, seguir, recuperar y generar los pagos en algunos trámites.
			\item Permite al alumno la consulta de información académica.
			\item Pago digital.
		\end{itemize}
		\item Desventajas:
		\begin{itemize}
			\item El precio de cada kiosco interactivo es de 165 mil pesos.
			\item Solo los estudiantes tienen acceso.
		\end{itemize}	
	\end{itemize}
	\item {\bf Campus Pay}
	\begin{itemize}
		\item Ventajas:
		\begin{itemize}
			\item Permite a toda la comunidad universitaria realizar pagos de forma electrónica.
			\item Permite el pago a todas las áreas que lo requieran.
			\item Permite al alumno la consulta de información académica.
			\item No tiene costo para los usuarios de la aplicación.
			\item Permite cualquier tarjeta de débito o crédito.
		\end{itemize}
		\item Desventajas:
		\begin{itemize}
			\item Genera un costo en la institución educativa en la que se implementa.
		\end{itemize}
	\end{itemize}	
	\item {\bf School control}
	\begin{itemize}
		\item Ventajas:
		\begin{itemize}
			\item Permite el pago en línea.
			\item Elimina las comisiones de tarjeta de crédito.
			\item Realiza métricas de información sobre el colegio.
			\item Permite una administración de accesos al sistema.
			\item Tiene un módulo de apoyo para maestros.
		\end{itemize}
		\item Desventajas:
		\begin{itemize}
			\item Tiene un costo básico de \$149.00 por alumno.
		\end{itemize}
	\end{itemize}	
	
	\item {\bf Aplicación escolar}
	\begin{itemize}
		\item Ventajas:
		\begin{itemize}
			\item Permite mensajes de pagos.
			\item Permite seguimientos académicos.
			\item Genera notificaciones.
			\item Aplicación personalizada por institución.
			\item Tiene un módulo de apoyo para maestros.
		\end{itemize}
		\item Desventajas:
		\begin{itemize}
			\item Pago inicial de \$31,000, posterior a la prueba piloto se cobran \$6,000 mensuales por cada 400 alumnos.
		\end{itemize}			
	\end{itemize}
\end{itemize}

Contextualizando todos estos sistemas y aplicaciones a nuestro trabajo, nos percatamos que la {\bf Ventanilla Virutal UdeG} y la aplicación móvil {\bf Campus pay} son los modelos de referencia que tenemos para el desarrollo y mejora de nuestro proyecto. Entendemos que son bastante funcionales, pero también creemos que se encuentran carentes de escalabilidad e inclusión hacia otros sistemas para la gestión de procesos.\\

\section{Marco Teórico}
Para poder comprender mejor el desarrollo de este Trabajo Terminal definiremos algunos conceptos principales a partir de los cuales basamos nuestro proyecto. Los conceptos a considerar son: aplicación web; aplicación móvil; Interfaz de Programación de Aplicaciones (API); aplicación móvil nativa; aplicaciones web sobre móviles; aplicación móvil híbrida; Modelo Vista Controlador; framework; Spring Boot; Struts 2; Hibernate ORM; y por último, Xamarin.\\

\subsection{Aplicación Web}
Una aplicación web (web-based application) es un tipo especial de aplicación cliente/servidor, donde tanto el cliente (el navegador, explorador o visualizador) como
el servidor (el servidor web) y el protocolo mediante el que se comunican (HTTP)
están estandarizados y no han de ser creados por el programador de aplicaciones.\\

El protocolo HTTP forma parte de la familia de protocolos de comunicaciones TCP/IP, que son los empleados en Internet. Estos protocolos permiten la conexión de sistemas heterogéneos, lo que facilita el intercambio de información entre distintos ordenadores.\\

\begin{itemize}
	\item {\bf El cliente: }El cliente web es un programa con el que interacciona el usuario para solicitar a un servidor web el envío de los recursos que desea obtener mediante HTTP.
	\item {\bf El servidor: }El servidor web es un programa que está esperando permanentemente las solicitudes	de conexión mediante el protocolo HTTP por parte de los clientes web. [16]
\end{itemize}

Algunos de los usos principales de una aplicación web son los siguientes:
\begin{itemize}
	\item Permitir a los usuarios localizar información de forma rápida y sencilla en un sitio Web en el que se almacena gran cantidad de contenido.
	\item Recoger, guardar y analizar datos suministrados por los visitantes de los sitios.
	\item Actualizar sitios Web cuyo contenido cambia constantemente. [17]
\end{itemize}

\subsection{Aplicación móvil}
Una aplicación móvil es una aplicación informática desarrollada para ser ejecutada en teléfonos inteligentes, tabletas y otros dispositivos móviles.\\

Por lo general, se encuentran disponibles a través de plataformas de distribución, operadas por las compañías propietarias de los sistemas operativos móviles como Android, iOS, BlackBerry OS, Windows Phone, entre otros. Existen aplicaciones móviles gratuitas u otras de pago, donde en promedio el 20 a 30\% del costo de la aplicación se destina al distribuidor y el resto es para el desarrollador.

Existen tres tipos de aplicaciones que se pueden desarrollar:
\begin{itemize}
	\item Aplicaciones móviles nativas.
	\item Aplicaciones Web.
	\item Aplicaciones Híbridas.
\end{itemize}

Cada uno de estos tipos de aplicaciones se explicarán a detalle.\\

\subsection{Interfaz de Programación de Aplicaciones (API)}
Una Interfaz de Programación de Aplicaciones es una especificación destinada a ser utilizada como interfaz de componentes de software para comunicarse entre ellos. Una API puede incluir especificaciones para rutinas, estructuras de datos, objetos clases y variables.\\

Una API puede describir las formas en que se realiza una tarea en particular. Por lo tanto, la API generalmente incluye una descripción de todas las funciones o rutinas que proporciona.\\

En los lenguajes orientados a objetos, una API generalmente incluye una descripción de un conjunto de definiciones de clase, con un conjunto de comportamientos asociados con esas clases. La API en este caso se puede concebir como la totalidad de todos los métodos públicamente expuestos por las clases (generalmente llamado la interfaz de clase). Esto significa que la API prescribe los métodos por los cuales uno interactúa con los objetos derivados de las definiciones de clase. [18]\\

\subsection{Aplicación móvil nativa}
Las aplicaciones nativas tienen archivos ejecutables binarios que se descargan directamente al dispositivo y se almacenan localmente.  La manera más común de descargar una aplicación nativa es visitando una tienda de aplicaciones, como App Store de Apple, PlayStore de Android o App World de BlackBerry, pero existen otros métodos
que a veces ofrece el proveedor móvil. La aplicación nativa puede acceder libremente a todas las APIs que el proveedor del SO ponga a disposición y, en muchos casos, tiene características y funciones únicas que son típicas de ese sistema operativo móvil en particular.\\

Para crear una aplicación nativa, los desarrolladores deben escribir el código fuente y crear recursos adicionales, como imágenes, segmentos de audio y diversos archivos de declaración específicos del sistema operativo. Utilizando herramientas provistas por el distribuidor del sistema operativo, se compila el código fuente para crear un ejecutable en formato binario que se pueda empaquetar junto con el resto de los recursos y estar listo para la distribución.\\

Si bien el proceso de desarrollo suele ser similar para diferentes sistemas operativos, el Software Development Kit (SDK) es específico de la plataforma, y cada sistema operativo móvil viene con sus propias herramientas. En la tabla \ref{tabla:SDK} se muestran los formatos de aplicación para cada sistema operativo.\\
\begin{table}[htbp]
	\begin{center}
		\begin{tabular}{p{25mm}|p{25mm}|p{25mm}|p{30mm}}
			\hline
			 {\bf Apple iOS} &  {\bf Android} & {\bf Blackberry OS} & {\bf Windows Phone} \\
			\hline
			 .app & .apk & .cod &.xap\\ \hline		
		\end{tabular}
		\caption{Formatos de aplicación.}
		\label{tabla:SDK}
	\end{center}
\end{table}

Estas diferencias entre plataformas ocasionan una de las desventajas más criticas del enfoque de desarrollo nativo: el código
escrito para una plataforma móvil no se puede usar en otra, por lo cual el desarrollo y el mantenimiento de aplicaciones nativas para
múltiples sistemas operativos se convierte en una tarea muy ardua y costosa. [19]

\subsection{Aplicaciones Web sobre Móviles}
Las aplicaciones web sobre móviles son aplicaciones que no necesitan ser instaladas en el dispositivo para poder ejecutase. Están basadas en tecnologías HTML, CSS y Javascript, que se ejecutan en un navegador. [20]\\

Una de las principales ventajas de una aplicación Web es su soporte para múltiples plataformas y el bajo costo de desarrollo.
La mayoría de los proveedores móviles utilizan el mismo motor de búsqueda en sus navegadores, llamado WebKit, que es un proyecto de fuente abierta conducido principalmente por Google y Apple y que ofrece la más completa implementación de HTML5 disponible en la actualidad. [19]\\

\subsection{Aplicación móvil híbrida}
El enfoque híbrido combina desarrollo nativo con tecnología Web. Usando este enfoque, los desarrolladores escriben gran parte de su aplicación en tecnologías Web para múltiples plataformas, y mantienen el acceso directo a APIs nativas cuando lo necesitan.\\

El enfoque híbrido ofrece un término medio que, en muchas situaciones, constituye lo mejor de ambos mundos, en especial si el desarrollador desea emplearlo en múltiples sistemas operativos. [19]\\

\subsection{Modelo Vista Controlador (MVC)}
MVC por sus siglas en inglés, es un patrón de diseño de arquitectura de software usado principalmente en aplicaciones que manejan gran cantidad de datos y transacciones complejas donde se requiere una mejor separación de conceptos para que el desarrollo esté estructurado de una mejor manera, facilitando la programación en diferentes capas de manera paralela e independiente. MVC sugiere la separación del software en 3 estratos: Modelo, Vista y Controlador.  [21]\\

\begin{itemize}
	\item {\bf Modelo: }Es la representación de la información que maneja la aplicación.
	\item {\bf Vista: }Es la representación del modelo en forma gráfica disponible para la interacción con el usuario.
	\item {\bf Controlador: }Es la capa encargada de manejar y responder las solicitudes del usuario, procesando la información necesaria y modificando el Modelo en caso de ser necesario.
\end{itemize}

Este modelo de arquitectura presenta varias ventajas [22]:
\begin{itemize}
	\item Separación clara entre los componentes de un programa; lo cual permite su implementación por
	separado.
	\item API muy bien definida;
	cualquiera que use el API, podrá reemplazar el Modelo, la Vista o el Controlador, sin aparente dificultad.
	\item Conexión entre el Modelo y sus Vistas dinámica; se produce en tiempo de ejecución, no en tiempo de compilación. 
\end{itemize}

\subsection{Framework}
Un framework no es ningún software ni herramienta que se ejecuta y que nos ofrece una interfaz gráfica desde la que trabajar, sino que es un conjunto de archivos y
directorios que facilitan la creación de aplicaciones, ya que incorporan funcionalidades ya desarrolladas y
probadas, implementadas en un determinado lenguaje de programación.\\

El objetivo principal de todo framework es facilitar las cosas a la hora de desarrollar una aplicación, haciendo que nos centremos en el verdadero problema y nos olvidemos de implementar funcionalidades que son de uso común como puede ser el registro de un usuario, establecer conexión con la base de datos, manejo de sesiones de usuario o el almacenamiento en base de datos de contenido cacheado. [23]\\

\subsection{Spring Boot}
Spring Boot es un framework que facilita la creación de aplicaciones independientes, de grado de producción, que simplemente se ejecutan. La mayoría de las aplicaciones Spring Boot necesitan muy poca configuración de Spring.\\

Se recomienda que se use una herramienta de compilación que admita la administración de dependencias (como Maven o Gradle). [24]\\

Apache Maven es una herramienta de gestión y comprensión de proyectos de software. Basándose en el concepto de un modelo de objeto de proyecto (POM), Maven puede gestionar la compilación, los informes y la documentación de un proyecto a partir de una pieza central de información.\\

\subsection{Struts 2}
El framework web de Apache Struts es una solución gratuita de código abierto para la creación de aplicaciones web en Java.\\

Favorece la convención sobre la configuración, es extensible usando una arquitectura de complemento y se envía con complementos para admitir REST, AJAX y JSON. [26]\\

\subsection{Hibernate ORM}
Hibernate ORM permite a los desarrolladores escribir aplicaciones con mayor facilidad, cuyos datos sobreviven al proceso de solicitud. Hibernate se preocupa por la persistencia de los datos tal como se aplica a las bases de datos relacionales (a través de JDBC).\\

Hibernate es una herramienta de Mapeo Objeto-Relacional (ORM) para la plataforma JAVA que facilita el mapeo de atributos entre una base de datos relacional tradicional y el modelo de objetos de una aplicación, mediante archivos declarativos (XML) o anotaciones en los beans de las entidades que permiten establecer este tipo de relaciones. [27]\\

\subsection{Xamarin}
Xamarin es una plataforma que proporciona al desarrollador herramientas que pueden ayudarlo a crear aplicaciones móviles multiplataforma. Las aplicaciones pueden tener todas las características nativas y también compartir la base de código común al mismo tiempo.\\

Xamarin permite llamar al código existente escrito en otros lenguajes específicos de la plataforma, como Java en Android. Pero eso es solo cuando se está construyendo algo muy específico que no se puede implementar en diferentes plataformas. Xamarin ha convertido todo el SDK de Android e iOS a C\# para que se pueda codificar en un lenguaje más familiar. Y como se puede usar C\# para codificar ambas plataformas, se necesita recordar menos sintaxis. Se Puede acceder a casi cualquier API de iOS o Android en C\# con las herramientas de Xamarin. [28]\\

\subsection{PostgreSQL}
PostgreSQL es un sistema de base de datos objeto-relacional potente y abierto que utiliza y amplía el lenguaje SQL combinado con muchas características que almacenan y escalan de forma segura las cargas de trabajo de datos más complicadas.\\

PostgreSQL se ha ganado una sólida reputación por su arquitectura comprobada, confiabilidad, integridad de datos, sólido conjunto de características, extensibilidad y la dedicación de la comunidad de código abierto detrás del software para entregar constantemente soluciones eficaces e innovadoras. PostgreSQL se ejecuta en todos los principales sistemas operativos. [29]

