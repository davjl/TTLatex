% Chapter 2

\chapter{Antecedentes}
\label{Capitulo 2} 

En este cápitulo se da a conocer un análisis sobre el funcionamiento y problematicas  que presenta la Escuela Superior de Cómputo, así como también la descripción de propuestas que podrían satisfacer las necesidades tomando en cuenta los pros y contras de cada una.
  
\section{Estado del arte}

Durante nuestra investigaci\'on de mercado, nos encontramos con sistemas que tienen una fuerte relaci\'on con el proyecto a trabajar pero implementado en situaciones distintas, adem\'as de tener funcionalidades variadas.\\
Los sistemas con mayor similitud los podemos encontrar en la tabla \ref{tabla:Productos}

\begin{table}[htbp]
\begin{center}
\begin{tabular}{|p{25mm}|p{75mm}|p{25mm}|}
\hline
Software & Caracter\'isticas & Costo \\
\hline \hline
"SISTEMA DE MONEDERO VIRTUAL PARA PAGOS ESCOLARES" & El Sistema de Monedero Virtual para Pagos Escolares es un sistema de prepago, que permite a los alumnos realizar pagos dentro y fuera de la Unidad Profesional Interdisciplinaria en Ingenier\'ia y Tecnolog\'ias Avanzadas (UPIITA) & Sin costo \\ \hline

TTR-12-1-029 Prototipo para el manejo de “Cero Papel”. & Es un sistema que permite el manejo, intercambio y control de la informaci\'on dentro de una organizaci\'on para optimizar los procedimientos y tareas, disminuyendo el uso de papel mediante la implementaci\'on de un sistema que permita administrar los usuarios y los documentos & Sin costo\\ \hline

“Ventanilla Virtual UdeG” & Ventanilla Virtual tiene tecnolog\'ias desarrolladas por universitarios y busca brindar una plataforma para que, en una primera etapa, los estudiantes puedan hacer tr\'amites, dar seguimiento, recuperarlos y hacer pagos respectivos & Sin costo \\ \hline

“Campus Pay” & Desarrollo que ofrece a estudiantes la posibilidad de realizar todos los pagos relacionados con sus estudios desde dispositivos m\'oviles con cualquier tarjeta de cr\'edito o d\'ebito & Sin costo \\ \hline

“School control” & Aplicaci\'on m\'ovil que realiza reportes, asignaci\'on de pagos, consulta de estad\'isticas en tiempo real, monitoreo/edici\'on de la informaci\'on escolar, entrega de calificaciones, control de asistencia y comportamiento de los alumnos entre otras soluciones dise\~nadas espec\'ificamente para colegios que les deja tiempo valioso para dedicarlo a lo que verdaderamente saben hacer, que es enseñar & desde \$149 por alumno al a\~no, costo absorbido totalmente por el colegio. \\ \hline

“Aplicaci\'on escolar” & Es una aplicaci\'on m\'ovil para dispositivos Android e IOS en la cual las escuelas pueden enviar informaci\'on como mensajes de pagos, tareas, circulares, así como seguimientos acad\'emicos y calificaciones graficadas directo al celular de los padres o alumnos con notificación tipo WhatsApp. Los pap\'as descargan la aplicaci\'on con el nombre de su colegio desde Play Store o App Store ya que es personalizada a cada escuela & Pago inicial de \$31,000, posterior a la prueba piloto se cobran \$6,000 mensuales por cada 400 alumnos. El costo es absorbido por el colegio. \\ \hline

\end{tabular}
\caption{Productos similares}
\label{tabla:Productos}
\end{center}
\end{table}

\newpage
De estos sistemas encontramos las siguientes ventajas y desventajas:
\begin{itemize}
	\item {\bf SISTEMA DE MONEDERO VIRTUAL PARA PAGOS ESCOLARES}
	\begin{itemize}
		\item Ventajas:
		\begin{itemize}
			\item Permite la realización de pagos de forma electrónica.
		\end{itemize}
		\item Desventajas:
		\begin{itemize}
			\item Presenta un diseño poco responsivo.
			\item Para realizar el pago es necesario imprimir un comprobante.
			\item Se necesita realizar un abono previo.
		\end{itemize}
	\end{itemize}
	\item {\bf TTR-12-1-029 Prototipo para el manejo de Cero Papel}
	\begin{itemize}
		\item Ventajas:
		\begin{itemize}
			\item Disminución del uso de papel.
			\item Control y seguridad de documentos.
		\end{itemize}
		\item Desventajas:
		\begin{itemize}
			\item Solo es una sistema de ahorro de papel.
		\end{itemize}
	\end{itemize}
	\item {\bf Ventanilla Virtual UdeG}
	\begin{itemize}
		\item Ventajas:
		\begin{itemize}
			\item Presenta distintos medios de acceso como kioscos interactivos, sitio web y aplicación móvil.
			\item Pueden realizar, seguir, recuperar y generar los pagos en algunos trámites.
			\item Permite al alumno la consulta de información académica.
			\item Pago digital.
		\end{itemize}
		\item Desventajas:
		\begin{itemize}
			\item El precio de cada kiosco interactivo es de 165 mil pesos.
			\item Solo los estudiantes tienen acceso.
		\end{itemize}	
	\end{itemize}
	\item {\bf Campus Pay}
	\begin{itemize}
		\item Ventajas:
		\begin{itemize}
			\item Permite a toda la comunidad Universitaria realizar pagos de forma electrónica.
			\item Permite el pago a todas las áreas que lo requieran.
			\item Permite al alumno la consulta de información académica.
			\item No tiene costo.
			\item Permite cualquier tarjeta.
		\end{itemize}
		\item Desventajas:
		\begin{itemize}
			\item Genera un costo en la institución educativa en la que se utiliza.
		\end{itemize}
	\end{itemize}	
	\item {\bf School control}
	\begin{itemize}
		\item Ventajas:
		\begin{itemize}
			\item Permite el pago en línea.
			\item Elimina las comisiones de tarjeta de crédito.
			\item Realiza metricas de información sobre el colegio.
			\item Permite una administración de accesos al sistema.
			\item Tiene un módulo de apoyo para maestros.
		\end{itemize}
		\item Desventajas:
		\begin{itemize}
			\item Tiene un costo basico de \$149.00 por alumno.
		\end{itemize}
	\end{itemize}	
	
	\item {\bf Aplicación escolar}
	\begin{itemize}
		\item Ventajas:
		\begin{itemize}
			\item Permite mensajes de pagos.
			\item Permite seguimientos académicos.
			\item Genera Notificaciónes.
			\item Aplicación personalizada por institución.
			\item Tiene un módulo de apoyo para maestros.
		\end{itemize}
		\item Desventajas:
		\begin{itemize}
			\item Pago inicial de \$31,000, posterior a la prueba piloto se cobran \$6,000 mensuales por cada 400 alumnos.
		\end{itemize}			
	\end{itemize}
\end{itemize}

De todos los sistemas presentados, {\bf Ventanilla Virutal UdeG} y {\bf Campus pay} son considerados como los módelos de mayor importancia con respecto a sus beneficios y funcionalidades que presenta. Sin embargo, el uso de estos servicios genera un gasto económico en la obtención de equipo y licencias dentro de la institución.


\section{Situación Actual}

Para una idea clara del problema que se ataca, es necesario dar una explicación del actual funcionamiento de trabajo y comunicaci\'on entre las \'areas de servicios y recursos financieros.\\
Comencemos con el \'area de servicios dentales, el cual al recibir un servicio dental, es necesario llevar una nota de pago otorgada por el/la dentista en turno y realizar el pago de la misma en \'area de caja otorgando informaci\'on de boleta para casos de personal institucional y curp en caso de personal externo, seguido a esto se nos otorgar\'a un comprobante de pago realizado, el cual tendr\'a que ser llevado al \'area de dentales y ser almacenado por el dentista.\\
Junto a esto, el archivo de expediente cl\'inico para el alumno sera almacenado por reglamento por un periodo de 5 años.
Siguiendo con otro departamento, CELEX maneja un proceso de diferente en comparaci\'on con dentales al tener la necesidad de tener que realizar un pago en una sucursal BANCOMER y llevar el vaucher de pago  a caja, anotando en el vaucher informaci\'on que se requiere para la inscripci\'on (nombre,boleta,curp, idioma y nivel) , generando un nuevo comprobante que sera llevado al departamento de CELEX donde se terminara el proceso de inscripci\'on.

\section{Metodología}
	Para el desarrollo del sistema utilizaremos la metodologia incremental de Harlan Mills, ésta se basa en la idea de diseñar una implementación inicial, exponerla al comentario del usuario, y luego desarrollarla en sus diversas versiones hasta producir un sistema adecuado. %referenciar
	
	Se tomo en cuenta esta metodología por los siguientes beneficios:
	\begin{itemize}
		\item Permite descomponer el proyecto en varios incrementos aislados.
		\item En cada incremento se incorporan los requisitos básicos. 
		\item Es posible realizar un trabajo en paralelo por parte de los integrantes del equipo.
		\item Es sencillo obtener retroalimentación de los directores y coordinadores de área.
	\end{itemize}

\subsection{Propuesta de software}
Dentro de la arquitectura de nuestro sistema encontraremos la interacción de varios elementos que llevaran acabo una función importante para el correcto funcionamiento. Entre estos elementos estará un servidor web para el alojamiento de la pagina y una base de datos para la persistencia de información ORM, su core y módulo Spring MVC. \\ 

\subsection{Spring Boot}
Spring Boot es un framework que se ha empleado en este trabajo terminal debido a que su arquitectura se adapta a los requerimientos de este sistema, tiene ciertas ventajas que con ayuda de otras herramientas como Apache Maven nos han ayudado a levantar una infraestructura de manera relativamente sencilla, los módulos más reelevantes utilizados para este trabajo son: el módulo de mails, el módulo de tratamiento de datos. \\

Otro de los enfoques utilizados para el desarrollo de este trabajo es la inyección de dependencias, el poder inyectar dependencias dentro del proyecto permite dividir el proyecto de forma más natural en MVC (modelo, vista, controlador), se ha incluido derivado de ello el uso de anotaciónes para incluir referencias a servicios en capas más internas del sistema, así en conjunto con las tecnologías como Hibernate se ha podido contruir una infraestructura estable. \\

\subsection{Struts 2}
Una de las ventajas que encontramos en usar Struts 2 es la fácil inclusión de controladores con la vista, el presente sistema utiliza una arquitectura mvc, cabe mencionar que entre otras tecnologías utiliza para maquetar pantallas Bootstrap esto en conjunto permite crear vistas complejas con una complejidad baja y al mismo tiempo gracias a Struts2 icluir funcionalidades más robustas.\\

Las acciones de Struts 2 implementan objetos JavaBeans (clases Java simples) para cada grupo de datos enviado en la consulta. Cada parámetro de la consulta se declara en la clase de acción con un nombre idéntico para realizar automáticamente la asignación de valores. La nalidad de la acción es devolver una cadena de caracteres, permitiendo seleccionar el resultado que se va a mostrar.\\

\subsection{Hibernate}
Hibernate es una herramienta de Mapeo Objeto-Relacional (Object-Relational Mapping) para la plataforma JAVA que facilita el mapeo de atributos entre una base de datos relacional tradicional y el modelo de objetos de una aplicación, mediante archivos declarativos (XML) o anotaciones en los beans de las entidades que permiten establecer este tipo de relaciones.
Cuando desarrollamos aplicaciones en muchos de los casos ocurre que en muchas secciones todo termina siendo un conjunto de ABM (alta, baja y modicaciones de datos) que luego consultamos. Para ello se utiliza una base de datos donde hay muchas tareas repetidas: por cada objeto que quiero persistir debo crear una clase que me permita insertarlo, eliminarlo, modificarlo y consultarlo. Con excepción de consultas especiales, el resto es siempre lo mismo.
La solución que tenemos ante esto es usar un ORM para poder ecientar las tareas y reducir todos los pasos que se mencionaron antes. Con solo congurar correctamente los archivos usados por Hibernate todas estas tareas se ejecutarían automáticamente y sólo tendremos que preocuparnos por las consultas especiales.