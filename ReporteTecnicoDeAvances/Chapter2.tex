% Chapter 2

\chapter{Antecedentes}
\label{Capitulo 2} 
En este capítulo se habla del actual proceso de pagos que se realiza en la ESCOM y se menciona de forma general el problema que existe derivado de esto. Se proporciona un marco teórico para entender más los conceptos y tecnologías que se utilizan. \\

Se plantea el objetivo general del proyecto y los avances logrados en la primera parte del Trabajo Terminal. Se incluyen las observaciones hechas por parte de los sinodales y directores, así como una descripción de las soluciones dadas a cada una de ellas. 

\section{Situación Actual}
Para entender el actual proceso de pagos es necesario hablar de los servicios que proporcionan las áreas de CELEX, dentales, biblioteca y fotocopiado. Así también, es importante mencionar cuál es el fin de estos servicios y lo que obtiene la ESCOM por los pagos que se realizan por ellos.\\

Los servicios que proporcionan cada una de las áreas forman parte de los ingresos auto generados de la ESCOM, con el objetivo de obtener recursos económicos independientes del presupuesto federal asignado. Estos ingresos que se obtienen se utilizan principalmente para la manutención de las instalaciones, tales como, laboratorios de cómputo, aulas de clase, sanitarios, zonas verdes, entre otros.\\

Lo anterior, no sólo lo realiza la ESCOM, también las demás escuelas que conforman a todo el IPN, pero cada una de estas definen qué productos o servicios pondrán a disposición de sus estudiantes, empleados o externos. Todos esos productos o servicios se encuentran en un catálogo de cuotas de productos y cuotas de aprovechamiento de Cobro Regular emitido por la Secretaría de Hacienda y Crédito Público (SHCP) a través del sistema DEPAMIN: "Módulo de Estimación de Ingresos por Concepto de Derechos, Productos y Aprovechamientos" \cite{DEPAMIN} \\

Dicho catálogo se renueva año con año presentando una relación aproximadamente de 1200 cuotas de productos y 19 cuotas de aprovechamientos de Cobro Regular \cite{catalogo}. Cada una de esas cuotas es incorporada en el Sistema Institucional de Gestión Administrativa (SIG@) del IPN \cite{SIGA}, poniendo a disposición un comprobante SIG@, relacionada con cada cuota.\\

ESCOM, en lo particular toma sólo algunos de esos conceptos descritos en el catálogo y los ofrece a toda su comunidad, incluyendo también a externos. Suele tener una clasificación y un área responsable para cada servicio o producto que oferta, estas clasificaciones son las siguientes:\\

\begin{itemize}
	\item Cursos de educación continua
	\begin{itemize}
		\item {\bf Área responsable: }Coordinación de Educación Continua (CEC) \cite{coordinacionContinua}.
	\end{itemize}
	\item Posgrado
	\begin{itemize}
		\item {\bf Área responsable: }Sección de Estudios de Posgrado e Investigación (SEPI) \cite{SEPI}.
	\end{itemize}	
	\item CELEX
	\begin{itemize}
		\item {\bf Área responsable: }Coordinación de Cursos Extracurriculares de Lenguas Extranjeras \cite{CELEX}.
	\end{itemize}	
	\item Odontológicos
	\begin{itemize}
		\item {\bf Área responsable: }Servicio Dental.
	\end{itemize}
	\item Productos
	\begin{itemize}
		\item {\bf Área responsable: }Fotocopiado y Biblioteca \cite{productos}.
	\end{itemize}
\end{itemize}

En total son 36 conceptos de pago los que ESCOM considera para la oferta de servicios o productos repartidos en cada una de las clasificaciones anteriores. Los fondos recabados mediante el pago en caja, pago en sucursal bancaria o transferencia electrónica son depositados en una cuenta de débito de BBVA Bancomer a nombre de Fondo de investigación cientifíca y desarrollo tecnológico - ESCOM, misma que el IPN monitorea para la declaración de ingresos, por lo tanto, es necesario que el departamento de Recursos Financieros de la ESCOM tenga siempre disponibles todas los comprobantes SIG@ emitidas durante cada periodo escolar.\\

Tan es así, que se exige conservarlas por un periodo de cinco años a partir del año en que se expide. Esta regla no sólo aplica para el departamento de Recursos Financieros, también para cada una de las áreas de donde derivan la prestación de los productos o servicios.\\

Si bien comentamos que esas boletas de pago se deben de almacenar por un periodo de cinco años, hasta el momento, ninguna de las áreas tiene claro qué es lo que sucede con toda esa documentación una vez finalizado dicho periodo. Actualmente, sólo las almacenan en carpetas y se conservan en los estantes.\\

Todos los ingresos obtenidos durante el día deberán de comprobarse, puesto que se genera un corte de caja de manera diaria que permite saber el total generado, que a su vez, es notificado al Subdirector Administrativo quien será el último en dar el visto bueno para los ingresos obtenidos del día corriente.

\subsection{Pagos en ESCOM}
El proceso de pago en ESCOM se lleva prácticamente de la misma manera en todos los servicios o productos que se ofertan, partiendo de la realización del pago en caja, sucursal bancaria o transferencia electrónica y concluyendo en la entrega al interesado por el producto o servicio de un comprobante SIG@ la cual se da posteriormente al área involucrada como garantía de que se efectúo correctamente la cuota. \IUref{}{Proceso de pago Biblioteca, CELEX, Fotocopiado} \newpage %Pueden existir algunas variantes en este proceso, pero no afectan para nada el flujo principal comentado anteriormente.\\

\IUfig[1]{gui/pago}{}{Proceso de pago Biblioteca, CELEX, Fotocopiado}

Si bien este es el flujo comun de la mayoria de las áreas de servicio también pueden existir algunas variantes en este proceso como es el caso de área dental el cual ponemos ver en la siguiente figura \IUref{}

\IUfig[1]{gui/dental}{}{Proceso de pago área dental}

Enfocados en el alcance del proyecto daremos el detalle del proceso de pago para cada una de las áreas que estamos considerando, además de mostrar su catálogo de servicios disponibles y sus costos para este año en curso. Esto se ha escrito derivado de múltiples entrevistas con los responsables de las distintas áreas.\\

Con base en el catalogo de productos y cuotas de aprovechamiento de Cobro Regular emitido por la SHCP a través del sistema DEPAMIN \cite{DEPAMIN}, la Escuela Superior de Cómputo y el presente trabajo terminal hace uso de los siguientes servicios descritos a continuación:

\subsubsection{Servicio Dental}


Comencemos con el área de servicios dentales, sus costos se muestran en la tabla \ref{tabla:Dentales}.\\

\begin{table}[htbp]
	\begin{center}
		\begin{tabular}{|p{110mm}|p{25mm}|}
			\hline
			{\bf Producto/Servicio} &  {\bf Monto} \\
			\hline
			Servicio Médico de Aplicación de Amalgama. Estudiantes del IPN & \$ 74.00\\ \hline
			Servicio Médico de Aplicación de Amalgama. Público en General & \$ 142.00\\ \hline
			Servicio Médico de Aplicación de Amalgama. Trabajadores del IPN & \$ 101.00\\ \hline
			Servicio Médico de Cementación Incrustación. Estudiantes del IPN y Comunidad del IPN & \$ 68.00\\ \hline
			Servicio Médico de Consulta de Odontología. Comunidad del IPN & \$ 27.00\\ \hline
			Servicio Médico de Curación. Estudiantes del IPN & \$ 27.00\\ \hline
			Servicio Médico de Curación. Comunidad del IPN & \$ 74.00\\ \hline
			Servicio Médico de Curación. Trabajadores & \$ 47.00\\ \hline
			Servicio Médico de Curación. Trabajadores & \$ 59.00\\ \hline
			Servicio Médico de Extracción de piezas dentales. Docentes y empleados & \$ 66.00\\ \hline
			Servicio Médico de Extracción de piezas dentales. Docentes y empleados & \$ 180.00\\ \hline
			Servicio Médico de Impresión Parcial (hilo retractor, yeso velmix, hule, rectificador, alginato, yeso piedra y cera) & \$ 81.00\\ \hline
			Servicio Médico de Profilaxis. Estudiantes del IPN & \$ 108.00\\ \hline
			Servicio Médico de Profilaxis. Comunidad Politécnica & \$163.00\\ \hline
			Servicio Médico de Profilaxis. Público en General & \$ 196.00\\ \hline
			Servicio Médico de Resina autopolimerizable. Estudiantes del IPN & \$ 155.00\\ \hline
			Servicio Médico de Resina autopolimerizable. Docentes y empleados & \$ 175.00\\ \hline
			Servicio Médico de Resina chica & \$ 41.00\\ \hline
			Servicio Médico de Resina. Comunidad Politécnica & \$ 172.00\\ \hline		
		\end{tabular}
		\caption{Catálogo de servicios dentales.}
		\label{tabla:Dentales}
	\end{center}
\end{table}

En este caso, cuando un usuario requiere de algún servicio odontológico se presenta en el consultorio sin cita previa para la atención con algún dentista en turno, después, éste hace un análisis del problema del usuario, sin previo pago realiza el servicio diagnosticado o solicitado para finalmente realizar una nota de pago que le entrega al usuario la cual debe de presentar a caja para efectuar su pago en efectivo. Una vez realizado, el cajero genera e imprime en dos ocasiones el comprobante SIG@ para después entregarle uno de ellos al usuario y otro para su resguardo. Este comprobante es el que el usuario entrega al dentista para confirmar su pago, mismo que almacena por políticas de la institución. Si el usuario necesita más servicios odontológicos el proceso se repite.\\

\subsubsection{CELEX}
Ahora bien, hablemos del área de CELEX, mencionando primero sus costos. Estos se muestran en la tabla \ref{tabla:Celex}.\\
\newpage
\begin{table}[htbp]
	\begin{center}
		\begin{tabular}{|p{110mm}|p{25mm}|}
			\hline
			{\bf Producto/Servicio} &  {\bf Monto} \\
			\hline
			Curso de Idiomas semanal o sabatino, 40 horas. Comunidad IPN & \$ 544.00\\ \hline
			Curso de Idiomas semanal o sabatino, 40 horas. Público en General & \$ 1,053.00\\ \hline			
		\end{tabular}
	\caption{Catálogo de servicios CELEX.}
\label{tabla:Celex}
\end{center}
\end{table}

Para esta área, el usuario tiene que efectuar el pago directamente en cualquier sucursal bancaria BBVA Bancomer a la cuenta que previamente se le otorgó acudiendo a las oficinas del CELEX. Es importante decir, que también se puede efectuar el pago a través de una transferencia electrónica hacia el mismo banco. Aquí, no son permitidos los pagos directamente en la caja de la ESCOM.\\

Una vez efectuado el pago el usuario debe de acudir a la caja de la ESCOM para presentar el voucher emitido por el banco, o bien la impresión que compruebe la transferencia electrónica. Hecho eso, el cajero emite el comprobante SIG@ y la imprime en dos ocasiones, una la entrega al usuario como garantía de su pago y otra la deja para su resguardo. El usuario acude a las oficinas del CELEX con este comprobante y la entrega a la coordinadora del área para que ella lo registre en sus listas de inscripciones y posteriormente le asigne un grupo y un horario. 

Cabe destacar que el CELEX actualmente no cuenta con ningún sistema independiente de la ofimática que le apoye para la administración de sus inscripciones y grupos. Su gestión se basa en el uso de hojas de cálculo en Excel.

\subsubsection{Área de Fotocopiado}
Esta área se basa en un solo concepto de pago del que se derivan algunas consideraciones para la oferta de todos sus servicios. El concepto de pago y su costo se muestran en la tabla \ref{tabla:Fotocopiado}.\\

\begin{table}[htbp]
	\begin{center}
		\begin{tabular}{|p{110mm}|p{25mm}|}
			\hline
			{\bf Producto/Servicio} &  {\bf Monto} \\
			\hline
			Impresiones láser t/carta (equivalente a 15 impresiones) & \$ 9.28\\ \hline			
		\end{tabular}
		\caption{Catálogo de servicios de fotocopiado.}
		\label{tabla:Fotocopiado}
	\end{center}
\end{table}

Los servicios proporcionados por esta área se concentran en las impresiones a blanco y negro, a color, copias y ploteos. El costo para cada uno de ellos como bien lo apreciamos en la tabla parte de un solo concepto de pago equivalente a 15 impresiones.  Estas equivalencias entre el n\'umero de impresiones disponibles y las necesarias para poder utilizar alguno de estos servicios son definidas semestre tras semestre por el departamento de Recursos Financieros. Hasta este momento durante el semestre vigente 2018-2 las tabulaciones están estipuladas de la siguiente manera:
\begin{itemize}
	\item 1 impresi\'on = \$0.62 = 1 impresi\'on o copia en blanco y negro.
	\item 5 impresiones = \$3.083 = 1 impresi\'on o copia a color.
	\item 15 impresiones = \$9.25 = 1 impresi\'on o copia doble carta o 1 impresión 1/4  plotter.
	\item 30 impresiones = \$18.50 = 1 impresi\'on en 1/4 plotter.
	\item 60 impresiones = \$37.00 = 1 impresi\'on en plotter completo
\item O m\'as en m\'ultiplos de 15 impresiones.
\end{itemize} 

El proceso de pago comienza cuando el usuario acude a la caja de la ESCOM para realizar su pago en efectivo equivalente al número de impresiones deseadas. El cajero emite el comprobante SIG@ y realiza la impresión de la misma en dos ocasiones, una de ellas se la otorga al usuario para que haga válido su pago en el área y otra la deja a su resguardo. 
El usuario con esa boleta se presenta en el área de fotocopiado y solicita cualquiera de los servicios que ya mencionamos.\\

Una vez efectuado el servicio, el encargado del área solicita el comprobante SIG@ al usuario y la intercambia por un ticket de impresiones, cada vez que un usuario hace una impresión el encargado representa con un símbolo en este papel el número de impresiones usadas hasta que estas se terminan y si el usuario necesita más tiene que empezar todo el proceso nuevamente.

Este ticket de impresiones será válido únicamente por el semestre en curso y es responsabilidad totalmente del usuario la conservación del mismo. Si éste se extravía se tendrá que realizar de nuevo el pago a pesar de que se pudieron haber tenido todavía impresiones disponibles.

\subsubsection{Área de Biblioteca}
Los conceptos de pago que en esta área se consideran son únicamente dos y se muestran en la tabla \ref{tabla:Biblioteca}.\\

\begin{table}[htbp]
	\begin{center}
		\begin{tabular}{|p{110mm}|p{25mm}|}
			\hline
			{\bf Producto/Servicio} &  {\bf Monto} \\
			\hline
			Multa de biblioteca & \$ 6.50\\ \hline	
			Reposición de credencial de biblioteca & \$ 26.00\\ \hline		
		\end{tabular}
		\caption{Catálogo de servicios de fotocopiado.}
		\label{tabla:Biblioteca}
	\end{center}
\end{table}

El proceso de pago comienza con la asistencia del usuario a la Biblioteca de la ESCOM, ahí, el encargado del área le dará una nota de pago sin ningún formato con el concepto y la cantidad a pagar. El usuario acude a la caja de la ESCOM con dicha nota y efectúa el pago.\\

El cajero por su parte, emite el comprobante SIG@ y realiza la impresión del mismo en dos ocasiones, uno se lo entrega al usuario y el otro lo resguarda. Hecho lo anterior, el usuario regresa a la biblioteca y presenta el comprobante que le fue dado en caja para que sea liberada la multa o bien, la reposición de la credencial de la biblioteca.\\

Si el usuario en otra ocasión presenta una multa en su historial o pierde de nuevo la credencial tendrá que realizar el procedimiento nuevamente.\\

Es importante recordar que todos los comprobantes de pago emitidos por el SIG@ deberán de ser guardados por un periodo de cinco años tanto en el departamento de Recursos Financieros como en cada una de las áreas. Esto por normas establecidas por el IPN.

\cfinput{PlanteamientoDelProblema}

\section{Marco Teórico}
Para la realización de este Trabajo Terminal consideramos varios aspectos técnicos que deben ser definidos adecuadamente. Todos estos son resultado del análisis del problema que se detalla en la siguiente sección.\\

Comenzaremos por definir lo que es una aplicación web y una aplicación móvil, así como los tipos de aplicaciones móviles que existen. Definimos la arquitectura del sistema para la organización de nuestro código y las plataformas de desarrollo que utilizaremos en conjunto con nuestra arquitectura. También, describimos la tecnología que utilizaremos para la responsividad de nuestro sistema y por último, definimos la base de datos y la herramienta de Mapeo Objeto-Relacional (ORM) que utilizaremos para el manejo de la información.\\

\subsection{Aplicación Web}
Una aplicación web (web-based application) es un tipo especial de aplicación cliente/servidor, donde tanto el cliente (el navegador, explorador o visualizador) como
el servidor (el servidor web) y el protocolo mediante el que se comunican (HTTP)
están estandarizados y no han de ser creados por el programador de aplicaciones.\\

El protocolo HTTP forma parte de la familia de protocolos de comunicaciones TCP/IP, que son los empleados en Internet. Estos protocolos permiten la conexión de sistemas heterogéneos, lo que facilita el intercambio de información entre distintos ordenadores.\\

\begin{itemize}
	\item {\bf El cliente: }El cliente web es un programa con el que interacciona el usuario para solicitar a un servidor web el envío de los recursos que desea obtener mediante HTTP.
	\item {\bf El servidor: }El servidor web es un programa que está esperando permanentemente las solicitudes	de conexión mediante el protocolo HTTP por parte de los clientes web. \cite{aplicacionWeb}
\end{itemize}

Algunos de los usos principales de una aplicación web son los siguientes:
\begin{itemize}
	\item Permitir a los usuarios localizar información de forma rápida y sencilla en un sitio Web en el que se almacena gran cantidad de contenido.
	\item Recoger, guardar y analizar datos suministrados por los visitantes de los sitios.
	\item Actualizar sitios Web cuyo contenido cambia constantemente.
\end{itemize}

\subsection{Aplicación móvil}
Una aplicación móvil es una aplicación informática desarrollada para ser ejecutada en teléfonos inteligentes, tabletas y otros dispositivos móviles.\\

Por lo general, se encuentran disponibles a través de plataformas de distribución, operadas por las compañías propietarias de los sistemas operativos móviles como Android, iOS, BlackBerry OS, Windows Phone, entre otros. Existen aplicaciones móviles gratuitas u otras de pago, donde en promedio el 20 a 30\% del costo de la aplicación se destina al distribuidor y el resto es para el desarrollador.\\

Existen tres tipos de aplicaciones móviles que se pueden desarrollar:\\

\begin{itemize}
	\item Aplicaciones Híbridas.
	\item Aplicaciones Web sobre móviles.
	\item Aplicaciones móviles nativas.
\end{itemize}

Cada uno de estos tipos de aplicaciones se explicarán a detalle.\\

\subsubsection{Aplicaciones Web sobre Móviles}
Las aplicaciones web sobre móviles son aplicaciones que no necesitan ser instaladas en el dispositivo para poder ejecutarse. Están basadas en tecnologías HTML, CSS y Javascript, que se ejecutan en un navegador \cite{webMovil}.\\

Una de las principales ventajas de una aplicación Web es su soporte para múltiples plataformas y el bajo costo de desarrollo.
La mayoría de los proveedores móviles utilizan el mismo motor de búsqueda en sus navegadores, llamado WebKit, que es un proyecto de fuente abierta conducido principalmente por Google y Apple y que ofrece la más completa implementación de HTML5 disponible en la actualidad \cite{movilHibrida}.\\

\subsubsection{Aplicación móvil nativa}
Las aplicaciones nativas tienen archivos ejecutables binarios que se descargan directamente al dispositivo y se almacenan localmente.  La manera más común de descargar una aplicación nativa es visitando una tienda de aplicaciones, como App Store de Apple, PlayStore de Android o App World de BlackBerry, pero existen otros métodos
que a veces ofrece el proveedor móvil. La aplicación nativa puede acceder libremente a todas las APIs que el proveedor del Sistema Operativo ponga a disposición y, en muchos casos, tiene características y funciones únicas que son típicas de ese sistema operativo móvil en particular.\\

Para crear una aplicación nativa, los desarrolladores deben escribir el código fuente y crear recursos adicionales, como imágenes, segmentos de audio y diversos archivos de declaración específicos del sistema operativo. Utilizando herramientas provistas por el distribuidor del sistema operativo, se compila el código fuente para crear un ejecutable en formato binario que se pueda empaquetar junto con el resto de los recursos y estar listo para la distribución.\\

Si bien el proceso de desarrollo suele ser similar para diferentes sistemas operativos, el Software Development Kit (SDK) es específico de la plataforma, y cada sistema operativo móvil viene con sus propias herramientas. En la tabla \ref{tabla:SDK} se muestran los formatos de aplicación para cada sistema operativo.\\

\begin{table}[htbp]
	\begin{center}
		\begin{tabular}{p{25mm}|p{25mm}|p{25mm}|p{30mm}}
			\hline
			{\bf Apple iOS} &  {\bf Android} & {\bf Blackberry OS} & {\bf Windows Phone} \\
			\hline
			.app & .apk & .cod &.xap\\ \hline		
		\end{tabular}
		\caption{Formatos de aplicación.}
		\label{tabla:SDK}
	\end{center}
\end{table}

Estas diferencias entre plataformas ocasionan una de las desventajas más criticas del enfoque de desarrollo nativo: el código escrito para una plataforma móvil no se puede usar en otra, por lo cual el desarrollo y el mantenimiento de aplicaciones nativas para múltiples sistemas operativos se convierte en una tarea muy ardua y costosa \cite{movilHibrida}.

Para el desarrollo móvil de nuestro proyecto se planteo usar el sistema operativo de Android por utilizar el mismo lenguaje de desarrollo que nuestra aplicación web. Como vemos, la fusión entre un desarrollo web y un desarrollo móvil es posible, lo que nos permite realizar un sistema con cierta robustez, además de ahorrarnos tiempo de desarrollo.

\subsubsection{Aplicación móvil híbrida}
El enfoque híbrido combina desarrollo nativo con tecnología Web. Usando este enfoque, los desarrolladores escriben gran parte de su aplicación en tecnologías Web para múltiples plataformas, y mantienen el acceso directo a APIs nativas cuando lo necesitan.\\

El enfoque híbrido ofrece un término medio que, en muchas situaciones, constituye lo mejor de ambos mundos, en especial si el desarrollador desea emplearlo en múltiples sistemas operativos \cite{movilHibrida}.\\

El mayor problema de este tipo de aplicación es que para el desarrollo se utiliza un lenguaje de programación especifico y al implementarse métodos muy precisos en la programación es necesario utilizar el lenguaje de desarrollo nativo que nos brinda la plataforma móvil.

Ahora bien, el desarrollo de esas aplicaciones implica una forma de organizar nuestro código para definir la arquitectura del sistema. Entre las principales arquitecturas tenemos las siguientes: arquitectura cliente/servidor, 3 capas y Modelo Vista Controlador (MVC). Para nuestro sistema hemos decidido el uso de una arquitectura MVC por las ventajas que nos brinda respecto a nuestro desarrollo. Esta arquitectura se explica a continuación:\\

\subsection{Modelo Vista Controlador (MVC)}
MVC por sus siglas en inglés, es un patrón de diseño de arquitectura de software usado principalmente en aplicaciones que manejan gran cantidad de datos y transacciones complejas donde se requiere una mejor separación de conceptos para que el desarrollo esté estructurado de una mejor manera, facilitando la programación en diferentes capas de manera paralela e independiente. MVC sugiere la separación del software en 3 estratos: Modelo, Vista y Controlador \cite{MVC}.\\

\begin{itemize}
	\item {\bf Modelo: }Es la representación de la información que maneja la aplicación.
	\item {\bf Vista: }Es la representación del modelo en forma gráfica disponible para la interacción con el usuario.
	\item {\bf Controlador: }Es la capa encargada de manejar y responder las solicitudes del usuario, procesando la información necesaria y modificando el Modelo en caso de ser necesario.
\end{itemize}

Este modelo de arquitectura presenta varias ventajas que nos gustaría recalcar:
\begin{itemize}
	\item Separación clara entre los componentes de un programa; lo cual permite su implementación por separado.
	\item Conexión dinámica entre el Modelo y sus Vistas; se produce en tiempo de ejecución, no en tiempo de compilación.
	\item Las modificaciones a las vistas no afectan al modelo, modificando sólo la representación de la información, no su tratamiento.
	\item Vuelve a la aplicación mucho más mantenible, extensible y modificable. 
\end{itemize}

Es por esas ventajas que esta arquitectura se acopla a nuestro sistema permitiéndonos esa separación por capas que hará más fácil su mantenibilidad y escalabilidad. Es importante mencionar, que esta arquitectura será la misma para ambas aplicaciones.\\

Teniendo nuestra arquitectura definida, podremos explicar las plataformas de desarrollo que utilizaremos. Para la aplicación web decidimos utilizar Spring Framework en conjunto con Struts 2, ambas plataformas son desarrolladas en lenguaje de programación Java. Para la aplicación móvil híbrida optamos por utilizar la plataforma Android la cual nos va a permitir reutilizar el código Java en un 60 a 80\%. Estas plataformas se explican a continuación:\\

\subsection{Spring Framework}
Spring es un framework de Java de código abierto. Spring proporciona un modelo de programación y configuración completa para las aplicaciones modernas basadas en Java. Un elemento clave de Spring es el apoyo infraestructural a nivel de aplicación: Spring se centra en la infraestructura de las aplicaciones para que los equipos puedan centrarse en la lógica de negocios a nivel de aplicación, sin ataduras innecesarias a los entornos de implementación específicos. \cite{Spring}\\ 

Se recomienda que se use una herramienta de compilación que admita la administración de dependencias (como Maven o Gradle).\\

Nosotros utilizamos la dependencia Apache Maven la cual es una herramienta de gestión y comprensión de proyectos de software. Basándose en el concepto de un modelo de objeto de proyecto (POM), Maven puede gestionar la compilación, los informes y la documentación de un proyecto a partir de una pieza central de información \cite{mavenApache}.\\

\subsection{Struts 2}
El framework web de Apache Struts es una solución gratuita de código abierto para la creación de aplicaciones web en Java.\\

Favorece la convención sobre la configuración, es extensible usando una arquitectura de complemento y se envía con complementos para admitir REST, AJAX y JSON. \cite{struts2}\\

Ambos frameworks están diseñados para una arquitectura MVC, pero el complemento de cada uno de ellos hace mucho más sencillo el desarrollo de nuestro sistema. Struts 2, por ejemplo, nos da la oportunidad de trabajar con su REST Plugin el cual nos proporciona soporte de alto nivel para la implementación de aplicaciones web basadas en recursos RESTful. La funcionalidad principal del complemento REST reside en la interpretación de las URL de las solicitudes entrantes según las reglas RESTful. \\

Ahora, describiremos la plataforma de desarrollo Andorid para la aplicación móvil.\\

%\subsection{Xamarin}

%Xamarin es una plataforma que nos  proporciona herramientas que pueden ayudarnos a crear aplicaciones móviles multiplataforma. Las aplicaciones pueden tener todas las características nativas y también compartir la base de código común al mismo tiempo.\\

%Xamarin permite llamar al código existente escrito en otros lenguajes específicos de la plataforma, como Java en Android. Pero eso es solo cuando se está construyendo algo muy específico que no se puede implementar en diferentes plataformas. Xamarin ha convertido todo el SDK de Android e iOS a C\# para que se pueda codificar en un lenguaje más familiar. Y como se puede usar C\# para codificar ambas plataformas, se necesita recordar menos sintaxis. Se Puede acceder a casi cualquier API de iOS o Android en C\# con las herramientas de Xamarin. [28]\\

%Esto nos brinda la oportunidad de reutilizar nuestro código Java realizado en la aplicación web para poderlo convertir en una aplicación móvil. Si bien, el lenguaje de la plataforma es C\#, no implica mayor problema puesto que la sintaxis es muy parecida a la de Java. 

\subsection{Android}

Android ofrece un completo framework de aplicaciones que te permite crear apps y juegos innovadores para dispositivos móviles en un entorno de lenguaje Java. Esto nos brinda la oportunidad de reutilizar nuestro código Java realizado en la aplicación web para poderlo convertir en una aplicación móvil.\\

Android proporciona un framework de apps adaptable que te permite ofrecer recursos exclusivos para diferentes configuraciones de dispositivos. Por ejemplo, puedes crear diferentes archivos de diseño XML para diferentes tamaños de pantalla y el sistema determina qué diseño aplicar en función del tamaño de pantalla del dispositivo actual \cite{android}.\\

Además, nuestro diseño de la aplicación web será responsivo lo que nos permite renderizar las vistas en cualquier dispositivo móvil sin perder su simetría.\\

Para lograr tener un diseño responsivo se tuvieron que considerar distintas tecnologías. Se consideraron las siguientes opciones: Bootstrap, Foundation 3 y HTML5 Boilerplate. Todos ellos son frameworks front-end enfocados en la responsividad de las aplicaciones para que se adapten a cualquier dispositivo.\\

En nuestro caso, la decisión que se tomó fue la de utilizar Bootstrap considerando nuestra experiencia de desarrollo. A continuación, se describe más a detalle este framework de diseño.\\

\subsection{Bootstrap}
Bootstrap es un kit de herramientas de código abierto para desarrollar con HTML, CSS y Javascript. Este código se tiene que incluir dentro del proyecto que se esté desarrollando para poder hacer referencia a éste durante la creación de alguna vista.\\

La descarga del código fuente de Bootstrap incluye los activos precompilados de CSS y JavaScript, junto con la fuente Sas. En esos archivos es donde tiene definidas todas sus clases de estilo las cuales se invocan desde las etiquetas del código HTML \cite{bootstrap}.\\

Hasta este punto, conocemos la definición técnica de las tecnologías que utilizaremos en nuestro sistema en lo que respecta al tipo de desarrollo, la arquitectura del sistema, las plataformas de desarrollo y las tecnologías de diseño para la responsividad. Hace falta definir técnicamente las tecnologías usadas para nuestro acceso a datos.\\

La decisión respecto a la base de datos se hizo en función de la experiencia de cada uno de nosotros y de la factibilidad de uso, pues se plantearon tres propuestas de inicio. Se pensó en una base de datos Oracle, MySQL y PostgreSQL, las tres son bases de datos relacionales que soportan correctamente transacciones de grandes cantidades de datos. Sin embargo, Oracle en específico genera un costo por su licencia de uso lo cual no la hace viable para nuestro sistema, en cambio MySQL y PostgreSQL son bases de datos de código abierto evitando así un costo por su uso.\\

Por tanto, teniendo dos bases de datos candidatas basamos nuestra decisión en la expertiz de cada uno de nosotros optando por utilizar la base de datos PostgreSQL. Esta se detalla a continuación:\\

\subsection{PostgreSQL}
PostgreSQL es un sistema de base de datos objeto-relacional potente y abierto que utiliza y amplía el lenguaje SQL combinado con muchas características que almacenan y escalan de forma segura las cargas de trabajo de datos más complicadas.\\

PostgreSQL se ha ganado una sólida reputación por su arquitectura comprobada, confiabilidad, integridad de datos, sólido conjunto de características, extensibilidad y la dedicación de la comunidad de código abierto detrás del software para entregar constantemente soluciones eficaces e innovadoras. PostgreSQL se ejecuta en todos los principales sistemas operativos \cite{postgresql}. \\

Teniendo en cuenta que nuestro sistema estará desarrollado en Windows, no tendremos problema con la utilización de este sistema de base de datos.\\

Lo último que nos queda por mencionar es la herramienta ORM que utilizaremos para el manejo de la información a través de nuestra aplicación. Esto nos va a permitir mapear nuestras relaciones de base de datos en objetos que podremos manipular en código Java haciendo posible la persistencia de la información. La tecnología elegida fue Hibernate ORM.\\

\subsection{Hibernate ORM}
Hibernate ORM permite a los desarrolladores escribir aplicaciones con mayor facilidad, cuyos datos sobreviven al proceso de solicitud. Hibernate se preocupa por la persistencia de los datos tal como se aplica a las bases de datos relacionales (a través de JDBC).\\

Hibernate es una herramienta de Mapeo Objeto-Relacional (ORM) para la plataforma Java que facilita el mapeo de atributos entre una base de datos relacional tradicional y el modelo de objetos de una aplicación, mediante archivos declarativos (XML) o anotaciones en los beans de las entidades que permiten establecer este tipo de relaciones \cite{hibernate}.\\

Bajo estos conceptos nuestro Trabajo Terminal fue desarrollado. En secciones posteriores veremos cómo es que cada uno de estos conceptos tienen una relación directa con el sistema final.\\

\cfinput{objetivoGeneral}
\cfinput{Alcance}



