% Chapter 2

\chapter{Antecedentes}
\label{Capitulo 2} 

En este capítulo se da a conocer un contexto de la problemática que tratamos de resolver mediante el desarrollo del presente Trabajo Terminal. Por tal motivo, es que hablamos sobre el actual proceso de pagos en la ESCOM y las problemáticas existentes derivadas de esto.\\

Se habla de sistemas y aplicaciones que tienen una funcionalidad parecida a nuestro proyecto, detallando sus características, ventajas y desventajas. Así también, se plantea el problema a resolver y los objetivos que buscamos cumplir para lograr una solución, misma que describimos mediante nuestra propuesta de trabajo. 
 


\section{Situación Actual}

Para tener una idea clara del problema que se ataca, es necesario dar un contexto sobre todo lo que se ve involucrado en el proceso que se lleva a cabo actualmente en la ESCOM para realizar un pago, con el fin de obtener un servicio o producto.\\

Estos servicios o productos de los que hablamos desde el principio del documento forman parte de una estrategia por parte de la ESCOM para poder adquirir recursos económicos independientes del presupuesto federal asignado, con la finalidad de dar manutención a las instalaciones, tales como, laboratorios de cómputo, aulas de clase, sanitarios, zonas verdes, entre otros.\\

Lo anterior, no sólo lo realiza la ESCOM, también las demás escuelas que conforman a todo el IPN, pero cada una de estas definen qué productos o servicios pondrán a disposición de sus estudiantes, empleados o externos. Todos esos productos o servicios se encuentran en un catálogo de cuotas de productos y cuotas de aprovechamiento de Cobro Regular emitido por la Secretaría de Hacienda y Crédito Público (SHCP) a través del sistema DEPAMIN: "Módulo de Estimación de Ingresos por Concepto de Derechos, Productos y Aprovechamientos" [1].\\

Dicho catálogo se renueva año con año presentando una relación aproximadamente de 1073 cuotas de productos y 19 cuotas de aprovechamientos de Cobro Regular [2]. Cada una de esas cuotas es incorporada en el SIG@, poniendo a disposición una boleta de pago, relacionada con cada cuota.\\

ESCOM, en lo particular toma sólo algunos de esos conceptos descritos en el catálogo y los ofrece a toda su comunidad, incluyendo también a externos. Suele tener una clasificación para cada servicio o producto que oferta, estas clasificaciones son las siguientes:\\

\begin{itemize}
	\item Cursos de educación continua
	\item Posgrado
	\item CELEX
	\item Odontológicos
	\item Productos 
\end{itemize}

En total son 36 conceptos de pago los que ESCOM considera para la oferta de servicios o productos repartidos en cada una de las clasificaciones anteriores. Los fondos recabados mediante el pago en caja, pago en sucursal bancaria o transferencia electrónica son depositados en una cuenta de débito de BBVA Bancomer, misma que el IPN monitorea para la declaración de ingresos, por lo tanto, es necesario que el departamento de Recursos Financieros de la ESCOM tenga siempre disponibles todas las boletas de pago emitidas por el SIG@ durante cada periodo escolar.\\

Tan es así, que se exige conservarlas por un periodo de cinco años a partir del año en que se expide. Esta regla no sólo aplica para el departamento de Recursos Financieros, también  para cada una de las áreas de donde derivan la prestación de los productos o servicios. Para ser más específicos, mencionaremos las áreas involucradas en cada una de las clasificaciones que se comentaron anteriormente.\\

\begin{itemize}
	\item {\bf Cursos de educación continua: }Coordinación de Educación Continua (CEC) [3].
	\item {\bf Posgrado: }Sección de Estudios de Posgrado e Investigación (SEPI) [4].
	\item {\bf CELEX: }Coordinación de Cursos Extracurriculares de Lenguas Extranjeras [5].
	\item {\bf Odontológicos: }Servicio Dental.
	\item {\bf Productos: }Fotocopiado y Biblioteca [6].
\end{itemize}

Si bien comentamos que esas boletas de pago se deben de almacenar por un periodo de cinco años, hasta el momento, ninguna de las áreas tiene claro qué es lo que sucede con toda esa documentación una vez finalizado dicho periodo. Actualmente, sólo las almacenan en carpetas y se conservan en los estantes.\\

Todos los ingresos obtenidos durante el día deberán de comprobarse, puesto que se genera un corte de caja de manera diaria que permite saber el total generado, que a su vez, es notificado al Subdirector Administrativo quien será el último en dar el visto bueno para los ingresos obtenidos del día corriente.

\subsection{Pagos en ESCOM}
El proceso de pago en ESCOM se lleva prácticamente de la misma manera en todos los servicios o productos que se ofertan, partiendo de la realización del pago en caja, sucursal bancaria o transferencia electrónica y concluyendo en la entrega al interesado por el producto o servicio de una boleta de pago emitida por el SIG@ la cual se da posteriormente al área involucrada como garantía de que se efectúo correctamente la cuota. Pueden existir algunas variantes en este proceso, pero no afectan para nada el flujo principal comentado anteriormente.\\

Enfocados en el alcance del proyecto daremos el detalle del proceso de pago para cada una de las áreas que estamos considerando, además de mostrar su catálogo de servicios disponibles y sus costos para este año en curso (2018). Esto se ha escrito derivado de múltiples entrevistas con las distintas áreas. Cabe mencionar que para efectos de este proyecto se entiende por usuario a todo aquel que se interese por algún producto o servicio.\\

\subsubsection{Servicio Dental}
Comencemos con el área de servicios dentales, sus costos se muestran en la tabla \ref{tabla:Dentales}.\\

\begin{table}[htbp]
	\begin{center}
		\begin{tabular}{|p{110mm}|p{25mm}|}
			\hline
			{\bf Producto/Servicio} &  {\bf Monto} \\
			\hline
			Servicio Médico de Aplicación de Amalgama. Estudiantes del IPN & \$ 74.00\\ \hline
			Servicio Médico de Aplicación de Amalgama. Público en General & \$ 142.00\\ \hline
			Servicio Médico de Aplicación de Amalgama. Trabajadores del IPN & \$ 101.00\\ \hline
			Servicio Médico de Cementación Incrustación. Estudiantes del IPN y Comunidad del IPN & \$ 68.00\\ \hline
			Servicio Médico de Consulta de Odontología. Comunidad del IPN & \$ 27.00\\ \hline
			Servicio Médico de Curación. Estudiantes del IPN & \$ 27.00\\ \hline
			Servicio Médico de Curación. Comunidad del IPN & \$ 74.00\\ \hline
			Servicio Médico de Curación. Trabajadores & \$ 47.00\\ \hline
			Servicio Médico de Curación. Trabajadores & \$ 59.00\\ \hline
			Servicio Médico de Extracción de piezas dentales. Docentes y empleados & \$ 66.00\\ \hline
			Servicio Médico de Extracción de piezas dentales. Docentes y empleados & \$ 180.00\\ \hline
			Servicio Médico de Impresión Parcial (hilo retractor, yeso velmix, hule, rectificador, alginato, yeso piedra y cera) & \$ 81.00\\ \hline
			Servicio Médico de Profilaxis. Estudiantes del IPN & \$ 108.00\\ \hline
			Servicio Médico de Profilaxis. Comunidad Politécnica & \$163.00\\ \hline
			Servicio Médico de Profilaxis. Público en General & \$ 196.00\\ \hline
			Servicio Médico de Resina autopolimerizable. Estudiantes del IPN & \$ 155.00\\ \hline
			Servicio Médico de Resina autopolimerizable. Docentes y empleados & \$ 175.00\\ \hline
			Servicio Médico de Resina chica & \$ 41.00\\ \hline
			Servicio Médico de Resina. Comunidad Politécnica & \$ 172.00\\ \hline		
		\end{tabular}
		\caption{Catálogo de servicios dentales.}
		\label{tabla:Dentales}
	\end{center}
\end{table}

En este caso, cuando un usuario requiere de algún servicio odontológico se presenta en el consultorio sin cita previa para la atención con algún dentista en turno, después, éste hace un análisis del problema del paciente, sin previo pago realiza el servicio diagnosticado o solicitado para finalmente realizar una nota de pago que le entrega al usuario la cual debe de presentar a caja para efectuar su pago en efectivo. Una vez realizado, el cajero genera e imprime en dos ocasiones el comprobante del SIG@ para después entregarle uno de ellos al usuario y otro para su resguardo. Este comprobante es el que el usuario entrega al dentista para confirmar su pago, mismo que almacena por políticas de la institución. Si el paciente necesita más servicios odontológicos el proceso se repite.\\

\subsubsection{CELEX}
Ahora bien, hablemos del área de CELEX, mencionando primero sus costos. Estos se muestran en la tabla \ref{tabla:Celex}.\\
\newpage
\begin{table}[htbp]
	\begin{center}
		\begin{tabular}{|p{110mm}|p{25mm}|}
			\hline
			{\bf Producto/Servicio} &  {\bf Monto} \\
			\hline
			Curso de Idiomas semanal o sabatino, 40 horas. Comunidad IPN & \$ 544.00\\ \hline
			Curso de Idiomas semanal o sabatino, 40 horas. Público en General & \$ 1,053.00\\ \hline			
		\end{tabular}
	\caption{Catálogo de servicios CELEX.}
\label{tabla:Celex}
\end{center}
\end{table}

Para esta área, el proceso de pago es muy similar, el usuario tiene que efectuar el pago directamente en cualquier sucursal bancaria BBVA Bancomer a la cuenta que previamente se le otorgó acudiendo a las oficinas del CELEX. Es importante decir, que también se puede efectuar el pago a través de una transferencia electrónica hacia el mismo banco. Aquí, no son permitidos los pagos directamente en la caja de la ESCOM.\\

Una vez efectuado el pago el usuario debe de acudir a la caja de la ESCOM para presentar el voucher emitido por el banco, o bien, la captura de pantalla o impresión que compruebe la transferencia electrónica. Hecho eso, el cajero emite la boleta de pago del SIG@ y la imprime en dos ocasiones, una la entrega al usuario como garantía de su pago y otra la deja para su resguardo. El usuario acude a las oficinas del CELEX con esa boleta y la entrega a la coordinadora del área para que ella lo registre en sus listas de inscripciones y posteriormente le asigne un grupo y un horario. 

Cabe destacar que el CELEX actualmente no cuenta con ningún sistema independiente de la ofimática que le apoye para la administración de sus inscripciones y grupos. Su gestión se basa en el uso de hojas de cálculo en Excel.

\subsubsection{Área de Fotocopiado}
Esta área se basa en un solo concepto de pago del que se derivan algunas consideraciones para la oferta de todos sus servicios. El concepto de pago y su costo se muestran en la tabla \ref{tabla:Fotocopiado}.\\

\begin{table}[htbp]
	\begin{center}
		\begin{tabular}{|p{110mm}|p{25mm}|}
			\hline
			{\bf Producto/Servicio} &  {\bf Monto} \\
			\hline
			Impresiones láser t/carta (equivalente a 15 impresiones) & \$ 9.28\\ \hline			
		\end{tabular}
		\caption{Catálogo de servicios de fotocopiado.}
		\label{tabla:Fotocopiado}
	\end{center}
\end{table}

Los servicios proporcionados por esta área se concentran en las impresiones a blanco y negro, a color, copias y ploteos. El costo para cada uno de ellos como bien lo apreciamos en la tabla parte de un solo concepto de pago equivalente a 15 impresiones.  Estas equivalencias entre el n\'umero de impresiones disponibles y las necesarias para poder utilizar alguno de estos servicios son definidas semestre tras semestre por el departamento de Recursos Financieros. Hasta este momento durante el semestre vigente 2018-2 las tabulaciones están estipuladas de la siguiente manera:
\begin{itemize}
	\item 1 impresi\'on = \$0.62 = 1 impresi\'on o copia en blanco y negro.
	\item 5 impresiones = \$3.083 = 1 impresi\'on o copia a color.
	\item 15 impresiones = \$9.25 = 1 impresi\'on o copia doble carta o 1 impresión 1/4  plotter.
	\item 30 impresiones = \$18.50 = 1 impresi\'on en 1/4 plotter.
	\item 60 impresiones = \$37.00 = 1 impresi\'on en plotter completo
\item O m\'as en m\'ultiplos de 15 impresiones.
\end{itemize} 

El proceso de pago comienza cuando el usuario acude a la caja de la ESCOM para realizar su pago en efectivo equivalente al número de impresiones deseadas. El cajero emite la boleta de pago del SIG@ y realiza la impresión de la misma en dos ocasiones, una de ellas se la otorga al usuario para que haga válido su pago en el área y otra la deja a su resguardo. 
El usuario con esa boleta se presenta en el área de fotocopiado y solicita cualquiera de los servicios que ya mencionamos.\\

Una vez efectuado el servicio, el encargado del área solicita la boleta de pago al usuario y la intercambia por un pequeño papel que entre otros datos contiene el número de impresiones por el que es válido, cada vez que un usuario hace una impresión el encargado representa con un símbolo en este papel el número de impresiones usadas hasta que estas se terminan y si el usuario necesita más tiene que empezar todo el proceso nuevamente.

Este papel del que hablamos, será válido únicamente por el semestre en curso y es responsabilidad totalmente del usuario la conservación del mismo. Si éste se extravía se tendrá que realizar de nuevo el pago a pesar de que se pudieron haber tenido todavía impresiones disponibles.

\subsubsection{Área de Biblioteca}
Los conceptos de pago que en esta área se consideran son únicamente dos y se muestran en la tabla \ref{tabla:Biblioteca}.\\

\begin{table}[htbp]
	\begin{center}
		\begin{tabular}{|p{110mm}|p{25mm}|}
			\hline
			{\bf Producto/Servicio} &  {\bf Monto} \\
			\hline
			Multa de biblioteca & \$ 6.50\\ \hline	
			Reposición de credencial de biblioteca & \$ 26.00\\ \hline		
		\end{tabular}
		\caption{Catálogo de servicios de fotocopiado.}
		\label{tabla:Biblioteca}
	\end{center}
\end{table}

El proceso de pago comienza con la asistencia del usuario a la Biblioteca de la ESCOM, ahí, el encargado del área le dará una nota de pago sin ningún formato con el concepto y la cantidad a pagar. El usuario acude a la caja de la ESCOM con dicha nota y efectúa el pago.\\

El cajero por su parte, emite el comprobante del SIG@ y realiza la impresión del mismo en dos ocasiones, uno se lo entrega al usuario y el otro lo resguarda. Hecho lo anterior, el usuario regresa a la biblioteca y presenta el comprobante que le fue dado en caja para que sea liberada la multa o bien, la credencial de la biblioteca.\\

Si el usuario en otra ocasión presenta una multa en su historial o pierde de nuevo la credencial tendrá que realizar el procedimiento nuevamente.\\

Es importante recordar que todos los comprobantes de pago emitidos por el SIG@ deberán de ser guardados por un periodo de cinco años tanto en el departamento de Recursos Financieros como en cada una de las áreas. Esto por reglas de negocio del mismo IPN.

\section{Estado del arte}

Durante nuestra investigaci\'on de mercado, nos encontramos con sistemas que tienen una fuerte relaci\'on con el proyecto a trabajar, pero implementados en situaciones distintas, adem\'as de tener funcionalidades variadas. Si bien, actualmente existen muchas aplicaciones o sistemas enfocados en la gestión de pagos, la gran mayoría funcionan como desarrollos independientes de un contexto dejando de lado la integración a futuro de más módulos o incluso sistemas. Es por ello, que nosotros buscamos desarrollar un sistema que permita la escalabilidad a favor de la gestión de procesos tomando como punto de partida justamente la gestión de pagos.\\

Los sistemas con mayor similitud a nuestro desarrollo los podemos encontrar en la tabla \ref{tabla:Productos}
\begin{table}[htbp]
	\begin{center}
		\begin{tabular}{|p{30mm}|p{75mm}|p{25mm}|}
			\hline
			{\bf Software} & {\bf Caracter\'isticas} & {\bf Costo} \\
			\hline 
			"SISTEMA DE MONEDERO VIRTUAL PARA PAGOS ESCOLARES" & El Sistema de Monedero Virtual para Pagos Escolares es un sistema de prepago, que permite a los alumnos realizar pagos dentro y fuera de la Unidad Profesional Interdisciplinaria en Ingenier\'ia y Tecnolog\'ias Avanzadas (UPIITA) [7] & Sin costo \\ \hline
			
			TTR-12-1-029 Prototipo para el manejo de “Cero Papel”. & Es un sistema que permite el manejo, intercambio y control de la informaci\'on dentro de una organizaci\'on para optimizar los procedimientos y tareas, disminuyendo el uso de papel mediante la implementaci\'on de un sistema que permita administrar los usuarios y los documentos [8] & Sin costo\\ \hline
			
			“Ventanilla Virtual UdeG” & Ventanilla Virtual tiene tecnolog\'ias desarrolladas por universitarios y busca brindar una plataforma para que, en una primera etapa, los estudiantes puedan hacer tr\'amites, dar seguimiento, recuperarlos y hacer pagos respectivos [9] & Sin costo \\ \hline
			
			“Campus Pay” & Desarrollo que ofrece a estudiantes la posibilidad de realizar todos los pagos relacionados con sus estudios desde dispositivos m\'oviles con cualquier tarjeta de cr\'edito o d\'ebito [10][11] & Sin costo \\ \hline
			
			“School control” & Aplicaci\'on m\'ovil que realiza reportes, asignaci\'on de pagos, consulta de estad\'isticas en tiempo real, monitoreo/edici\'on de la informaci\'on escolar, entrega de calificaciones, control de asistencia y comportamiento de los alumnos entre otras soluciones dise\~nadas espec\'ificamente para colegios que les deja tiempo valioso para dedicarlo a lo que verdaderamente saben hacer, que es enseñar [12][13]& desde \$149 por alumno al a\~no, costo absorbido totalmente por el colegio. \\ \hline
			
			“Aplicaci\'on escolar” & Es una aplicaci\'on m\'ovil para dispositivos Android e IOS en la cual las escuelas pueden enviar informaci\'on como mensajes de pagos, tareas, circulares, así como seguimientos acad\'emicos y calificaciones graficadas directo al celular de los padres o alumnos con notificación tipo WhatsApp. Los pap\'as descargan la aplicaci\'on con el nombre de su colegio desde Play Store o App Store ya que es personalizada a cada escuela [14][15]& Pago inicial de \$31,000, posterior a la prueba piloto se cobran \$6,000 mensuales por cada 400 alumnos. El costo es absorbido por el colegio. \\ \hline		
		\end{tabular}
		\caption{Sistemas o aplicaciones relacionadas}
		\label{tabla:Productos}
	\end{center}
\end{table}

\newpage
De estos sistemas encontramos las siguientes ventajas y desventajas considerando el contexto bajo el cual nosotros estaremos trabajando:
\begin{itemize}
	\item {\bf SISTEMA DE MONEDERO VIRTUAL PARA PAGOS ESCOLARES}
	\begin{itemize}
		\item Ventajas:
		\begin{itemize}
			\item Permite la realización de pagos de forma electrónica.
		\end{itemize}
		\item Desventajas:
		\begin{itemize}
			\item Carece de un diseño responsivo.
			\item Para realizar el pago es necesario imprimir un comprobante.
			\item Se necesita efectuar un abono previo.
			\item Sólo se planteó como un prototipo.
		\end{itemize}
	\end{itemize}
	\item {\bf TTR-12-1-029 Prototipo para el manejo de Cero Papel}
	\begin{itemize}
		\item Ventajas:
		\begin{itemize}
			\item Disminución del uso de papel.
			\item Control y seguridad de documentos.
		\end{itemize}
		\item Desventajas:
		\begin{itemize}
			\item Solo es un sistema que busca la optimización de recurso material (papel).
		\end{itemize}
	\end{itemize}
	\item {\bf Ventanilla Virtual UdeG}
	\begin{itemize}
		\item Ventajas:
		\begin{itemize}
			\item Presenta distintos medios de acceso como kioscos interactivos, sitio web y aplicación móvil.
			\item Pueden realizar, seguir, recuperar y generar los pagos en algunos trámites.
			\item Permite al alumno la consulta de información académica.
			\item Pago digital.
		\end{itemize}
		\item Desventajas:
		\begin{itemize}
			\item El precio de cada kiosco interactivo es de 165 mil pesos.
			\item Solo los estudiantes tienen acceso.
		\end{itemize}	
	\end{itemize}
	\item {\bf Campus Pay}
	\begin{itemize}
		\item Ventajas:
		\begin{itemize}
			\item Permite a toda la comunidad universitaria realizar pagos de forma electrónica.
			\item Permite el pago a todas las áreas que lo requieran.
			\item Permite al alumno la consulta de información académica.
			\item No tiene costo para los usuarios de la aplicación.
			\item Permite cualquier tarjeta de débito o crédito.
		\end{itemize}
		\item Desventajas:
		\begin{itemize}
			\item Genera un costo en la institución educativa en la que se implementa.
		\end{itemize}
	\end{itemize}	
	\item {\bf School control}
	\begin{itemize}
		\item Ventajas:
		\begin{itemize}
			\item Permite el pago en línea.
			\item Elimina las comisiones de tarjeta de crédito.
			\item Realiza métricas de información sobre el colegio.
			\item Permite una administración de accesos al sistema.
			\item Tiene un módulo de apoyo para maestros.
		\end{itemize}
		\item Desventajas:
		\begin{itemize}
			\item Tiene un costo básico de \$149.00 por alumno.
		\end{itemize}
	\end{itemize}	
	
	\item {\bf Aplicación escolar}
	\begin{itemize}
		\item Ventajas:
		\begin{itemize}
			\item Permite mensajes de pagos.
			\item Permite seguimientos académicos.
			\item Genera notificaciones.
			\item Aplicación personalizada por institución.
			\item Tiene un módulo de apoyo para maestros.
		\end{itemize}
		\item Desventajas:
		\begin{itemize}
			\item Pago inicial de \$31,000, posterior a la prueba piloto se cobran \$6,000 mensuales por cada 400 alumnos.
		\end{itemize}			
	\end{itemize}
\end{itemize}

Contextualizando todos estos sistemas y aplicaciones a nuestro trabajo, nos percatamos que la {\bf Ventanilla Virutal UdeG} y la aplicación móvil {\bf Campus pay} son los modelos de referencia que tenemos para el desarrollo y mejora de nuestro proyecto. Entendemos que son bastante funcionales, pero también creemos que se encuentran carentes de escalabilidad e inclusión hacia otros sistemas para la gestión de procesos. 

\section{Planteamiento del problema}
A pesar de que dentro de los departamentos de servicios ya se cuenta con un equipo de computo y conexión a Internet, en ninguna \'area se tiene un sistema que ayude a la administraci\'on de ingresos o gesti\'on de servicios, obligando al personal a tener que realizar todos sus procesos de gesti\'on manualmente o con programas de su mayor entendimiento.

Aunado a que existen distintas aplicaciones de administraci\'on, en los departamentos de CELEX, Biblioteca, Dentales e Impresiones, no se tiene un sistema que comunique directamente con el servicio de caja para confirmar un pago, y solo mantienen una comunicaci\'on por medio de comprobantes impresos. Obligando al usuario a presentar un rol de mediador entre estas 2 áreas, esta obligación al usuario lo compromete a tener que estar presencialmente en todo servicio que requiera.

 Estos comprobantes en realidad presentan un gasto inecesario de recursos, ya que en todas las áreas de servicios solo los almacenan por razones de normatividad, el cual al cumplir con el tiempo establecido son desechados, y solo el área administrativa es la que da un uso especifico a estos comprobantes.

En caso de los usuarios la problematica es muy similar, después de realizar un pago existen situaciones por la cual no hacen entrega del comprobante de pago y nunca llegan a efectuar por perdida de este. Además en caso de área de impresines se presenta la situación de la entrega de otro comprobante que sirve para realizar posteriores impresiones, el cual también llega a ser extraviado.

Si bien el número de impresiones de estos comprobantes no es múy grande por alumno, si lo es en proporción de toda la comunidad, ya que en un día normal se pueden llegar realizar aproximadamente 50 impresiones, y en días finales de parcial imprimer hasta 200 hojas.



\section{Metodología}
	Para el desarrollo del sistema utilizaremos la metodologia incremental de Harlan Mills, ésta se basa en la idea de diseñar una implementación inicial, exponerla al comentario del usuario, y luego desarrollarla en sus diversas versiones hasta producir un sistema adecuado. %===========================Referenciar---------------
	%Libro de ingeneria de software somerville 
	
	Se tomo en cuenta esta metodología por los siguientes beneficios:
	\begin{itemize}
		\item Permite descomponer el proyecto en varios incrementos aislados.
		\item En cada incremento se incorporan los requisitos básicos. 
		\item Es posible realizar un trabajo en paralelo por parte de los integrantes del equipo.
		\item Es sencillo obtener retroalimentación de los directores y coordinadores de área.
	\end{itemize}

\section{Objetivo general}
Desarrollar una aplicaci\'on m\'ovil sobre el sistema operativo Android en conjunto con una aplicaci\'on web para permitir el seguimiento de los pagos realizados en el departamento de recursos financieros de la ESCOM, limitándose al pago de multas de biblioteca, reposici\'on de credencial de biblioteca, servicio de impresiones, dentales y CELEX, con la finalidad de optimizar en tiempo, recursos materiales y espacios f\'isicos en el proceso entre los departamentos involucrados y alumnos.

\section{Objetivos particulares}
\begin{itemize}
	\item Optimizar recurso material, espec\'ificamente papel durante el proceso de pago de alg\'un servicio, mediante la 				digitalizaci\'on de documentos.
	\item Optimizar el espacio f\'isico de los departamentos involucrados por medio de la reducci\'on de comprobantes de pago, notas de pago e historiales.
	\item Agilizar el proceso de seguimiento a pagos tanto para el alumno, como para las \'areas involucradas.
	\item Desarrollar una herramienta de reporteo derivada del historial de servicios pagados, que les permita a los distintos 				  departamentos llevar a cabo toma de decisiones.
	\item Permitir el acceso a la aplicaci\'on web a aquellos alumnos carentes de tel\'efono inteligente Android para hacer uso de 			  las funciones b\'asicas de este sistema (consulta de servicios, historial y servicios por efectuar).
	\item Brindar una herramienta tecnol\'ogica escalable para futuros desarrollos.
\end{itemize}

\section{Justificaci\'on}
Hoy en d\'ia, la ESCOM ofrece distintos servicios a su comunidad a cambio de un pago efectuado en el departamento de recursos financieros con el objetivo de contar con ingresos auto generados para el continuo desarrollo de la instituci\'on. Ejemplos de este tipo de ingreso son: pago de multas de biblioteca, reposici\'on de credencial de biblioteca, servicio de impresiones, dentales y CELEX. Estos servicios requieren de un procedimiento post pago que involucra una gran demanda de recurso material y espacio f\'isico, refiriéndonos con esto al alto consumo de papel al momento de la impresi\'on de boletas de pago y/o notas de pago, as\'i como su almacenamiento, debiendo de estar guardadas por un periodo de cinco años por motivos fiscales, las cuales una vez transcurrido este periodo pasan a ser parte de un archivo muerto. Adem\'as, a esto se suma el tiempo que le toma a la comunidad realizar este proceso, puesto que una vez realizado el pago se tiene que esperar a la entrega de un comprobante f\'isico, el cual se otorgar\'a al \'area correspondiente (biblioteca, centro de impresiones, CELEX, servicio dental) con el fin de comprobar el pago. En todos los casos este comprobante sirve como garant\'ia para la prestaci\'on de dicho servicio, lo que implica una obligaci\'on personal para el alumno el realizar una copia del mismo, la cual en muchas de las ocasiones por motivos de tiempo no la efectuamos, quedándonos de esta manera sin un amparo ante cualquier complicaci\'on que surja derivado del pago del servicio.\\

Es por eso que desarrollaremos una aplicaci\'on m\'ovil que permitirá\'a dar  seguimiento a los pagos efectuados en la ESCOM. De este modo, pretendemos optimizar parte del proceso manual, dejando de lado la impresi\'on de notas de pago al menos para la entrega al alumno y permitiéndoles conservar un comprobante de pago de forma permanente.\\

Adem\'as de crear un medio de interacci\'on web para los proveedores de los servicios, con lo cual se pueda almacenar y administrar la informaci\'on sobre el alumno, su pago y su estatus en el departamento.\\

As\'i, se buscar\'a agilizar los procesos derivados de un pago en caja, de manera que al ser realizados se sustituya el papel del comprobante y se genere un archivo digital que ayudar\'ia a ahorrar recursos materiales (papel y t\'oner), espacios f\'isicos y tiempo. Permitiendo tambi\'en, una interacci\'on con los distintos departamentos involucrados (sala de impresiones, biblioteca, CELEX ESCOM, servicios dentales) para un mejor seguimiento y comunicaci\'on con el alumnado.

\section{Marco Teórico}

\subsection{Pagina WEB}
Aunque los inicios de Internet se remontan a los años sesenta, no ha sido hasta los años noventa cuando, gracias a la Web, se ha extendido su uso por todo el mundo. En pocos años la Web ha evolucionado enormemente: se ha pasado de páginas sencillas, con pocas imágenes y contenidos estáticos a páginas complejas con contenidos dinámicos que provienen de bases de datos, lo que permite la creación de "aplicaciones web". De forma breve, una aplicación web se puede definir como una aplicación en la cual un usuario por medio de un navegador realiza peticiones a una aplicación remota accesible a través de Internet (o a través de una intranet) y que recibe una respuesta que se muestra en el propio navegador. %===========================Referenciar---------------
%http://rua.ua.es/dspace/handle/10045/16995
\subsection{Aplicaciones Web sobre Móviles}
Las aplicaciones web sobre móviles son aplicaciones que no necesitan ser instaladas en el dispositivo para poder ejecutase. Están basadas en tecnologías HTML, CSS y Javascript, y que se ejecutan en un navegador. A diferencia de las web móviles, cuyo objetivo básico es mostrar información, estas aplicaciones tienen como objetivo interaccionar con el dispositivo y con el usuario. De esta manera, se le saca un mayor partido a la contextualización. %===========================Referenciar---------------
%https://www.exabyteinformatica.com/uoc/Informatica/Tecnologia_y_desarrollo_en_dispositivos_moviles/Tecnologia_y_desarrollo_en_dispositivos_moviles_(Modulo_4).pdf


\section{Descripci\'on de la propuesta}

\subsection{Alcance del proyecto}
El sistema de “Escomunidad-Servicios” descrito en esta propuesta cumplir\'a con los siguientes requerimientos.
\begin{itemize}
	\item Los administrativos en \'areas de servicios podr\'an visualizar y gestionar los pagos que reciban de caja para realizar un 		  servicio
	\item Los administrativos en \'areas de biblioteca y dentales podr\'an mandar una nota digital de pago a los usuarios.
	\item El contador y el encargado de recursos financieros podr\'an visualizar todos los conceptos de pago e imprimirlos en caso de ser necesarios
	\item El personal de caja podr\'a validar dos tipos de pago, en efectivo y por medio de un voucher de pago.
	\item El personal de caja podr\'a visualizar, aceptar o rechazar los voucher de pago.
	\item El alumno podr\'a visualizar los servicios disponibles y sus precios desde una pagina web o aplicaci\'on m\'ovil
	\item El alumno podr\'a seleccionar entre realiar un pago por transferencia o mandar una nota de pago a caja para realizar el 				  pago en efectivo.
	\item El alumno podr\'a agendar citas de servicio con el \'area de dentales.
	\item Los alumnos podr\'an escoger el metodo de pago que realizen, por transferencia o efectivo.
\end{itemize}

\subsection{Interección con el usuario}
En nuestra arquitectura de sistema es necesario una comunicación entre varios elementos que al trabajar en conjunto permitan un correcto funcionamiento. Entre estos elementos estará una base de datos para la persistencia de información, un servidor web para el alojamiento de la pagina.\\ 
\IUfig[1]{gui/a}{}{Arquitectura}
\newpage

\subsection{Spring Boot}
Spring Boot es un framework que se ha empleado en este trabajo terminal debido a que su arquitectura se adapta a los requerimientos de este sistema, tiene ciertas ventajas que con ayuda de otras herramientas como Apache Maven nos han ayudado a levantar una infraestructura de manera relativamente sencilla, los módulos más reelevantes utilizados para este trabajo son: el módulo de mails, el módulo de tratamiento de datos. \\

Otro de los enfoques utilizados para el desarrollo de este trabajo es la inyección de dependencias, el poder inyectar dependencias dentro del proyecto permite dividir el proyecto de forma más natural en MVC (modelo, vista, controlador), se ha incluido derivado de ello el uso de anotaciónes para incluir referencias a servicios en capas más internas del sistema, así en conjunto con las tecnologías como Hibernate se ha podido contruir una infraestructura estable. \\

\subsection{Struts 2}
Una de las ventajas que encontramos en usar Struts 2 es la fácil inclusión de controladores con la vista, el presente sistema utiliza una arquitectura mvc, cabe mencionar que entre otras tecnologías utiliza para maquetar pantallas Bootstrap esto en conjunto permite crear vistas complejas con una complejidad baja y al mismo tiempo gracias a Struts2 icluir funcionalidades más robustas.\\

Las acciones de Struts 2 implementan objetos JavaBeans (clases Java simples) para cada grupo de datos enviado en la consulta. Cada parámetro de la consulta se declara en la clase de acción con un nombre idéntico para realizar automáticamente la asignación de valores. La nalidad de la acción es devolver una cadena de caracteres, permitiendo seleccionar el resultado que se va a mostrar.\\

\subsection{Hibernate}
Hibernate es una herramienta de Mapeo Objeto-Relacional (Object-Relational Mapping) para la plataforma JAVA que facilita el mapeo de atributos entre una base de datos relacional tradicional y el modelo de objetos de una aplicación, mediante archivos declarativos (XML) o anotaciones en los beans de las entidades que permiten establecer este tipo de relaciones.
Cuando desarrollamos aplicaciones en muchos de los casos ocurre que en muchas secciones todo termina siendo un conjunto de ABM (alta, baja y modicaciones de datos) que luego consultamos. Para ello se utiliza una base de datos donde hay muchas tareas repetidas: por cada objeto que quiero persistir debo crear una clase que me permita insertarlo, eliminarlo, modificarlo y consultarlo. Con excepción de consultas especiales, el resto es siempre lo mismo.
La solución que tenemos ante esto es usar un ORM para poder eficientar las tareas y reducir todos los pasos que se mencionaron antes. Con solo congurar correctamente los archivos usados por Hibernate todas estas tareas se ejecutarían automáticamente y sólo tendremos que preocuparnos por las consultas especiales.\\

\subsection{Xamarin}
	Xamarin es un entorno de desarrollo para aplicaciones nativas en multiples sistemas operativos móviles, permitiendo una sutil integración de Android y IOS en un solo código de programación.
	