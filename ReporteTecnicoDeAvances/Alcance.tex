\subsection{Alcance del proyecto}
El sistema de “Escomunidad-Servicios” descrito en esta propuesta cumplir\'a con los siguientes requerimientos de usuario.\\

\begin{itemize}
	\item El alumno, empleado o externo podrá crear una cuenta de usuario para el uso del sistema.
	\item Los coordinadores de las \'areas de servicios podr\'an visualizar y gestionar los comprobantes SIG@ que reciban de caja para realizar un servicio.
	\item Los coordinadores de las \'areas de biblioteca y dentales podr\'an mandar una nota digital de pago a los usuarios.
	\item El contador podr\'a visualizar todos los pagos recibidos.
	\item Los coordinadores de área, contador y subdirector administrativo podrá imprimir los comprobantes SIG@ en caso de ser necesario.
	\item El cajero podr\'a validar dos tipos de pago, en efectivo y por medio de un voucher de pago.
	\item El cajero podr\'a visualizar el voucher de pago del usuario. 
	\item El cajero podrá aceptar o rechazar los voucher de pago del usuario.
	\item El usuario podr\'a visualizar el catálogo de servicios disponibles en el área de CELEX, biblioteca, fotocopiado y servicios dentales.
	\item El usuario podr\'a adjuntar su comprobante de pago para enviarlo a caja.
	\item El usuario podr\'a agendar citas con el área de servicios dentales.
	\item El cajero podrá adjuntar el comprobante SIG@.
	\item Las áreas de servicios dispondrán de un archivo de pagos con los recibos de pago de los últimos cinco años.
	\item El usuario podrá recibir notificaciones de confirmación o rechazo de sus pagos.
	\item El cajero recibirá notificaciones cada vez que reciba un voucher de pago.
\end{itemize}

Todos estos requerimientos se desarrollarán  considerando solo a las áreas de CELEX, servicios dentales, biblioteca y fotocopiado.\\ 

Así, este sistema está pensado para ser un desarrollo web y móvil que en conjunto brindarán una solución para la gestión del proceso de pagos en la ESCOM. Si bien, el cumplimiento de los objetivos recaen en gran parte en el desarrollo web, se planea realizar una aplicación móvil haciendo más accesible nuestra aplicación.